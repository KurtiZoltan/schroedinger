\documentclass[pdftex,12pt,a4paper]{article}
\pdfpagewidth 8.5in
\pdfpageheight 11.6in
\linespread{1.3}
\usepackage{anysize}
\marginsize{3.5cm}{2.5cm}{2.5cm}{2.5cm}

\usepackage[utf8]{inputenc}
\usepackage[T1]{fontenc}
\usepackage[magyar]{babel}
\usepackage{amsmath}
\usepackage{float}
\usepackage{graphicx}
\usepackage{braket}

\newcommand{\Ai}[1]{\mathrm{Ai}\left(#1\right)}
\newcommand{\Bi}[1]{\mathrm{Bi}\left(#1\right)}
\newcommand{\Aip}[1]{\mathrm{Ai}^\prime\left(#1\right)}
\newcommand{\Bip}[1]{\mathrm{Bi}^\prime\left(#1\right)}
\newcommand{\Ti}[1]{\mathrm{Ti}\left(#1\right)}
\newcommand{\op}[1]{\hat{#1}}
\newcommand{\norm}[1]{\left\lVert#1\right\rVert}

\frenchspacing
\begin{document}
\numberwithin{equation}{section}
\numberwithin{figure}{section}
\numberwithin{table}{section}
\addtolength{\marginparwidth}{50pt}

\pagenumbering{roman}
\label{cimlap}
\thispagestyle{empty} 

\null\vskip-0.8truein
\centerline{\Large\sc Szakdolgozat}\vskip0.6truein

\centerline{\bf\LARGE Falak közé zárt kvantum részecske homogén térben:}\vskip0.15truein
\centerline{\bf\LARGE "Schrödinger macskája dobozban"}

\vskip0.4truein\centerline{\Large\sc Kürti Zoltán}\vskip0.10truein
\centerline{\Large\sl Fizika BSc., fizikus szakirány}\vskip0.06truein
%\centerline{\Large\sl III. évfolyam }\vskip0.3truein


%\centerline{\psfig{file= ./fig/elte.pdf}}
%\centerline{\psfig{file=../fig/elte.ps}}
%\centerline{\psfig{file=./fig/elte_cimer_szines.jpg}}
\centerline{\includegraphics[scale=0.5]{./figs/elte_cimer_color.pdf}}
\vskip0.4truein
\centerline{\Large Témavezetők:}\vskip0.2truein
\centerline{\Large{\sc{ Dr. Cserti József}} }\vskip0.001truein
\centerline{egyetemi tanár}\vskip0.15truein
\centerline{\Large\sc Dr. Györgyi Géza}\vskip0.001truein
\centerline{egyetemi docens}\vskip0.2truein
\centerline{\Large \sc \bf Eötvös Loránd Tudományegyetem}\vskip0.010truein
\centerline{\Large  Komplex Rendszerek Fizikája Tanszék}\vskip0.15truein
\centerline{\Large\bf 2021}
\newpage

\begin{abstract}
	Kvantummechanikai iskolapélda a homogén térbe helyezett egydimenziós
	részecske. Ezt három dimenzióra kiterjesztve és két fal közé zárva
	keressük az energia sajátállapotokat. Annyi előrelátható, hogy a nyílt
	vagy félig nyílt esetekben használható, reguláris Airy függvény itt nem
	elegendő a megoldáshoz, ennyiben túlmegyünk a tankönyvi feladaton. Az
	aszimptotikus függvényalakok segítségével előállítjuk a magasan
	gerjesztett állapotok energiáit és hullámfüggvényeit, s ezeket
	összehasonlítjuk a közvetlenül a Bohr--Sommerfeld-módszerrel kapott
	eredménnyel. Numerikusan szemléltetjük fizikailag érdekes kezdőállapotok
	időfejlődését. Vizsgáljuk a rezolvenst és az állapotsűrűséget.%, továbbá a sokrészecske rendszerekre való általánosítás lehetőségét.
	
%\centerline{\bf Köszönetnyilvánítás }\vskip0.15truein
	
\end{abstract}

\newpage
%\phantomsection
\tableofcontents
\newpage
\listoffigures
\listoftables
\newpage
\pagenumbering{arabic}
%\phantomsection
\section{Schrödinger}
	A probléma egy 1D doboba zárt résecske homogén erőtérben. $F(x)=-F$, azaz $V(x) = Fx$.
	Az egyenlethez tartozó határfeltételek, ha a doboz hossza $L$:
	\begin{equation}
		\psi \big\rvert_0 = \psi \big \rvert_L = 0
	\end{equation}
	A megoldandó időfüggetlen Schrödinger-egyenlet:
	\begin{equation}
		-\frac{\hbar^2}{2m}\frac{d^2\psi}{dx^2} + Fx\psi = E\psi
	\end{equation}
	\begin{equation}
		\frac{d^2\psi}{dx^2} - \frac{2mFx}{\hbar^2}\psi = -\frac{2mE}{\hbar^2}\psi
	\end{equation}
	\begin{equation}
		\frac{d^2\psi}{dx^2} - \left(\frac{2mF}{\hbar^2}x - \frac{2mE}{\hbar^2}\right)\psi = 0
	\end{equation}
	Az Airy egyenlet ilyen alakra hozható a változó affin lineáris transzformációjával:
	\begin{equation}
		\frac{d^2y}{dx^{\prime 2}} - x^\prime y = 0
	\end{equation}
	$x^\prime = ax - b$, azaz $\frac{d}{dx} = a\frac{d}{dx^\prime}$:
	\begin{equation}
		\frac{d^2y}{dx^2} - \left(a^3x - a^2b\right)y = 0
	\end{equation}
	Az együtthatók összevetése alapján $a = \sqrt[3]{\frac{2mF}{\hbar^2}}$ és $b = \sqrt[3]{\frac{2m}{\hbar^2F^2}}E$. Így a Schrödinger-gyenlet megoldása:
	\begin{equation}
		\psi(x) = y(x^\prime) = y\left(\sqrt[3]{\frac{2mF}{\hbar^2}}x - \sqrt[3]{\frac{2m}{\hbar^2F^2}}E\right)
	\end{equation}
	, ahol $y(x) = \alpha \Ai{x} + \beta \Bi{x}$.
	A $\psi \big\rvert_0 = 0$ feltételből következik, hogy $\psi \propto \Bi{-b}\Ai{ax-b} - \Ai{-b}\Bi{ax-b}$. A második határfeltétel pedig meghatározza a lehetséges energiákat. A feltétel:
	\begin{equation}
		\Bi{-b}\Ai{aL-b} - \Ai{-b}\Bi{aL-b} = 0
	\end{equation}
	\begin{equation}
		\label{box_energiaszintek_egyenlet}
		\Ti{aL-b} - \Ti{-b} = 0
	\end{equation}
	\begin{equation}
		\Ti{\sqrt[3]{\frac{2mF}{\hbar^2}}L - \sqrt[3]{\frac{2m}{\hbar^2F^2}}E} - \Ti{-\sqrt[3]{\frac{2m}{\hbar^2F^2}}E} = 0
	\end{equation}
	\begin{figure}[H]
		\includegraphics[scale=1]{./figs/energiaszintek.pdf}
		\caption{Energiaszintek $L$ függvényében}
		\label{box_energiaszintek_abra}
	\end{figure}
	Amikor $FL \ll \frac{\pi^2\hbar^2}{2mL^2}$, a potenciál jól közelíthető konstans potenciállal, mivel az alapállapot energiájához képest is elhanyagolható a lineáris potenciál eltérése a konstans potenciáltól. Eben a esetben $E \propto n^2$. $E \ll FL$ esetben az energiaszintek jó közelítéssel konstanssá válnak. Ennek az oka, hogy $\lim_{L \to \infty}\psi(x) = \alpha \Ai{ax-b}$, mert a $Bi(x)$ exponenciálisan növekszik nagy $x$-ek esetén. Ebben az eseten az energiaszinteket a $\Ai{- \sqrt[3]{\frac{2m}{\hbar^2F^2}}E} = 0$ egyenlet határozza meg. Ezeket az aszimptotikus viselkedéseket \aref{box_energiaszintek_abra}. ábra jól mutatja.
    
    TODO: link 1D videóról


\section{Kvantumos közelítése}
	$x \rightarrow \infty$ aszimptotikus alak:
	\begin{equation}
		\Ai{-x} = \frac{1}{\sqrt{\pi}x^{1/4}}\cos\left(\frac{2}{3}x^{3/2} - \frac{\pi}{4}\right) + \mathcal{O}\left(x^{-5/4}\right)
	\end{equation}
	\begin{equation}
		\Bi{-x} = -\frac{1}{\sqrt{\pi}x^{1/4}}\sin\left(\frac{2}{3}x^{3/2} - \frac{\pi}{4}\right) + \mathcal{O}\left(x^{-5/4}\right)
	\end{equation}
	\begin{equation}
		\Ti{-x} = -\cot\left( \frac{2}{3}x^{3/2} - \frac{\pi}{4} \right) + \mathcal{O}\left(x^{-5/4}\right)
	\end{equation}
	
	Ezzel a közelítéssel \aref{box_energiaszintek_egyenlet}. egyenlet alakja:
	\begin{equation}
		\cot\left(\frac{2}{3}\left(b-aL\right)^{3/2} - \frac{\pi}{4}\right) = \cot\left(\frac{2}{3}b^{3/2} - \frac{\pi}{4}\right)
	\end{equation}
	, azaz
	\begin{equation}
		\frac{2}{3}b^{3/2} - \frac{2}{3}\left(b-aL\right)^{3/2} = n\pi
	\end{equation}
	. Az $a$ és $b$ behelyettesítésével az egyenlet
	\begin{equation}
		\frac{2\sqrt{2m}}{3F\hbar}\left(E^{3/2} - \left(E - FL\right)^{3/2}\right) = n\pi
	\end{equation}
	Ez megegyezik a szemiklasszikus kvantálással kapott eredménnyel, ami azt jelenti, hogy a szemiklasszikus közelítés jól működik nagy energiáknál, hibája $\mathcal{O}\left(E^{-5/4}\right)$ nagyságrendű.


\section{Szemiklasszikus}
	\begin{equation}
		nh = \oint p \, dq = 
	\end{equation}
	$E/F < L$ esete:
	\begin{equation}
		2\int_0^{E/F}\sqrt{2m\left( E-Fx \right)}\,dx = -\frac{2}{3mF}\left(2m\left( E-Fx \right)\right)^{\frac{3}{2}}\bigg \rvert_0^{E/F} = \frac{4\sqrt{2m}E^{3/2}}{3F}
	\end{equation}
	\begin{equation}
		E_n = \left(\frac{3nhF}{4\sqrt{2m}}\right)^{2/3}
	\end{equation}
	$E/F > L$ esete:
	\begin{equation}
		-\frac{2}{3mF}\left(2m\left( E-Fx \right)\right)^{\frac{3}{2}}\bigg \rvert_0^{L} = \frac{4\sqrt{2m}}{3F}\left(E^{3/2} - \left(E - FL\right)^{3/2}\right) = nh
	\end{equation}
	$E \gg FL$ esetén a különbség az $E^{3/2}$ függvény deriváltjának segítségével helyettesíthető:
	\begin{equation}
		nh \approx 2\sqrt{2m}E^{1/2}L
	\end{equation}
	\begin{equation}
		E_n \approx \frac{n^2h^2}{8mL^2}
	\end{equation}
	
	\begin{figure}[H]
		\includegraphics[scale=1]{./figs/energiaszintkozelites.pdf}
		\caption[Szemiklasszikus energiaszintek]{Az ábra a szemiklasszikus energiaszinteket hasonlítja össze az egzakt energiaszintekkel. Ez az ábra is a $bE$ és $aL$ közötti relációt ábrázolja. A szemiklasszikus közelítés nagy kvantumszámok illetve $E \gg FL$ esetén pontos. Utóbbi oka, hogy ebben az esetben a potenciál elhanyagolható, és a potenciál nélküli végtelen potenciálgödör energiaszintjeit pedig a szemiklasszikus közelítés egzaktul megadja.}
	\end{figure}
	
	\begin{figure}[H]
		\includegraphics[scale=1]{./figs/infsquareenergia.pdf}
		\caption[Végtelen potenciálgödör energiaszintjei]{Az ábrán a végtelen potenciálgödör és az egzakt energiaszintek összehasonlítása látható. Ez csak az $E \gg FL$ esetben jó közelítés, a szemiklasszikus energiaszintek jóval pontosabbak.}
	\end{figure}


\section{Momentumok}
	\input{tex/momentum.tex}
	
\section{Plafon érintés 1D}
	Azokat a parmaétereket keresem, ahol az alapállapot $E = FL$:
	\begin{equation}
		\Ti{\sqrt[3]{\frac{2mF}{\hbar^2}}L - \sqrt[3]{\frac{2m}{\hbar^2F^2}}FL} - \Ti{-\sqrt[3]{\frac{2m}{\hbar^2F^2}}FL} = 0
	\end{equation}
	, azaz
	\begin{equation}
		\Ai{-\sqrt[3]{\frac{2mF}{\hbar^2}}L} = 0
	\end{equation}
	. Az első gyöke az Airy függvénynek megadja azt az esetet, amikor az alapállapot energiája $FL$, és nem pedig valamelyik gerjesztett állapoté.
	\begin{equation}
		-a_1 = \sqrt[3]{\frac{2mF}{\hbar^2}}L \approx 2.338
	\end{equation}



\section{3D doboz, ferde tér}
    A rendszer egy téglatest alakú dobozba zárt részecske. A doboz mérete $L_x$, $L_y$ és $L_z$. A dobozban homogén erőtér hat a részecskére, azaz $\boldsymbol{F} = \text{const}$. A potenciál így $V(x, y, z) = -\boldsymbol{F}_xx-\boldsymbol{F}_yy-\boldsymbol{F}_zz$. A rendszer időfüggő Schrödinger-egyenlete
\begin{equation}
	i\hbar\frac{\partial \psi(x, y, z, t)}{\partial t} = -\frac{\hbar^2}{2m} \Laplace \psi(x, y, z, t) + V(x, y, z)\psi(x, y, z, t).
	\label{3dbox:3dscheq}
\end{equation}
Az egyenlet kezdőfeltétele egy kezdeti állapot $t_0$-ban, $\psi(x, y, z, t_0) = \psi_0(x, y, z)$, az egyenlet határfeltételei pedig a hullámfüggvény határokon való eltűnése, $0=\left.\psi\right|_{x=0}=\left.\psi\right|_{x=L_x}=\left.\psi\right|_{y=0}=\left.\psi\right|_{y=L_y}=\left.\psi\right|_{z=0}=\left.\psi\right|_{z=L_z}$. Mivel ez a potenciál lineáris $x$, $y$ és $z$-ben, a Schrödinger-egyenlet szeparálható a
\begin{equation}
	\psi_{klm}(x, y, z, t) = e^{-\frac{iE_{klm}t}{\hbar}}\psi^{(x)}_k(x)\psi^{(y)}_l(y)\psi^{(z)}_m(z)
	\label{3dox:3dansatz}
\end{equation}
próbafüggvénnyel. A $\psi^{(i)}_n$ ($i=x, y, z$) függvényekre így az egy dimenziós stacionárius Schrödinger-egyenlet vonatkozik. A $\psi^{(x)}$-re vonatkozó egyenlet 
\begin{equation}
	-\frac{\hbar^2}{2m}\frac{d^2\psi^{(x)}_k(x)}{dx^2} + \boldsymbol{F}_xx\psi^{(x)}_k(x) = E^{(x)}_k\psi^{(x)}_k(x),
\end{equation}
a határfeltételek $0=\left.\psi^{(x)}_k\right|_{x=0}=\left.\psi^{(x)}_k\right|_{x=L_x}$. $\psi^{(y)}_l$ és $\psi^{(z)}_m$-re vonatkozó egyenletek hasonlóak. Az $E_{klm}$ energia a három egy dimenziós stacionárius Schrödinger-egyenlet sajátenergiáinak összege,
\begin{equation}
	E_{klm} = E^{(x)}_k+E^{(y)}_l+E^{(z)}_m.
\end{equation}
\Aeqref{3dbox:3dscheq} egyenlet általános megoldása \aeqref{3dox:3dansatz} próbafüggvények kezdőfeltételhez illesztett lineáris kombinációja,
\begin{equation}
	\psi(x,y,z,t) = \sum_{klm}C_{klm}\psi_{klm}(x,y,z,t).
\end{equation}
$C_{klm}$ együtthatók meghatározásához a szokásos hely reprezentáció beli skalárszorzást kell használni,
\begin{equation}
	C_{klm} = \frac{1}{N_{klm}}\int_0^{L_x}dx\int_0^{L_y}dy\int_0^{L_z}dz\,\psi_{klm}(x, y, z, t=0)^*\psi_0(x, y, z)
	\label{3dbox:ceq}
\end{equation}
\begin{equation}
	N_{klm} = \int_0^{L_x}dx\int_0^{L_y}dy\int_0^{L_z}dz\,|\psi_{klm}(x,y,z,t=0)|^2.
	\label{3dbox:3norm}
\end{equation}
\Aeqref{3dbox:ceq} egyenlet nem egyszerűsíthető tovább általános $\psi_0$ esetén, viszont \aeqref{3dbox:3norm} igen. Mivel $\psi_{klm}$ szorzat alakú, nem kell a tripla integrált elvégezni, hanem csak három egyszeres integrál szorzatát kell kiszámítani. Ez numerikus számításokban jelentős.
\begin{equation}
	N_{klm} = N^{(x)}_kN^{(y)}_lN^{(z)}_m,
\end{equation}
ahol az egyes $N$ tagok az egy dimenziós sajátfüggvények normájaként vannak definiálva.
\begin{equation}
	N^{(x)}_k = \int_0^{L_x}dx\,\left|\psi^{(x)}_k(x)\right|^2,
\end{equation}
$N^{(y)}_l$-re és $N^{(z)}_m$-re hasonló képletek vonatkoznak. 



%    Ahol $x^\prime = \sqrt[3]{\frac{2m\boldsymbol{F}_x}{\hbar^2}}x - \sqrt[3]{\frac{2m}{\hbar^2\boldsymbol{F}_x^2}}E_k$, $E_k$ pedig az 1 dimenziós probléma, $\Ti{\sqrt[3]{\frac{2m\boldsymbol{F}_x}{\hbar^2}}L - \sqrt[3]{\frac{2m}{\hbar^2\boldsymbol{F}_x^2}}E} - \Ti{-\sqrt[3]{\frac{2m}{\hbar^2\boldsymbol{F}_x^2}}E} = 0$, $k$. $\phi_k \left( x^\prime \right)$ az 1D-s rész TODO:REFERENCIA hullámfüggvénye. $y^\prime$, $z^\prime$, valamint $E_l$ és $E_m$ hasonlóan vannak definiálva a hozzájuk tartozó 1 dimenziós probléma alapján. A 3D-s hullámfüggvényhez tartozó energia az 1D-s megoldásokhoz tartozó energiák összege.
%    \begin{equation}
%        E = E_k + E_l + E_m
%    \end{equation}
%    Amennyiben valamelyik irányú komponense $\boldsymbol{F}$-nek 0, abban az esetben a hozzá tartozó 1D-s pprobléma a híres végtelen mély potenciálgödör, ahol
%    \begin{equation}
%        \phi_n = \sqrt{\frac{2}{L}}\sin\left(\frac{nx\pi}{L}\right)
%    \end{equation}
%    valamint
%    \begin{equation}
%        E_n = \frac{\hbar^2n^2}{2mL^2}
%    \end{equation}
%    
%    TODO: ÁBRA AZ EGYSZER FÜGGŐLEGES ESET ENERGIÁIRÓL, esetleg szintén L függvényében.
%    
%    TODO: ábra 2D -quantum chaos in billiards-
%    
%    TODO: 2D (3D?) videó link időfelődésről

    
\section{Green függvény}
	A reolvens operátor definíciója
\begin{equation}
    \op{G}\left( E \right) = \frac{1}{\op{H} - E}
\end{equation}
és ezen operátorhoz tartozó két változós függvény a Green-függény.
\begin{equation}
    G\left( x, y; E \right) = \Bra{x}G\left(E\right)\Ket{y}
\end{equation}
A Green-függvény név indokolt, és ennek a segítségével fogom meghatározni a Green-függvényeket konkrét esetben. A teljességi reláció beszúrásával látható, hogy a kvantummechanikai Green-függény megegyezik a differenciálegyenletek elméletéből ismert Green-függvénnyel.
\begin{equation}
    \left(\op{H} - E\right) \op{G}\left( E \right) = \op{I}
\end{equation}

\begin{equation}
    \int \mathrm{d}x^\prime \Bra{x}\left(\op{H} - E\right) \Ket{x^\prime}\Bra{x^\prime} \op{G}\left( E \right)\Ket{y} = \Bra{x}\op{I}\Ket{y} = \delta \left(x - y\right)
\end{equation} 
A $\Bra{x}\left(\op{H} - E\right) \Ket{x^\prime}$ maggal vett konvolúció a $\op{H} - E$ operátor hatása. Ezért
\begin{equation}
    \left(\op{H}_x - E\right) G\left(x, y; E\right) = \delta\left(x - y\right)
    \label{green:deltaeq}
\end{equation}
ami a differenciálegyenletek elméletéből ismert Green-függvény definíciója. Ebben a konkrét esetben
\begin{equation}
    \left( -\frac{\hbar^2}{2m}\frac{\partial^2}{\partial x^2} + Fx - E \right) G\left(x, y; E\right) = \delta\left(x - y\right)
\end{equation}
ami azt jelenti, hogy az $x < y$ tartományban
\begin{equation}
    G\left(x, y; E\right) = C_1 \Ai{\sqrt[3]{\frac{2mF}{\hbar^2}}x - \sqrt[3]{\frac{2m}{\hbar^2F^2}}E} + C_2 \Bi{\sqrt[3]{\frac{2mF}{\hbar^2}}x - \sqrt[3]{\frac{2m}{\hbar^2F^2}}E}
\end{equation}
illetve az $x > y$ tartományban
\begin{equation}
    G\left(x, y; E\right) = C_3 \Ai{\sqrt[3]{\frac{2mF}{\hbar^2}}x - \sqrt[3]{\frac{2m}{\hbar^2F^2}}E} + C_4 \Bi{\sqrt[3]{\frac{2mF}{\hbar^2}}x - \sqrt[3]{\frac{2m}{\hbar^2F^2}}E}
\end{equation}
, ahol a $C$ együtthatók függhetnek $y$ és $E$ értékétől. A $C$ együtthatók meghatározásához a doboz eredeti határfeltételeit $x = 0$ és $x = L$ pontban, valamint az $x = y$ pontban \aref{green:deltaeq}. egyenlet $y$ körüli integrálásából kapot feltételeket kell felhasználni. 

















    
\section{Videó gyártás leírása}
    
    
    
\end{document}















