Az Airy egyenlet
\begin{equation}
	\frac{d^2y}{dx^2} - xy = 0,
	\label{airy:airyeq}
\end{equation}
ennek az egyenletnek a megfelelő kezdőfeltételekhez illesztett megoldásai az úgynevezett Airy-függvények, $\Ai(x)$ és $\Bi(x)$.

Az Airy-függvények szorosan kapcsolódnak a Bessel-függvényekhez. Ez elentős mind az aszimptotikus alakjuk meghatározásához, mind a függvények numerikus kiértékeléséhez. A megoldást
\begin{equation}
	y(x) = x^{\frac{1}{2}}v\left(\frac{2}{3}x^{\frac{3}{2}}\right)
\end{equation}
alakban keresve a $x \geq 0$ tartományban a $v(x)$-re vonatkozó egyenlet a módosított Bessel-egyenlet $t=\frac{2}{3}x^{\frac{3}{2}}$ bevezetésével.
\begin{equation}
	t^2\frac{d^2v(t)}{dt^2} + t\frac{dv(t)}{dt} - \left(t^2 + \frac{1}{9}\right)v(t) = 0
\end{equation}
Leolvasható, hogy $\nu^2 = \frac{1}{9}$, azaz a $v(x)$-re vonatkozó egyenlet megoldásai az $I_{\frac{1}{3}}(x)$ és $I_{-\frac{1}{3}}(x)$ módosított Bessel-függvények lineáris kombinációi.
A két hagyományosan választott lineáris kombinációk a következőek:
\begin{equation}
	\Ai(x) = \frac{\sqrt{x}}{3}\left(I_{-\frac{1}{3}}\left(\frac{2}{3}x^{\frac{3}{2}}\right)-I_{\frac{1}{3}}\left(\frac{2}{3}x^{\frac{3}{2}}\right)\right)
	\label{airy:ai+}
\end{equation}
\begin{equation}
	\Bi(x) = \sqrt{\frac{x}{3}}\left(I_{-\frac{1}{3}}\left(\frac{2}{3}x^{\frac{3}{2}}\right)+I_{\frac{1}{3}}\left(\frac{2}{3}x^{\frac{3}{2}}\right)\right).
	\label{airy:ai+}
\end{equation}
$x \leq 0$ tartományban
\begin{equation}
	y(x) = (-x)^{\frac{1}{2}}v\left(\frac{2}{3}(-x)^{\frac{3}{2}}\right)
\end{equation}
alakban keresve a megoldást a $v(x)$-re kapott egyenlet a Bessel-egyenlet, megint $\nu^2 = \frac{1}{9}$.
\begin{equation}
	t^2\frac{d^2v(t)}{dt^2} + t\frac{dv(t)}{dt} + \left(t^2 - \frac{1}{9}\right)v(t) = 0
\end{equation}
Az $x=0$ pontban megkövetelt analitikusságnak megfelelően $x \geq 0$ esetén
\begin{equation}
	\Ai(-x) = \frac{\sqrt{x}}{3}\left(J_{-\frac{1}{3}}\left(\frac{2}{3}x^{\frac{3}{2}}\right)-J_{\frac{1}{3}}\left(\frac{2}{3}x^{\frac{3}{2}}\right)\right)
	\label{airy:ai-}
\end{equation}
\begin{equation}
	\Bi(-x) = \sqrt{\frac{x}{3}}\left(J_{-\frac{1}{3}}\left(\frac{2}{3}x^{\frac{3}{2}}\right)+J_{\frac{1}{3}}\left(\frac{2}{3}x^{\frac{3}{2}}\right)\right),
	\label{airy:bi-}
\end{equation}
ahol $J_\nu(x)$ a Bessel-függvények. Érdemes definiálni a
\begin{equation}
	\Ti(x) = \frac{\Ai(x)}{\Bi(x)}
\end{equation}
függvényt.

$x \to \infty$ aszimptotikus alak:
\begin{equation}
	\Ai\left(-x\right) = \frac{1}{\sqrt{\pi}x^{1/4}}\cos\left(\frac{2}{3}x^{3/2} - \frac{\pi}{4}\right) + \mathcal{O}\left(x^{-5/4}\right)
\end{equation}
\begin{equation}
	\Bi\left(-x\right) = -\frac{1}{\sqrt{\pi}x^{1/4}}\sin\left(\frac{2}{3}x^{3/2} - \frac{\pi}{4}\right) + \mathcal{O}\left(x^{-5/4}\right)
\end{equation}
\begin{equation}
	\Ti\left(-x\right) = -\cot\left( \frac{2}{3}x^{3/2} - \frac{\pi}{4} \right) + \mathcal{O}\left(x^{-5/4}\right)
\end{equation}
\begin{equation}
	\Ai(x) = \frac{1}{2\sqrt{\pi}x^{1/4}}e^{-\frac{2}{3}x^{\frac{3}{2}}}+\mathcal{O}\left(x^{-5/4}\right)
\end{equation}
\begin{equation}
	\Bi(x) = \frac{1}{ \sqrt{\pi}x^{1/4}}e^{ \frac{2}{3}x^{\frac{3}{2}}}+\mathcal{O}\left(x^{-5/4}\right)
\end{equation}









