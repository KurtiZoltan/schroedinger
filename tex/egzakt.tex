A Green-függvény név indokolt: a teljességi reláció beszúrásával látható, hogy a kvantummechanikai Green-függény megegyezik a differenciálegyenletek elméletéből ismert Green-függvénnyel.
\begin{equation}
    \left(E-\op{H}\right)\op{G}\left( E \right) = \op{I},
\end{equation}
azaz
\begin{equation}
    \int dx^\prime \Bra{x}\left(E-\op{H}\right) \Ket{x^\prime}\Bra{x^\prime} \op{G}\left( E \right)\Ket{y} = \Bra{x}\op{I}\Ket{y} = \delta \left(x - y\right).
\end{equation} 
A $\Bra{x}\left(E-\op{H}\right) \Ket{x^\prime}$ maggal vett konvolúció az $E-\op{H}$ operátor hatása, ezért
\begin{equation}
    \left(E-\op{H}_x\right) G\left(x, y; E\right) = \delta\left(x - y\right),
\end{equation}
amely a differenciálegyenletek elméletéből ismert Green-függvény definíciója. Ebben a konkrét esetben
\begin{equation}
    \left(E +\frac{\hbar^2}{2m}\frac{\partial^2}{\partial x^2} - Fx \right) G\left(x, y; E\right) = \delta\left(x - y\right),
	\label{green:deltaeq}
\end{equation}
amely azt jelenti, hogy az $x < y$ tartományban, illetve $y < x$ tartományban a Green-függvény a homogén egyenlet megoldása. A homogén megoldások illesztését az eredeti differenciálegyenlet határfeltételei, valamint az $x = y$ pontban \aeqref{green:deltaeq} egyenlet $y$ körüli integrálásából kapott feltételek határozzák meg. A doboz falára vonatkozó határfeltételek
\begin{equation}
	\left. G\left(x,y;E\right)\right\rvert_{x = 0} = 0,
	\label{green:01}
\end{equation}
\begin{equation}
	\left. G\left(x,y;E\right)\right\rvert_{x = L} = 0.
	\label{green:02}
\end{equation}
\Aref{green:deltaeq}. egyenlet $x$ szerinti integrálja $y$ körüli $\epsilon$ sugarú környezetében az $\epsilon \to 0^+$ határesetben
\begin{equation}
	\lim_{\epsilon \to 0^+}\left.\frac{\partial}{\partial x}G\left(x,y;E \right)\right\rvert_{x = y - \epsilon}^{x = y + \epsilon} = \frac{2m}{\hbar^2}.
\end{equation}
Itt a jobb oldal integrálja $\left. \theta\left(x - y\right) \right\rvert_{x = y - \epsilon}^{x = y + \epsilon} = 1$ az előírt határesetben. Mivel $G(x,y;E)$-ről feltesszük, hogy folytonos, a bal oldal integrálja is folytonos, leszámítva a deriváltakat tartalmazó tagokat. A határeset elvégzése közben a deriváltakat nem tartalmazó tagok így kiesnek. \Aref{green:deltaeq}. egyenlet $\int_{y-\epsilon}^{y+\epsilon}dx^\prime \int_{y-\epsilon}^{x^\prime} \,dx$ integrálja az $\epsilon \to 0^+$ határesetben
\begin{equation}
	\lim_{\epsilon \to 0^+}\left.G\left(x,y;E \right)\right\rvert_{x = y - \epsilon}^{x = y + \epsilon} = 0
	\label{green:continuity}
\end{equation}
folytonossági feltételt adja. A jobb oldal integrálja $\left. \left(x - y\right) \theta\left(x - y\right) \right\rvert_{x=y-\epsilon}^{x=y+\epsilon}$, ami a határesetben $0$. Az $\left(Fx - E\right)G\left(x,y;E\right)$ integrálja is $0$ a határesetben, az előző integrálhoz hasonló módon.


%\Aref{green:xy}. és \aref{green:yx}. egyenlet behelyettesítése meghatározza a $C$ együtthatókra vonatkozó egyenleteket:
%\begin{equation}
%	\frac{C_2}{C_1} = -\Ti\left(-bE\right)
%	\label{green:Cbegin}
%\end{equation}
%\begin{equation}
%	\frac{C_4}{C_3} = -\Ti\left(aL - bE\right)
%\end{equation}
%\begin{equation}
%	\frac{C_3}{C_1} = \frac{\Ti\left(ay - bE\right) - \Ti\left(-bE\right)}{\Ti\left(ay - bE\right) - \Ti\left(aL - bE\right)}
%\end{equation}
%TODO: $b$ lecserélése $bE$-re az előző részekben.
%\begin{equation}
%	C_1 = -\frac{2m}{a\hbar^2}\frac{1}{\left( \left(\frac{C_3}{C_1}-1\right)\Aip\left(ay - bE\right) + \left(\frac{C_4}{C_3}\frac{C_3}{C_1} - \frac{C_2}{C_1}\right) \Bip\left(ay - bE\right) \right)}
%	\label{green:Cend}
%\end{equation}
Valós energiákra $G(x,y;E)=G(y,x;E)^*$. Ezt a szimmetria tulajdonságot fel lehet használni a Green-függvényre adott ansatz pontosítására az $x<y$ és $y<x$ $x$-$y$ csere szimmetriájának megkövetelésével. Ez automatikusan kielégíti \aeqref{green:continuity} egyenletet. A tartomány peremén a homogén megoldás eltűnését megkövetelve \aeqref{green:01} és \aeqref{green:02} teljesül. Érdemes bevezetni a
\begin{equation}
	u = ax-bE,
\end{equation}
\begin{equation}
	v = ay-bE
\end{equation}
jelöléseket. A fent leírt három kritériumot és szimmetria tulajdonságot teljesítő ansatz a
\begin{equation}
	G\left(x,y;E\right) = C_0(E)\times
	\begin{cases}
		\begin{split}
			\Bigl(\Ti(aL-bE)\Bi(v)-\Ai(v)\Bigr)\times\\
			\Bigl(\Ti(bE)\Bi(u)-\Ai(u)\Bigr)
		\end{split}& x \leq y\\
		\begin{split}
			\Bigl(\Ti(aL-bE)\Bi(u)-\Ai(u)\Bigr)\times\\
			\Bigl(\Ti(bE)\Bi(v)-\Ai(v)\Bigr)
		\end{split}& x \geq y
	\end{cases}.
\end{equation}
%\begin{equation}
%	C_1 = -\frac{a^2}{F}\frac{1}{\left( \left(\frac{C_3}{C_1}-1\right)\Aip\left(ay - bE\right) + \left(\frac{C_4}{C_3}\frac{C_3}{C_1} - \frac{C_2}{C_1}\right) \Bip\left(ay - bE\right) \right)}
%\end{equation}
%\Aref{green:Cbegin}-\ref{green:Cend}, \ref{green:xy}. és \aref{green:yx}. egyenletek explicit, analitikus módon előállítják a $G\left( x, y; E \right)$ Green-függvényt. Valós energiákra $G\left(x, y; E\right) = G\left(y, x; E\right)^*´$. Ebből következik, hogy a Green-függvény $x<y$ eset $y$ függése kiemelhető lesz, és megegyezik az $x>y$ eset $x$ függésével. Ezek szerint $\Ai\left(ay-bE\right)-\Ti\left(aL-bE\right)\Bi\left(ay-bE\right)$ kiemelhető a $C_1$ együtthatóból,
%\begin{equation}
%	C_1 = \frac{a^2}{F}\frac{\Ai\left(ay-bE\right)-\Ti\left(aL-bE\right)\Bi\left(ay-bE\right)}{\left(\Ti\left(-bE\right)-\Ti\left(aL-bE\right)\right)\left(\Bi\left(-bE\right)\Aip\left(-bE\right)-\Ai\left(-bE\right)\Bip\left(-bE\right)\right)}.
%\end{equation}
%Az algebrai átalakításokon túl fel kellett használni, hogy $\Ai\left(ay-bE\right)\Bip\left(ay-bE\right)-\Bi\left(y-bE\right)\Aip\left(ay-bE\right)$ $y$-tól független konstans tehát $y=0$ helyettesíthető bele. Ez onnan látható, hogy $y$ szerinti deriváltja $0$,
%\begin{dmath}
%	\left(\Ai\left(ay-bE\right)\Bip\left(ay-bE\right)-\Bi\left(ay-bE\right)\Aip\left(ay-bE\right)\right)^\prime = a\Ai^\prime\left(ay-bE\right)\Bi^\prime\left(ay-bE\right) + a\Ai\left(ay-bE\right)\Bi^{\prime\prime}\left(ay-bE\right) - a\Bi^\prime\left(ay-bE\right)\Ai^\prime\left(ay-bE\right) - a\Bi\left(ay-bE\right)\Ai^{\prime\prime}\left(ay-bE\right) = a\Ai\left(ay-bE\right)\left(ay-bE\right)\Bi\left(ay-bE\right) - a\Bi\left(ay-bE\right)\left(ay-bE\right)\Ai\left(ay-bE\right) = 0.
%\end{dmath}
%Ez után már az $x$-$y$ szimmetriája jól látható a Green-függvénynek.
\begin{equation}
	C_0 = \frac{a^2}{F}\frac{\pi}{\Ti(-bE)-\Ti(aL-bE)}
\end{equation}
%bevezetésével a Green függvény egyszerűbb alakra hozható,
%\begin{equation}
%	G\left(x,y;E\right) = C_0\times
%	\begin{cases}
%		\begin{split}
%			\left(\Ai\left(ay-bE\right)-\Ti\left(aL-bE\right)\Bi\left(ay-bE\right)\right)\times\\
%			\left(\Ai\left(ax-bE\right)-\Ti\left(-bE\right)\Bi\left(ax-bE\right)\right)
%		\end{split} & x\leq y\\
%		\begin{split}
%			\left(\Ai\left(ay-bE\right)-\Ti\left(-bE\right)\Bi\left(ay-bE\right)\right)\times\;\;\;\;\;\;\;\;\\
%			\left(\Ai\left(ax-bE\right)-\Ti\left(aL-bE\right)\Bi\left(ax-bE\right)\right)
%		\end{split}& x>y\\
%	\end{cases}
%\end{equation} 

%A rezolvens operátornak pólusai vannak a rendszer $E_k$ sajátenergiáinál:
%\begin{equation}
%	\op{G}\left(E\right) = \sum_n \frac{\Ket{n}\Bra{n}}{E_n - E}
%	\label{green:greensum}
%\end{equation}

Így ha $E$ kielégíti \aref{box_energiaszintek_egyenlet}. egyenletet, akkor a rezolvensnek és ezért a Green-függénynek is pólusa kell hogy legyen. Ezt a $C_1$ szingularitásán lehet a leg könnyebben belátni. Ha $C_1$ szinguláris, az összes többi $C$ együttható is, és így a Green-függvény is. \Aref{box_energiaszintek_egyenlet}. egyenlet szerint a $\frac{C_3}{C_1}$ számlálójának és nevezőének ,ásodik tagjai egyenlőek. Első tagjuk bármely $E$ esetén egyenlő, így  hányadosuk $1$, valamint \aref{box_energiaszintek_egyenlet}. egyenlet esetén $\frac{C_2}{C_1} = \frac{C_4}{C_3}$. Ezeknek a következtében mind $\frac{C_3}{C_1} - 1$, mind $\frac{C_4}{C_3}\frac{C_3}{C_1} - \frac{C_2}{C_1}$ $0$-val egyenlő, így a $C_1$-re vonatkozó kifejezés nevezője $0$. Ezek a $\frac{1}{E_n - E}$ típusú pólusok \aref{green:greensum}. egyenletből.

Egy érdekes matematikai következmény, hogy a Green-függvényre vonatkozó differenciál egyenlet megoldásával elvégeztem \aref{green:greensum}. egyenlet összegzését. Ez az összeg az Airy függvények szorzatának összege lenne, osztva $E_k-E$-vel és a megfelelő normálási faktorral, ami Airy függvények szorzatának $0$ és $L$ közötti integrálj, valamint $E_k$-t \aref{box_energiaszintek_egyenlet}. transzcendens egyenlet határozza meg. A Green-függvényre vonatkozó differenciálegyenlet nélkül az összeg elvégzése reménytelennek látszana.