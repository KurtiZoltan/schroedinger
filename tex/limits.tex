A két falú doboz Green-függvényéből megfelelő határesetekben előállítható más fizikai rendszerek Green-függvénye is. Például az $L\to\infty$ határeset visszaadja a felül nyitott doboz Green-függvényét, avagy a földön pattogó kvantum részecske ("quantum bouncer") Green-függvényét. Egy következő transzformáció határeseteként megkaphatjuk a falak nélküli végtelen lineáris potenciálban mozgó részecske Green-függvényét. Ehhez mind a helykoordinátát, mind az energiát meg kell változtatni: $x\to x^\prime=x+d$, $y\to y\prime=y+d$ és $E\to E^\prime=ˇE+Fd$, végül a $d\to\infty$ határesetet kell venni.

Az $L\to\infty$ határeset könnyen elvégezhető. \Aeqref{airy:ai+approx} és \aeqref{airy:bi+approx} egyenletek szerint $\Ti(aL-bE)$ gyorsan $0$-hoz tart. Ezt az eredményt felhasználva az $x=0$-ban fallal bezárt részecske Green-függvénye $=Fx$ potenciálban
\begin{equation}
	G_{\text{egy fal}}\left(x,y;E\right) = -\frac{a^2}{F}\frac{\pi}{\Ti(-bE)}\times
	\begin{cases}
		\Ai(v)\Bigl(\Ti(-bE)\Bi(u)-\Ai(u)\Bigr)& x \leq y\\
		\Ai(u)\Bigl(\Ti(-bE)\Bi(v)-\Ai(v)\Bigr)& x \geq y
	\end{cases}.
	\label{limits:semiinfinite}
\end{equation}

A következő határesetet valamivel nehezebb kiszámítani. Ezt előre lehet sejteni, mert az eddigi Green-függvények olyan rendszereket írtak le, ahol minden állapot kötött állapot. A falak nélküli lineáris potenciálhoz nem tartoznak kötött állapotok, csak szórásállapotok vannak. Ez a változás megmutatkozik a Green-függvény pólusszerkezetében, utalva arra, hogy ez a határeset jelentősen megváltoztatja a Green-függvényt matematikai értelemben is. A feljebb említett átmenet,
\begin{equation}
	\begin{aligned}
		x^\prime&=x+d\\
		y^\prime&=y+d\\
		E^\prime&=E+Fd\\
		d       &\to\infty
	\end{aligned}.
	\label{limits:transitiontonowall}
\end{equation}
E az átmenet eltolja a helykoordinátát, miközben a részecske kinetikus energiáját, változatlanul tartja. Az $u$ $v$ változók értéke \eqref{egzakt:uv} egyenlet szerint változatlan marad, a $d\to\infty$ határérték nem változtatja az alakjukat. Mivel a falak nélküli rendszernek az egész valós energiatengely a spektruma, a Green-függvényt az $E^\prime=E+Fd\pm i\epsilon$ energiában vizsgáljuk, a $\Ti(-bE^\prime)$ viselkedését kell meghatározni nagy $E^\prime$ esetén. Felhasználva \aeqref{airy:tiapprox} egyenletet
\begin{dmath}
	\Ti(-x-i\epsilon)
	\approx-\frac{\cos\left(\frac{2}{3}(x+i\epsilon)^{3/2}-\frac{\pi}{4}\right)}{\sin\left(x+i\epsilon)^{3/2}-\frac{\pi}{4}\right)}
	\approx-\frac{\cos\left(\frac{2}{3}x^{3/2}+i\sqrt{x}\epsilon-\frac{\pi}{4}\right)}{\sin\left(\frac{2}{3}x^{3/2}+i\sqrt{x}\epsilon-\frac{\pi}{4}\right)}
	=-\frac{\cos\left(\frac{2}{3}x^{3/2}-\frac{\pi}{4}\right)\cosh\left(\sqrt{x}\epsilon\right)-i\sin\left(\frac{2}{3}x^{3/2}-\frac{\pi}{4}\right)\sinh\left(\sqrt{x}\epsilon\right)}{\sin\left(\frac{2}{3}x^{3/2}-\frac{\pi}{4}\right)\cosh\left(\sqrt{x}\epsilon\right)+i\cos\left(\frac{2}{3}x^{3/2}-\frac{\pi}{4}\right)\sinh\left(\sqrt{x}\epsilon\right)}
	=-\frac{\cos\left(\frac{2}{3}x^{3/2}-\frac{\pi}{4}\right)-i\sin\left(\frac{2}{3}x^{3/2}-\frac{\pi}{4}\right)\tanh\left(\sqrt{x}\epsilon\right)}{\sin\left(\frac{2}{3}x^{3/2}-\frac{\pi}{4}\right)+i\cos\left(\frac{2}{3}x^{3/2}-\frac{\pi}{4}\right)\tanh\left(\sqrt{x}\epsilon\right)}
	\approx-\frac{\cos\left(\frac{2}{3}x^{3/2}-\frac{\pi}{4}\right)-i\sin\left(\frac{2}{3}x^{3/2}-\frac{\pi}{4}\right)\sgn\left(\epsilon\right)}{\sin\left(\frac{2}{3}x^{3/2}-\frac{\pi}{4}\right)+i\cos\left(\frac{2}{3}x^{3/2}-\frac{\pi}{4}\right)\sgn\left(\epsilon\right)}.
\end{dmath}
A sorok közötti lépésekhez felhasználtuk a $(x+a)^\alpha\approx x^\alpha + \alpha x^{\alpha-1}a$ közelítést, a trigonometrikus addíciós képleteket, a képzetes argumentumú trigonometrikus függvények és hiperbolikus függvények kapcsolatát, valamint az előel függvény közelítését a $\tanh$ függvénnyel. Ezek a közelítések egzaktak az $x\to\infty$ határesetben, ezért
\begin{equation}
	\lim_{x\to\infty}\Ti(-x-i\epsilon)=
	\begin{cases}
		i &\epsilon > 0\\
		-i&\epsilon < 0
	\end{cases}.
	\label{limits:ti}
\end{equation}
Ez az eredmény kellett ahhoz, hogy \aeqref{limits:transitiontonowall} átmenet alapján meghatározzuk a fal nélküli lineáris $V=Fx$ potenciálhoz tartozó Green-függényt. Ha $\Im(E)>0$
\begin{dmath}
	G_{\text{nincs fal}}(x,y;E)=\lim_{d\to\infty}G_{\text{egy fal}}(x+d,y+d;E+Fd)=\frac{\pi a^2}{F}\times
	\begin{cases}
		\Ai(v)\Bigl(\Bi(u)-i\Ai(u)\Bigr)&x\leq y\\
		\Ai(u)\Bigl(\Bi(v)-i\Ai(v)\Bigr)&x\geq y
	\end{cases}.
	\label{limits:nowallgreen1}
\end{dmath}
Ha $\Im(E)<0$, akkor \aeqref{limits:ti} egyenlet $-i$ a limeszben, így
\begin{dmath}
	G_{\text{nincs fal}}(x,y;E)=\lim_{d\to\infty}G_{\text{egy fal}}(x+d,y+d;E+Fd)=\frac{\pi a^2}{F}\times
	\begin{cases}
		\Ai(v)\Bigl(\Bi(u)+i\Ai(u)\Bigr)&x\leq y\\
		\Ai(u)\Bigl(\Bi(v)+i\Ai(v)\Bigr)&x\geq y
	\end{cases},
	\label{limits:nowallgreen2}
\end{dmath}
ez a kifejezés csak az $i$ előjelében különbözik az előzőtől. Az egész valós tengely mentén ugrása van ennek a Green-függvénynek a képzetes részének. Ez egybevág azzal a korábbi eredménnyel hogy tetszőleges energiájú sajátállapotai lehetnek a fal nélküli rendszernek, mert a Green-függvénynek vágása van a folytonos spektrumhoz tartozó energiák mentén.











