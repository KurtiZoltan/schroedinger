A reolvens operátor definíciója
\begin{equation}
    \op{G}\left( E \right) = \frac{1}{\op{H} - E}
\end{equation}
és ezen operátorhoz tartozó két változós függvény a Green-függény.
\begin{equation}
    G\left( x, y; E \right) = \Bra{x}G\left(E\right)\Ket{y}
\end{equation}
A Green-függvény név indokolt, és ennek a segítségével fogom meghatározni a Green-függvényeket konkrét esetben. A teljességi reláció beszúrásával látható, hogy a kvantummechanikai Green-függény megegyezik a differenciálegyenletek elméletéből ismert Green-függvénnyel.
\begin{equation}
    \left(\op{H} - E\right) \op{G}\left( E \right) = \op{I}
\end{equation}

\begin{equation}
    \int \mathrm{d}x^\prime \Bra{x}\left(\op{H} - E\right) \Ket{x^\prime}\Bra{x^\prime} \op{G}\left( E \right)\Ket{y} = \Bra{x}\op{I}\Ket{y} = \delta \left(x - y\right)
\end{equation} 
A $\Bra{x}\left(\op{H} - E\right) \Ket{x^\prime}$ maggal vett konvolúció a $\op{H} - E$ operátor hatása. Ezért
\begin{equation}
    \left(\op{H}_x - E\right) G\left(x, y; E\right) = \delta\left(x - y\right)
\end{equation}
ami a differenciálegyenletek elméletéből ismert Green-függvény definíciója. Ebben a konkrét esetben
\begin{equation}
    \left( -\frac{\hbar^2}{2m}\frac{\partial^2}{\partial x^2} + Fx - E \right) G\left(x, y; E\right) = \delta\left(x - y\right)
	\label{green:deltaeq}
\end{equation}
\subsection{Egzakt Green-függvény}
ami azt jelenti, hogy az $x < y$ tartományban
\begin{equation}
    G\left(x, y; E\right) = C_1 \Ai{\sqrt[3]{\frac{2mF}{\hbar^2}}x - \sqrt[3]{\frac{2m}{\hbar^2F^2}}E} + C_2 \Bi{\sqrt[3]{\frac{2mF}{\hbar^2}}x - \sqrt[3]{\frac{2m}{\hbar^2F^2}}E}
    \label{green:xy}
\end{equation}
illetve az $x > y$ tartományban
\begin{equation}
    G\left(x, y; E\right) = C_3 \Ai{\sqrt[3]{\frac{2mF}{\hbar^2}}x - \sqrt[3]{\frac{2m}{\hbar^2F^2}}E} + C_4 \Bi{\sqrt[3]{\frac{2mF}{\hbar^2}}x - \sqrt[3]{\frac{2m}{\hbar^2F^2}}E}
    \label{green:yx}
\end{equation}
, ahol a $C$ együtthatók függhetnek $y$ és $E$ értékétől. A $C$ együtthatók meghatározásához a doboz eredeti határfeltételeit $x = 0$ és $x = L$ pontban, valamint az $x = y$ pontban \aref{green:deltaeq}. egyenlet $y$ körüli integrálásából kapot feltételeket kell felhasználni. A doboz falára vonatkozó határfeltételek:
\begin{equation}
	\left. G\left(x,y;E\right)\right\rvert_{x = 0} = 0
\end{equation}
\begin{equation}
	\left. G\left(x,y;E\right)\right\rvert_{x = L} = 0
\end{equation}
\Aref{green:deltaeq}. egyenlet $\int_{y-\epsilon}^{y+\epsilon}\mathrm{d}x^\prime \int_{y}^{x^\prime} \mathrm{d}x$ szerinti integrálja az $\epsilon \to 0^+$ határesetben: 
\begin{equation}
	\lim_{\epsilon \to 0^+}\left.G\left(x,y;E \right)\right\rvert_{x = y - \epsilon}^{x = y + \epsilon} = 0
\end{equation}
A jobb oldal integrálja $\left. \left(x - y\right) \theta\left(x - y\right) \right\rvert_{x=y-\epsilon}^{x=y+\epsilon}$, ami a határesetben $0$. Az $\left(Fx - E\right)G\left(x,y;E\right)$ integrálja is $0$ a határesetben, mert az erdeti függvény is folytonos, így az integrálja is. \Aref{green:deltaeq}. egyenlet $x$ szerinti integrálja $y$ körüli $\epsilon$ sugarú környezetében az $\epsilon \to 0^+$ határesetben:
\begin{equation}
	\lim_{\epsilon \to 0^+}\left.\frac{\partial}{\partial x}G\left(x,y;E \right)\right\rvert_{x = y - \epsilon}^{x = y + \epsilon} = -\frac{2m}{\hbar^2}
\end{equation}
Itt a jobb oldal integrálja $\left. \theta\left(x - y\right) \right\rvert_{x = y - \epsilon}^{x = y + \epsilon} = 1$ a határesetben. A bal oldalon az előzőhöz hasonló módon csak a derivált integrálja marad meg. \Aref{green:xy}. és \aref{green:yx}. egyenlet behelyettesítése meghatározza a $C$ együtthatókra vonatkozó egyenleteket:
\begin{equation}
	\frac{C_2}{C_1} = -\Ti{-bE}
	\label{green:Cbegin}
\end{equation}
\begin{equation}
	\frac{C_4}{C_3} = -\Ti{b\left(FL - E\right)}
\end{equation}
\begin{equation}
	\frac{C_3}{C_1} = \frac{\Ti{ay - bE} + \Ti{-bE}}{\Ti{ay - bE} + \Ti{b\left(FL - E\right)}}
\end{equation}
TODO: $b$ lecserélése $bE$-re az előző részekben.
\begin{equation}
	C_1 = -\frac{2m}{a\hbar^2}\frac{1}{\left( \left(\frac{C_3}{C_1}-1\right)\Aip{ay - bE} + \left(\frac{C_4}{C_3}\frac{C_3}{C_1} - \frac{C_2}{C_1}\right) \Bip{ay - bE} \right)}
	\label{green:Cend}
\end{equation}
\begin{equation}
	C_1 = -\frac{a^2}{F}\frac{1}{\left( \left(\frac{C_3}{C_1}-1\right)\Aip{ay - bE} + \left(\frac{C_4}{C_3}\frac{C_3}{C_1} - \frac{C_2}{C_1}\right) \Bip{ay - bE} \right)}
\end{equation}
\Aref{green:Cbegin}-\ref{green:Cend}, \ref{green:xy}. és \aref{green:yx}. egyenletek explicit, analitikus módon előállítják a $G\left( x, y; E \right)$ Green-függvényt.
\subsection{Green-függvény perturbáció számítással}
A perturbációszámításhoz a Hamilton operátort két részre bontom fel:
\begin{equation}
	\op{H} = \op{H}_0 + \op{V}
\end{equation}
A $\op{H}_0$ operátorhoz tartozó rezolvens $\op{G}_0\left(E\right)$. $\op{H}$ és $\op{H}_0$ kifejezhetőek a rezolvenseikkel. Ha a kifejezéseket behelyettesítjük a fenti egyenletbe, implicit egyenletet kapunk $op{G}\left(E\right)$-re nézve, melyet fel lehet használni perturbációszámításra. Az egyenletet balról $\op{G}_0^{-1}\left(E\right)$-vel, jobbról $\op{G}^{-1}\left(E\right)$-vel szorzunk.
\begin{equation}
	\op{G}^{-1}\left(E\right) + E = \op{G}_0^{-1}\left(E\right) + E + \op{V}
\end{equation}
\begin{equation}
	\op{G}\left(E\right) = \op{G}_0\left(E\right) - \op{G}_0\left(E\right)\op{V}\op{G}\left(E\right)
	\label{green:pertmaster}
\end{equation}
Az alábbi módon definiálva $\op{G}_n\left(E\right)$ operátort, \aref{green:pertmaster}. egyenlethez hasonló rekurziós összefüggés áll fent:
\begin{equation}
	\op{G}_n\left(E\right) = \op{G}_0\left(E\right)\sum_{k=0}^n\left(-\op{V}\op{G}_0\left(E\right)\right)^k
\end{equation}
\begin{equation}
	\op{G}_{n+1}\left(E\right) = \op{G}_0\left(E\right) - \op{G}_0\left(E\right)\op{V}\op{G}_n\left(E\right)
\end{equation}
Ha $\norm{\op{V}\op{G}_0\left(E\right)} < 1$ akkor a $\op{G}_n$ sorozat konvergál, és kielégíti \aref{green:pertmaster}. egyenletet. Ezért konvergencia esetén:
\begin{equation}
	\op{G}\left(E\right) = \op{G}_0\left(E\right)\sum_{n=0}^\infty\left(-\op{V}\op{G}_0\left(E\right)\right)^n
\end{equation}
A perturbbálatlan operátornak a lineáris potenciál nélküli dobozba zárt részecske Hamilton operátorát választom, $\op{H}_0=\frac{1}{2m}\op{p}^2$, így a lineáris potenciál marad a perturbáció $\op{V} = F\op{x}$. A perturbálatlan $\op{G}_0\left(E\right)$ Green-függvényt is \aref{green:Cbegin}-\ref{green:Cend}, \ref{green:xy}. és \aref{green:yx}. egyenletek alapján határozom meg.
\begin{equation}
	G_0\left(x,y;E\right) =
	\begin{cases}
		-\frac{2m}{k\hbar^2}\frac{1}{\sin\left(kL\right)} \sin\left(k\left(y-L\right)\right)\sin\left(kx\right) & x\leq y\\
		-\frac{2m}{k\hbar^2}\frac{1}{\sin\left(kL\right)} \sin\left(k\left(x-L\right)\right)\sin\left(ky\right) & x>y\\
	\end{cases}
\end{equation}






