A Green-függvény a szilárdtest fizika egyik legtöbbet használt eszköze. A mérhető és egyéb jelentős egyensúlyi mennyiségek gyakran egyszerűen kifejezhetőek a Green-függvénnyel, mint például a (lokális) állapotsűrűség, imaginárius idő használatával pedig termodinamikai mennyiségek: egy részecske operátorok egyensúlyi várható értéke, bizonyos esetekben még két részecske operátorok várható értéke is.

A frekvenciatér beli Green-függvény a Hamilton operátor rezolvenseként definiálható. A rezolvens, avagy a Green operátor
\begin{equation}
    \op{G}\left( E \right) = \left(E-\op{H}\right)^{-1} = \frac{1}{E-\op{H}},
\end{equation}
és ezen operátorhoz tartozó magfüggvény, a Green függvény
\begin{equation}
    G\left( x, y; E \right) = \Bra{x}\op{G}\left(E\right)\Ket{y}.
\end{equation}
A projektor felbontással rendelkező operátorok függvényei felírhatóak összeg alakban is, ez a Green-operátor esetében
\begin{equation}
	\op{G}(E)=\sum_n \frac{\Ket{n}\Bra{n}}{E-E_n}.
\end{equation}

Több féle időfüggő Green-függvény van, ezek mind az időfüggő Schrödinger-egyenlet differenciálegyenletek elméletéből ismert Green-függvények, csupán a határfeltételekben különböznek. Amennyiben a Hamilton-operátor időfüggetlen,
\begin{equation}
	G(x, y, t) = \frac{1}{2\pi\hbar}\int_{-\infty}^{\infty}dE\,G(x,y;E)e^{-\frac{i}{\hbar}Et}.
\end{equation}
Mivel $G(x,y;E)$-nek valós $E$ mentén pólusai vannak, az integrál elvégzéséhez további előírásokra van szükség. A pólusok kerülési iránya határozza meg, hogy retardált vagy avanzsált Green-függvényt kapunk. A pólusok kerülési irányában különböző Green-függvények közötti különbség előállíthatóak a $\op{G}(E)e^{-\frac{i}{\hbar}Et}$ pólusai körül vett komplex $E$ kontúrintegrálokkal. Ezen kontúrintegráloknak az eredménye a reziduumtétel szerint viszont nem más, mint a hullámfüggvénynek a pólushoz tartozó sajátállapotra vett projekciójának időfejlesztő operátora,
\begin{equation}
	\frac{1}{2\pi\hbar i}\int_{C_n}\op{G}(E)e^{-\frac{i}{\hbar}Et}=\Ket{n}\Bra{n}e^{-\frac{i}{\hbar}E_nt},
	\label{green:residuum}
\end{equation}
ahol $C_n$ pozitív irányítású $\epsilon$ sugarú kör az $n$. pólus, azaz az $n$. sajátenergia körül. Ez tetszőleges állapotra hattatva megoldja az időfüggő Schrödinger-egyenletet, ezért lehet különböző kerülési irányokkal előírt Fourier szerű integrál időfüggő Green-függvény.

A retardált Green függvény kontúrra a pólusokat felülről, a pozitív képzetes résszel rendelkező irányban kerüli meg. Másképpen fogalmazva a kontúr a valós tengely, viszont a sajátenergiákat módosítva kell elvégezni az integrált, $E_n\to E_n-i\epsilon$, majd a számítás végén az $\epsilon\to 0^+$ határesetet venni. Ez fizikailag annak felel meg, hogy a sajátállapotoknak van időbeli lecsengése, $\epsilon$ időállandóval.
\begin{equation}
	G_R(x,y,t)=\frac{1}{2\pi\hbar}\lim_{\epsilon\to 0^+}\int_{-\infty}^{\infty}dE\,G(x,y;E+i\epsilon)e^{-\frac{i}{\hbar}Et},
\end{equation}
ez a típusú Green-függvény a múltban $0$ az időbeli lecsengés miatt. Egy másik nevezetes Green-függvény az avanzsált Green-függvény,
\begin{equation}
	G_A(x,y,t)=\frac{1}{2\pi\hbar}\lim_{\epsilon\to 0^+}\int_{-\infty}^{\infty}dE\,G(x,y;E-i\epsilon)e^{-\frac{i}{\hbar}Et},
\end{equation}
ez a Green-függvény az előzőhöz hasonló logika alapján $t>0$ esetén $0$. \Aeqref{green:residuum} egyenlet alapján e két Green-függvény különbsége előállítja az időfejlesztő operátor magját,
\begin{equation}
	\op{G}_A(t)-\op{G}_R(t)=\sum_n\frac{1}{2\pi\hbar}\int_{C_n}\op{G}(E)e^{-\frac{i}{\hbar}Et}=i\sum_n\Ket{n}\Bra{n}e^{-\frac{i}{\hbar}E_nt}=i\op{U}(t).
\end{equation}
A továbbiakban az egy dimenziós homogén tér Green-függvényével foglalkozunk.







