A reolvens operátor definíciója
\begin{equation}
    \op{G}\left( E \right) = \frac{1}{\op{H} - E}
\end{equation}
és ezen operátorhoz tartozó két változós függvény a Green-függény.
\begin{equation}
    G\left( x, y; E \right) = \Bra{x}G\left(E\right)\Ket{y}
\end{equation}
A Green-függvény név indokolt, és ennek a segítségével fogom meghatározni a Green-függvényeket konkrét esetben. A teljességi reláció beszúrásával látható, hogy a kvantummechanikai Green-függény megegyezik a differenciálegyenletek elméletéből ismert Green-függvénnyel.
\begin{equation}
    \left(\op{H} - E\right) \op{G}\left( E \right) = \op{I}
\end{equation}

\begin{equation}
    \int \mathrm{d}x^\prime \Bra{x}\left(\op{H} - E\right) \Ket{x^\prime}\Bra{x^\prime} \op{G}\left( E \right)\Ket{y} = \Bra{x}\op{I}\Ket{y} = \delta \left(x - y\right)
\end{equation} 
A $\Bra{x}\left(\op{H} - E\right) \Ket{x^\prime}$ maggal vett konvolúció a $\op{H} - E$ operátor hatása. Ezért
\begin{equation}
    \left(\op{H}_x - E\right) G\left(x, y; E\right) = \delta\left(x - y\right)
    \label{green:deltaeq}
\end{equation}
ami a differenciálegyenletek elméletéből ismert Green-függvény definíciója. Ebben a konkrét esetben
\begin{equation}
    \left( -\frac{\hbar^2}{2m}\frac{\partial^2}{\partial x^2} + Fx - E \right) G\left(x, y; E\right) = \delta\left(x - y\right)
\end{equation}
ami azt jelenti, hogy az $x < y$ tartományban
\begin{equation}
    G\left(x, y; E\right) = C_1 \Ai{\sqrt[3]{\frac{2mF}{\hbar^2}}x - \sqrt[3]{\frac{2m}{\hbar^2F^2}}E} + C_2 \Bi{\sqrt[3]{\frac{2mF}{\hbar^2}}x - \sqrt[3]{\frac{2m}{\hbar^2F^2}}E}
\end{equation}
illetve az $x > y$ tartományban
\begin{equation}
    G\left(x, y; E\right) = C_3 \Ai{\sqrt[3]{\frac{2mF}{\hbar^2}}x - \sqrt[3]{\frac{2m}{\hbar^2F^2}}E} + C_4 \Bi{\sqrt[3]{\frac{2mF}{\hbar^2}}x - \sqrt[3]{\frac{2m}{\hbar^2F^2}}E}
\end{equation}
, ahol a $C$ együtthatók függhetnek $y$ és $E$ értékétől. A $C$ együtthatók meghatározásához a doboz eredeti határfeltételeit $x = 0$ és $x = L$ pontban, valamint az $x = y$ pontban \aref{green:deltaeq}. egyenlet $y$ körüli integrálásából kapot feltételeket kell felhasználni. 
















