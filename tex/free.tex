Az $F=0$ eset megoldása egyszer, az egyik legalapvetőbb példa egyszerű kvantummechanikai rendszerekre. A sajátfüggvények
\begin{equation}
	\psi_n(x) = \sqrt{\frac{2}{L}}\sin\left(\frac{n\pi x}{L}\right),
\end{equation}
($n=1,2,\dots$), a normálási faktorok
\begin{equation}
	N_n = 1.
\end{equation}
Minden sajátfüggvény egyre normált szinusz függvény, melyek $n-1$ helyen veszik fel a $0$ értéket $x=0$ és $x=L$ között. Sajátenergiáik
\begin{equation}
	E_n = \frac{n^2\pi^2\hbar^2}{2mL^2}.
\end{equation}
Ezek az energiaszintek hasznosak lesznek a numerikus számításokban az $F\neq 0$ eseten is. 