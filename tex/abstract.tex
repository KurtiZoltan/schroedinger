\begin{abstract}
	Kvantummechanikai iskolapélda a homogén térbe helyezett egydimenziós
	részecske. Ezt három dimenzióra kiterjesztve és két fal közé zárva
	keressük az energia sajátállapotokat. Annyi előrelátható, hogy a nyílt
	vagy félig nyílt esetekben használható, reguláris Airy függvény itt nem
	elegendő a megoldáshoz, ennyiben túlmegyünk a tankönyvi feladaton. Az
	aszimptotikus függvényalakok segítségével előállítjuk a magasan
	gerjesztett állapotok energiáit és hullámfüggvényeit, s ezeket
	összehasonlítjuk a közvetlenül a Bohr--Sommerfeld-módszerrel kapott
	eredménnyel. Numerikusan szemléltetjük fizikailag érdekes kezdőállapotok
	időfejlődését. Vizsgáljuk a rezolvenst és az állapotsűrűséget.%, továbbá a sokrészecske rendszerekre való általánosítás lehetőségét.
	
%\centerline{\bf Köszönetnyilvánítás }\vskip0.15truein
	
\end{abstract}
