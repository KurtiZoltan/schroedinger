\begin{abstract}
	Kvantummechanikai iskolapélda a homogén térbe helyezett egydimenziós
	részecske. Ezt három dimenzióra kiterjesztve és két fal közé zárva
	keressük az energia sajátállapotokat. Annyi előrelátható, hogy a nyílt
	vagy félig nyílt esetekben használható, reguláris Airy függvény itt nem
	elegendő a megoldáshoz, ennyiben túlmegyünk a tankönyvi feladaton. Az
	aszimptotikus függvényalakok segítségével előállítjuk a magasan
	gerjesztett állapotok energiáit és hullámfüggvényeit, s ezeket
	összehasonlítjuk a közvetlenül a Bohr--Sommerfeld-módszerrel kapott
	eredménnyel. Numerikusan szemléltetjük fizikailag érdekes kezdőállapotok
	időfejlődését. Vizsgáljuk a rezolvenst és az állapotsűrűséget, továbbá a
	sokrészecske rendszerekre való általánosítás lehetőségét.
	
	Egydimenziós, $m$ tömegű, lineáris $F x$  potenciálban mozgó kvantumos részecskét zárjunk $L$ hosszú, merev falú dobozba (ekvivalens a padló és mennyezet között függőlegesen pattogó kvantum labdával).
	A stacionárius Schrödinger-egyenletből kiindulva, a határfeltételek figyelembe vételével, írjuk fel az energia sajátértékeket meghatározó szekuláris egyenletet, melyet oldjunk meg numerikusan. Ábrázoljuk az alacsonyabb nívókat a doboz méretének változtatása mellett, és szemléltessük grafikusan a stacionárius hullámfüggvényeket.  A szekuláris egyenletben fellépő függvények aszimptotikáinak ismeretében a magasabb nívókra próbáljunk egyszerűbb implicit formulát adni. Végezzük el a szemiklasszikus kvantálást is, hasonlítsuk össze az előző közelítő eredménnyel, és numerikusan néhány, az egzakt egyenletből kapott nívóval.

	További kérdések:  (a) Számítsuk ki a nívókat expliciten, kicsiny $L$-ek mellett. (b) Mely paraméterek mellett esik egybe $F L$ éppen az alapállapoti energiával?  (Ilyenkor a klasszikus labda éppen eléri a mennyezetet.)
	(c) Mutassuk meg, hogy e határesetnél kisebb $L$ belméret mellett minden nívó
	$F L$ fölé esik.
	(d) Írjuk fel a szemiklasszikus stacionárius hullámfüggvényeket, s grafikusan hasonlítsuk össze őket az egzaktakkal -- mikor jó a közelítés?
	(e) Írjuk fel a kicsiny L melletti hullámfüggvényeket expliciten, ezeket szintén hasonlítsuk össze a valódiakkal.
	
	-Miért nem Rodnik osztályba tartozik
	
	-fx, fy = 0 külön tárgyalás
	
	-program leírása
	
\centerline{\bf Köszönetnyilvánítás }\vskip0.15truein
	
\end{abstract}
