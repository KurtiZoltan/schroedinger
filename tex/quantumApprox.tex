\Aeqref{box_energiaszintek_egyenlet} egyenletet nagy $bE$ illetve nagy $bE-aL$ esetén \aeqref{airy:tiapprox} közelítés alkalmazható,
\begin{equation}
	\ctg\left(\frac{2}{3}\left(bE-aL\right)^{3/2}-\frac{\pi}{4}\right)-\ctg\left(\frac{2}{3}\left(bE\right)^{3/2}-\frac{\pi}{4}\right)=0.
	\label{quantumApprox:cot}
\end{equation}
A $\ctg(x)$ függvény $\pi$-ben periodikus, és mivel a $(0,\pi)$ intervallumban szigorúan monoton csökken, \aeqref{quantumApprox:cot} egyenletnek csak akkor van megoldása, ha a $\ctg(x)$ függvények argumentumainak különbsége $n\pi$, azaz
\begin{equation}
	\frac{2}{3}\left(bE\right)^{3/2}-\frac{2}{3}\left(bE-aL\right)^{3/2}=n\pi.
\end{equation}
Az $a$ és $b$ állandók behelyettesítésével ez az egyenlet \aeqref{semiclassicallevels:e2} egyenletet adja. Az $n$  értéke ugyan különbözik $1$-gyel a két egyenletben a Maslov indexek miatt, viszont mivel $n$ egész, ugyan azokat az energiaszinteket határozzák meg. Ennek nem feltétlenül kéne így lennie, viszont ebben az esetben a szemiklasszikus illetve az Airy-függvények aszimptotikus alakjából kapott közelítések egzaktul megegyeznek.

Amennyiben $bE-aL$ negatív, a $\Ti(bE-aL)$ gyorsan lecseng, \aeqref{quantumApprox:cot} egyenlet bal oldalának első tagja elhanyagolható. Ennek a tagnak az elhanyagolásával \aeqref{semiclassicallevels:e1} egyenletet kapjuk vissza. Ez a képlet felel meg az $L\to\infty$ határesetnek, ami a féltérben pattogó labdát írja le.

%$x \to \infty$ aszimptotikus alak:
%	\begin{equation}
%		\Ai\left(-x\right) = \frac{1}{\sqrt{\pi}x^{1/4}}\cos\left(\frac{2}{3}x^{3/2} - \frac{\pi}{4}\right) + \mathcal{O}\left(x^{-5/4}\right)
%	\end{equation}
%	\begin{equation}
%		\Bi\left(-x\right) = -\frac{1}{\sqrt{\pi}x^{1/4}}\sin\left(\frac{2}{3}x^{3/2} - \frac{\pi}{4}\right) + \mathcal{O}\left(x^{-5/4}\right)
%	\end{equation}
%	\begin{equation}
%		\Ti\left(-x\right) = -\cot\left( \frac{2}{3}x^{3/2} - \frac{\pi}{4} \right) + \mathcal{O}\left(x^{-5/4}\right)
%	\end{equation}
%	
%	Ezzel a közelítéssel \aref{box_energiaszintek_egyenlet}. egyenlet alakja:
%	\begin{equation}
%		\cot\left(\frac{2}{3}\left(bE-aL\right)^{3/2} - \frac{\pi}{4}\right) = \cot\left(\frac{2}{3}\left(bE\right)^{3/2} - \frac{\pi}{4}\right)
%	\end{equation}
%	, azaz
%	\begin{equation}
%		\frac{2}{3}\left(bE\right)^{3/2} - \frac{2}{3}\left(bE-aL\right)^{3/2} = n\pi
%	\end{equation}
%	. Az $a$ és $b$ behelyettesítésével az egyenlet
%	\begin{equation}
%		\frac{2\sqrt{2m}}{3F\hbar}\left(E^{3/2} - \left(E - FL\right)^{3/2}\right) = n\pi
%	\end{equation}
%	Ez megegyezik a szemiklasszikus kvantálással kapott eredménnyel, ami azt jelenti, hogy a szemiklasszikus közelítés jól működik nagy energiáknál, hibája $\mathcal{O}\left(E^{-5/4}\right)$ nagyságrendű.

