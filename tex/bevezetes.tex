%nem méréselmélet
A dolgozat címében a Schrödinger macskája méréselméleti utalás ellenére nem foglalkozunk méréselméleti kérdésekkel. A cím csupán a dobozba zárt macska és a dobozba zárt és homogén térbe helyezett kvantum részecske hasonlóságára utal.

%a fizikai rendszer
A dolgozatban tárgyalt rendszer egy belső szabadsági fokokkal nem rendelkező részecske homogén erőtérben, különböző határfeltételekkel. A központi probléma a zárt doboznak megfelelő határfeltétel esete, egy vagy háromdimenzióban. Egy dimenzióban vizsgáljuk az alulról zárt, felülről nyitott dobozt, az úgyevezett "quantum bouncer"-t \cite{vankov2009quantum}, \cite{doi:10.1119/1.10024}, \cite{doi:10.1119/1.16673}. A falak nélküli csupán a lineáris potenciálnak alávetett részecske esetét \cite[137-138.o.]{Vallee:2010:AFA} is vizsgáljuk egydimenzióban.

%az irodalom, hiányzik belőle a felülről zárt eset, Bi
Az irodalomban több helyen megtalálható a "quantum bouncer" ahogy ezt előzőleg említettük. Megtalálható továbbá a \cite{Landau1981Quantum}, \cite{Griffiths2004Introduction} és \cite{Sakurai:1167961} tankönyvekben is, külön elnevezés nélkül. Utóbbi a $V=k\lvert x \rvert$ potenciált vizsgálja, ami triviális kiterjesztése a "quantum bouncer" problémának a Dirichlet-határfeltételen kívül a Neumann-határfeltétellel kapott állapotok megengedésével. A megoldásokat meghatározó egyenlet egy másodrendű lineáris differenciálegyenlet, két független megoldása az úgynevezett $\Ai$ és $\Bi$ Airy-függvények. Ezek közül az $\Ai$ korlátos, míg a $\Bi$ exponenciálisan növekszik pozitív argumentumok esetén. Az előbb említett forrásokban mind csak az $\Ai$ Airy-függvény merül fel, a $\Bi$ esetleges fizikai jelentőségéről nincs szó, a végtelen beli exponenciális növekedés miatt a $\Bi$ függvény fel sem merül. Az $\Ai$ függvény természetesen felmerül minden szemiklasszikus közelítéssel foglalkozó tankönyvben, hiszen az analitikus fordulópontokban a szemiklasszikus megoldásokat az $\Ai$ függvény aszimptotikája illeszti össze. A \cite{doi:10.1007/s12043-001-0081-1} cikkben felmerül a $\Bi$ függvény is, mivel a véges potenciálgödröt vizsgálják és ebben az esetben csak az egyik tartományból lehet kizárni a $\Bi$ függvényt a végtelenben való növekedése miatt. A dolgozatban részletesebben kidolgozzuk a cikkben említett potenciálgödör végtelen mély esetét. Érdemes megjegyezni hogy az említett rendszerek Green-függvényeiben mind felbukkan a $\Bi$ Airy-függvény, még a falak nélküli esetben is.

%didaktika
Klasszikus mechanikában a szabad részecske tárgyalása után legtöbbször az egyenletesen gyorsuló részecske tárgyalása következik, így a kvantummechanika megalapozásának szempontjából jelentős didaktikai szerepe van a lineáris potenciál alapos vizsgálatának. A $\Bi$ függvény fizikai szerepének vizsgálata így indokolt lenne a kvantummechanikába bevezető tankönyvek esetében is, azonban az elterjedt tankönyvekből ez hiányzik.

%a fizikai jelentősége a problémánk
Talán a legjelentősebb fizikai alkalmazása a lineáris potenciálnak a szilárdtest-fizikában van. \cite{Beenakker_1991}-ben számos alkalmazásra lehet példát találni, a jelenségek elméleti leírását és a kapcsolódó kísérleteket s tárgyalják. Két anyag határán vagy a külső elektromos tér, vagy az anyagi minőségek különbségei miatt a vezetési elektronokra az anyaghatárra merőleges irányban ható potenciál jó közelítéssel lineáris, alul egy végtelen potenciálugrással modellezhető potenciálgáttal. P típusú félvezetőt bevonva szigetelő réteggel, és a szigetelő réteg másik oldalára nagy pozitív feszültséget kapcsolva az effektív potenciál az előbb leírt "quantum bouncer" potenciállal írható le. Ha a kapufeszültség jól van megválasztva, a p típusú félvezető a szigetelő síkhoz közeli tartományában az elektronok betölthetnek állapotokat a vezetési sávból. Hasonló helyzet alakulhat ki megfelelően választott p és n típusú félvezetők határán, a potenciál ugrását a vezetési sáv energiájának ugrása okozza a határon, a lineáris potenciált pedig az n típusú félvezetőben a határfelület környékén felhalmozódó pozitív töltések. Fontos, hogy az utóbbi eset megvalósításához nincs feltétlenül szükség külső feszültségforrásra. Így a határfelülethez közeli vezető elektronok hullámfüggvényének merőleges helyfüggésére a "quantum bouncer" Schrödinger-egyenlet vonatkozik. Ha a Fermi-energia és $k_BT$ megfelelő értékűek, akkor a vezetési elektronok a határra merőleges irányban bezáródnak, az alap, vagy esetleg az első néhány gerjesztett állapotban lehetnek. Ekkor ez elektronokat egy kétdimenziós effektív Schrödinger-egyenlet ír le. Ha a merőleges irányban fellépnek magasabb gerjesztett állapotok, akkor azokat belső szabadsági fokként kezelve több komponensű hullámfüggvénnyel lehet modellezni. Ezeket a kétdimenzióba korlátozott vezetési elektronokat nevezik kétdimenziós elektrongáznak (2DEG). További külső potenciálokkal bonyolult geometriájú csatornákat, kapukat lehet kialakítani. Fontos, hogy a kapuk feszültségének változtatásával a kapuk illetve csatornák geometriája és erőssége elektronikusan vezérelhető. Többek között hagyományos tranzisztorok előállítására, qubitek közötti kölcsönhatások szabályozására is alkalmasak.

%dolgozat menetének leírésa
A dolgozat első részét a háromdimenziós dobozba zárt részecske tárgyalásával kezdjük, tetszőleges irányú homogén erőtérben, és három egydimenziós egyenletre redukáljuk a Schrödinger-egyenletet. A dolgozat további részében főleg az egydimenziós problémát vizsgáljuk. Az Airy-függvények alapvető matematikai tulajdonságainak ismertetése után analitikus megoldást mutatunk az egydimenziós zárt doboz esetére. Az energiaszintekre vonatkozó transzcendens egyenletet leszámítva, az energia sajátfüggvényekre és normálási faktoraikra explicit analitikus képleteket vezetünk le. Röviden tárgyaljuk a falak nélküli esetet, és a hozzá tartozó sajátállaptok normálását és teljességi relációját.
A dolgozat második részében a szemiklasszikus közelítést vizsgáljuk. Összevetjük a szemiklasszikus és egyéb közelítésekkel kapott energiaszinteket az implicit egyenletből kapott energiákkal, és megadjuk a Airy-függvények aszimptotikus viselkedését a szemiklasszikus közelítés alapján.
A dolgozat harmadik részében az egy dimenziós eset Green-függvényét vizsgáljuk. Explicit analitikus képletet vezetünk le a zárt doboz esetére. Ezen Green-függvény határeseteiként levezetjük az egy fallal határolt "quantum bouncer", és a fal nélküli rendszer Green-függvényét. Ezek a képletek explicitek. Utóbbi esetében a Green-függvény diszkrét pólusai vágássá alakulnak a komplex energiasíkon. Ez után a dobozba zárt rendszer állapotsűrűségét és a fal nélküli rendszer lokális állapotsűrűségét meghatározzuk a Green-függvényeik alapján. Végül a Green-függvények perturbációs sorát vizsgáljuk numerikusan, a zárt doboz Green-függvényén szemléltetjük, hogy a perturbációs tag triviális változtatása (az egység operátor szám szorosának levonása) drámaian javíthatja a perturbációs sor konvergencia tartományát és sebességét, valamint numerikus módszerek esetén a végeredmény pontosságát is. Végül a függelékben bemutatjuk a Schrödinger-egyenlet időfejlődését ábrázoló kód működését.