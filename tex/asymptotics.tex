Klasszikus mechanikai megfontolások alapján meghatározhatóak az Airy-függvények aszimptotikus alakjai, a pontos fázistól eltekintve. Ez nem meglepő, mert a hullámfüggvény amplitúdója a megtalálási valószínűséggel van kapcsolatban. A hullámfüggvény lokális közelítése egy síkhullámmal, vagyis a fázis deriváltja az impulzussal van kapcsolatban. Így a klasszikus mechanika alapján lehet a hullámfüggvény amplitúdójára és fázisára következtetni.

\Aref{nowall}. fejezetben leírt rendszert vizsgáljuk, $E=0$ választásával, azaz a klasszikus esetben a fordulópont $x=0$-ban van. Kvantum mechanika szerint a megtalálási valószínűség $|\psi|$-tel arányos, klasszikus mechanikában pedig a $dx$ tartományon való áthaladás idejével, $\frac{dx}{v}$-vel arányos. Mivel a kérdéses állapot szórásállapot, nem normálható. Ezért a valószínűségeknél csak arányosságról beszélhetünk, egy részecske rendszerre vonatkozó valószínűségsűrűségként nem értelmezhető. Egy lehetséges interpretáció a szórásállapotok esetében $|\psi|^2$-re, hogy nem kölcsönható részecske áramról van szó, és a résecskék sűrűsége $|\psi|^2$-tel arányos. A klasszikus esetben hasonló a helyzet, a $\frac{dx}{v}$ a részecskesűrűséggel arányos. A két módon kapott részecskesűrűség egyenlőségének feltételezésével a hullámfüggvény amplitúdójának viselkedését kapjuk,
\begin{equation}
	\frac{dx}{v}=\sqrt{-\frac{m}{2Fx}}dx\propto \lvert\psi(x)\rvert^2dx,
\end{equation}
a klasszikus mechanikából ismert energia megmaradás szerint. Átrendezve
\begin{equation}
	\psi(x)\propto\frac{1}{\sqrt[4]{-x}}.
\end{equation}
A hullámfüggvény fázisának meghatározása a de Broglie hullámhossz, $p=\hbar k$, és a klasszikus impulzus alapján történik. Abban az esetben, ha az amplitúdó ami közelítőleg megkapható az előző egyenletből, kicsit változik a de Broglie hullámhossz alatt,
\begin{equation}
	\psi(x)\propto\exp\left(i\int_{x_0}^xk\left(x^\prime\right)\,dx^\prime\right).
\end{equation}
A klasszikus energia megmaradás meghatározza az impulzust, ami alapján a de Broglie hullámszám
\begin{equation}
	k=\frac{\sqrt{2mF}}{\hbar}\sqrt{-x}.
\end{equation}
A $k$ integrálja könnyen kiszámítható,
\begin{equation}
	\int \frac{\sqrt{2mF}}{\hbar}\sqrt{-x}\,dx=\frac{2}{3}\left(-ax\right)^{3/2}.
\end{equation}
Az amplitúdóra és hullámhosszra vonatkozó feltételeket összekombinálva
\begin{equation}
	\psi(x)\propto\Ai(ax)\propto\frac{1}{\sqrt[4]{-ax}}\sin\left(\frac{2}{3}\left(-ax\right)^{3/2}\right).
\end{equation}





