A rendszer egy téglatest alakú dobozba zárt részecske. A doboz mérete $L_x$, $L_y$ és $L_z$. A dobozban homogén erőtér hat a részecskére, azaz $\boldsymbol{F} = const$. A potenciál így $V(x, y, z) = -\boldsymbol{F}_xx-\boldsymbol{F}_yy-\boldsymbol{F}_zz$. Mivel az a potenciál lineáris $x$ $y$ és $z$-ben, a Schrödinger egyenlet szeparálható.
    \begin{equation}
        \psi_{klm}\left(x, y, z\right) = \phi_k \left( x^\prime \right)\phi_l\left(y^\prime\right)\phi_m\left(z^\prime\right)
    \end{equation}
    Ahol $x^\prime = \sqrt[3]{\frac{2m\boldsymbol{F}_x}{\hbar^2}}x - \sqrt[3]{\frac{2m}{\hbar^2\boldsymbol{F}_x^2}}E_k$, $E_k$ pedig az 1 dimenziós probléma, $\Ti{\sqrt[3]{\frac{2m\boldsymbol{F}_x}{\hbar^2}}L - \sqrt[3]{\frac{2m}{\hbar^2\boldsymbol{F}_x^2}}E} - \Ti{-\sqrt[3]{\frac{2m}{\hbar^2\boldsymbol{F}_x^2}}E} = 0$, $k$. $\phi_k \left( x^\prime \right)$ az 1D-s rész TODO:REFERENCIA hullámfüggvénye. $y^\prime$, $z^\prime$, valamint $E_l$ és $E_m$ hasonlóan vannak definiálva a hozzájuk tartozó 1 dimenziós probléma alapján. A 3D-s hullámfüggvényhez tartozó energia az 1D-s megoldásokhoz tartozó energiák összege.
    \begin{equation}
        E = E_k + E_l + E_m
    \end{equation}
    Amennyiben valamelyik irányú komponense $\boldsymbol{F}$-nek 0, abban az esetben a hozzá tartozó 1D-s pprobléma a híres végtelen mély potenciálgödör, ahol
    \begin{equation}
        \phi_n = \sqrt{\frac{2}{L}}\sin\left(\frac{nx\pi}{L}\right)
    \end{equation}
    valamint
    \begin{equation}
        E_n = \frac{\hbar^2n^2}{2mL^2}
    \end{equation}
    
    TODO: ÁBRA AZ EGYSZER FÜGGŐLEGES ESET ENERGIÁIRÓL, esetleg szintén L függvényében.
    
    TODO: ábra 2D -quantum chaos in billiards-
    
    TODO: 2D (3D?) videó link időfelődésről
