Falak hiányában a Schrödinger-egyenlet továbbra is \eqref{3dbox:1deq}, azonban a határfeltételek különböznek. A fizikai kép az, hogy $V(x)=Fx$ potenciál esetén az $x\to\infty$-ből nem jönnek részecskék, és nem is tartózkodnak ott. Ezek problémás állapotok lennének, végtelen energiával rendelkeznének. Tehát a szórásállapotokra vonatkozó feltétel, hogy
\begin{equation}
	\lim_{x\to\infty}\psi(x) = 0.
	\label{nowall:oundary}
\end{equation}
Mivel itt folytonos spektrumról van szó, az eddigi normálás helyett az állapotokat Dirac-deltára kell normálni. Ebben a feladatban az energia és energia sajátállapot között egy az egyhez megfeleltetés van, ellenben a jól ismert szabad részecske esetével. Ennek oka, hogy itt $x\to\infty$-ből nem jönnek részecskék. Ennek következtében az a sajátállapotokat $\Ket{E}$ egyértelmen jelöli.
\Aeqref{nowall:oundary} feltétel azt jelenti, hogy az Airy-függvények közül a $\Bi(ax-bE)$ nem szerepel a lineáris kominációban, a megoldás tisztán az $\Ai(ax-bE)$ függvény lesz,
\begin{equation}
	\Braket{x|E}=N\Ai(ax-bE).
\end{equation}
A szórásállapotokra vonatkozó normálási feltétel
\begin{equation}
	\Braket{E|E^\prime}=\delta(E-E^\prime).
\end{equation}
Ez alapján $N$ meghatározható \eqref{airy:delta} azonosság felhasználásával,
\begin{dmath}
	\delta(E-E^\prime)=N^2\int_{-\infty}^\infty\Ai(ax-bE)\Ai(ax-bE^\prime)\,dx=N^2\frac{1}{ab}\delta(E-E^\prime).
\end{dmath}
Ez alapján $N=\sqrt{ab}=\sqrt[3]{\frac{2m}{\hbar^2\sqrt{F}}}$, és
\begin{equation}
	\Braket{x|E}=\psi_E(x)=\sqrt{ab}\Ai(ax-bE).
\end{equation}
A teljességi reláció is leellenőrizhető \aeqref{airy:delta} egyenlet alapján,
\begin{dmath}
	\int_{-\infty}^\infty dE\,\Ket{E}\Bra{E}=ab\int_{-\infty}^\infty dE\int_{-\infty}^\infty dx\int_{-\infty}^\infty dy\,\Ai(ax-bE)\Ai(ay-bE)\Ket{x}\Bra{y}=\int_{-\infty}^\infty dx\int_{-\infty}^\infty dy\,\delta(x-y)\Ket{x}\Bra{y}=\op{I}.
\end{dmath}