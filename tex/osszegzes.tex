A dolgozat első részében visszavezettük a háromdimenziós időfüggő problémát egydimenziós időfüggetlen problémákra, majd analitikus képleteket adtunk az egydimenziós probléma megoldásaira. Ezeket az egzakt képleteket összevetettük a szemiklasszikus közelítés eredményével, és a formulák fizikai interpretációját diszkutáltuk. Explicit analitikus képletet vezettünk le az időfüggetlen Green-függvényre, a pólusszerkezetét összevetettük az első részben kapott energiaszinteket meghatározó transzcendens egyenlettel. Szemléltettük a Green-függvény alkalmazhatóságát az állapotsűrűség numerikus illetve analitikus meghatározására is.

Egy konkrét példán bemutattuk, hogy a Hamiton-operátor önkényes felbontása perturbálatlan Hamilton-operátorra és perturbáló operátorra nagy mértékben befolyásolja a perturbációs sor konvergencia tulajdonságait. A példánkban a perturbáló operátor normájának minimalizálása egy triviális tag levonásával jelentősen javította a perturbációs sor konvergenciáját. Több részecske rendszereket leíró Green-függvények perturbációszámítása hatalmas jelentőséggel bír, számos fizikai témakör egyik fő eszköze, így a jövőben érdemes megvizsgálni, hogy milyen lehetőség van esetleg triviális tagok levonásával módosított perturbáció szerinti sorfejtés optimalizálására.