\documentclass[pdftex,12pt,a4paper]{article}
\pdfpagewidth 8.5in
\pdfpageheight 11.6in
\linespread{1.3}
\usepackage{anysize}
\marginsize{3.5cm}{2.5cm}{2.5cm}{2.5cm}

\usepackage[utf8]{inputenc}
\usepackage[T1]{fontenc}
\usepackage[magyar]{babel}
\usepackage{amsmath}
\usepackage{float}
\usepackage{graphicx}
\usepackage{braket}
\usepackage[unicode,pdftex]{hyperref}
%\usepackage{hyperref}
\usepackage{breqn}

\DeclareMathOperator{\Ai}{Ai}
\DeclareMathOperator{\Bi}{Bi}
\DeclareMathOperator{\Aip}{Ai^\prime}
\DeclareMathOperator{\Bip}{Bi^\prime}
\DeclareMathOperator{\Ti}{Ti}
\newcommand{\op}[1]{\hat{#1}}
\newcommand{\norm}[1]{\left\lVert #1 \right\rVert}
\newcommand*\Laplace{\mathop{}\!\mathbin\bigtriangleup}

\newcommand{\aeqref}[1]{\az{\eqref{#1}}}
\newcommand{\Aeqref}[1]{\Az{\eqref{#1}}}

\hypersetup{
    colorlinks,
    citecolor=black,
    filecolor=black,
    linkcolor=black,
    urlcolor=black
}
\hypersetup{	
	pdftitle={Schrödinger macskája dobozban: falak közé zárt részecske kvantumállapotai homogén külső térben},
	pdfsubject={Kvantum mechanika lineáris potenciálban falak között.},
	pdfauthor={Kürti Zoltán}}

\frenchspacing
\begin{document}
\numberwithin{equation}{section}
\numberwithin{figure}{section}
\numberwithin{table}{section}
\addtolength{\marginparwidth}{50pt}

\pdfbookmark[1]{Címlap}{cim}
\pagenumbering{roman}
\label{cimlap}
\thispagestyle{empty} 

\null\vskip-0.8truein
\centerline{\Large\sc Szakdolgozat}\vskip0.6truein

\centerline{\bf\LARGE Falak közé zárt kvantum részecske homogén térben:}\vskip0.15truein
\centerline{\bf\LARGE "Schrödinger macskája dobozban"}

\vskip0.4truein\centerline{\Large\sc Kürti Zoltán}\vskip0.10truein
\centerline{\Large\sl Fizika BSc., fizikus szakirány}\vskip0.06truein
%\centerline{\Large\sl III. évfolyam }\vskip0.3truein


%\centerline{\psfig{file= ./fig/elte.pdf}}
%\centerline{\psfig{file=../fig/elte.ps}}
%\centerline{\psfig{file=./fig/elte_cimer_szines.jpg}}
\centerline{\includegraphics[scale=0.5]{./figs/elte_cimer_color.pdf}}
\vskip0.4truein
\centerline{\Large Témavezetők:}\vskip0.2truein
\centerline{\Large{\sc{ Dr. Cserti József}} }\vskip0.001truein
\centerline{egyetemi tanár}\vskip0.15truein
\centerline{\Large\sc Dr. Györgyi Géza}\vskip0.001truein
\centerline{egyetemi docens}\vskip0.2truein
\centerline{\Large \sc \bf Eötvös Loránd Tudományegyetem}\vskip0.010truein
\centerline{\Large  Komplex Rendszerek Fizikája Tanszék}\vskip0.15truein
\centerline{\Large\bf 2021}
\newpage

\begin{abstract}
	Kvantummechanikai iskolapélda a homogén térbe helyezett egydimenziós
	részecske. Ezt három dimenzióra kiterjesztve és két fal közé zárva
	keressük az energia sajátállapotokat. Annyi előrelátható, hogy a nyílt
	vagy félig nyílt esetekben használható, reguláris Airy függvény itt nem
	elegendő a megoldáshoz, ennyiben túlmegyünk a tankönyvi feladaton. Az
	aszimptotikus függvényalakok segítségével előállítjuk a magasan
	gerjesztett állapotok energiáit és hullámfüggvényeit, s ezeket
	összehasonlítjuk a közvetlenül a Bohr--Sommerfeld-módszerrel kapott
	eredménnyel. Numerikusan szemléltetjük fizikailag érdekes kezdőállapotok
	időfejlődését. Vizsgáljuk a rezolvenst és az állapotsűrűséget.%, továbbá a sokrészecske rendszerekre való általánosítás lehetőségét.
	
%\centerline{\bf Köszönetnyilvánítás }\vskip0.15truein
	
\end{abstract}

\newpage
\phantomsection
\pdfbookmark[1]{Tartalomjegyzék}{tartalom}
\tableofcontents
\newpage
\listoffigures
\listoftables
\newpage
\pagenumbering{arabic}
\phantomsection

\section{Bevezetés}
	%nem méréselmélet
A dolgozat címében a Schrödinger macskája méréselméleti utalás ellenére nem foglalkozunk méréselméleti kérdésekkel. A cím csupán a dobozba zárt macska és a dobozba zárt és homogén térbe helyezett kvantum részecske hasonlóságára utal.

%a fizikai rendszer
A dolgozatban tárgyalt rendszer egy belső szabadsági fokokkal nem rendelkező részecske homogén erőtérben, különböző határfeltételekkel. A központi probléma a zárt doboznak megfelelő határfeltétel esete, egy vagy háromdimenzióban. Egy dimenzióban vizsgáljuk az alulról zárt, felülről nyitott dobozt, az úgyevezett "quantum bouncer"-t \cite{vankov2009quantum}, \cite{doi:10.1119/1.10024}, \cite{doi:10.1119/1.16673}. A falak nélküli csupán a lineáris potenciálnak alávetett részecske esetét \cite[137-138.o.]{Vallee:2010:AFA} is vizsgáljuk egydimenzióban.

%az irodalom, hiányzik belőle a felülről zárt eset, Bi
Az irodalomban több helyen megtalálható a "quantum bouncer" ahogy ezt előzőleg említettük. Megtalálható továbbá a \cite{Landau1981Quantum}, \cite{Griffiths2004Introduction} és \cite{Sakurai:1167961} tankönyvekben is, külön elnevezés nélkül. Utóbbi a $V=k\lvert x \rvert$ potenciált vizsgálja, ami triviális kiterjesztése a "quantum bouncer" problémának a Dirichlet-határfeltételen kívül a Neumann-határfeltétellel kapott állapotok megengedésével. A megoldásokat meghatározó egyenlet egy másodrendű lineáris differenciálegyenlet, két független megoldása az úgynevezett $\Ai$ és $\Bi$ Airy-függvények. Ezek közül az $\Ai$ korlátos, míg a $\Bi$ exponenciálisan növekszik pozitív argumentumok esetén. Az előbb említett forrásokban mind csak az $\Ai$ Airy-függvény merül fel, a $\Bi$ esetleges fizikai jelentőségéről nincs szó, a végtelen beli exponenciális növekedés miatt a $\Bi$ függvény fel sem merül. Az $\Ai$ függvény természetesen felmerül minden szemiklasszikus közelítéssel foglalkozó tankönyvben, hiszen az analitikus fordulópontokban a szemiklasszikus megoldásokat az $\Ai$ függvény aszimptotikája illeszti össze. A \cite{doi:10.1007/s12043-001-0081-1} cikkben felmerül a $\Bi$ függvény is, mivel a véges potenciálgödröt vizsgálják és ebben az esetben csak az egyik tartományból lehet kizárni a $\Bi$ függvényt a végtelenben való növekedése miatt. A dolgozatban részletesebben kidolgozzuk a cikkben említett potenciálgödör végtelen mély esetét. Érdemes megjegyezni hogy az említett rendszerek Green-függvényeiben mind felbukkan a $\Bi$ Airy-függvény, még a falak nélküli esetben is.

%didaktika
Klasszikus mechanikában a szabad részecske tárgyalása után legtöbbször az egyenletesen gyorsuló részecske tárgyalása következik, így a kvantummechanika megalapozásának szempontjából jelentős didaktikai szerepe van a lineáris potenciál alapos vizsgálatának. A $\Bi$ függvény fizikai szerepének vizsgálata így indokolt lenne a kvantummechanikába bevezető tankönyvek esetében is, azonban az elterjedt tankönyvekből ez hiányzik.

%a fizikai jelentősége a problémánk
Talán a legjelentősebb fizikai alkalmazása a lineáris potenciálnak a szilárdtest-fizikában van. \cite{Beenakker_1991}-ben számos alkalmazásra lehet példát találni, a jelenségek elméleti leírását és a kapcsolódó kísérleteket s tárgyalják. Két anyag határán vagy a külső elektromos tér, vagy az anyagi minőségek különbségei miatt a vezetési elektronokra az anyaghatárra merőleges irányban ható potenciál jó közelítéssel lineáris, alul egy végtelen potenciálugrással modellezhető potenciálgáttal. P típusú félvezetőt bevonva szigetelő réteggel, és a szigetelő réteg másik oldalára nagy pozitív feszültséget kapcsolva az effektív potenciál az előbb leírt "quantum bouncer" potenciállal írható le. Ha a kapufeszültség jól van megválasztva, a p típusú félvezető a szigetelő síkhoz közeli tartományában az elektronok betölthetnek állapotokat a vezetési sávból. Hasonló helyzet alakulhat ki megfelelően választott p és n típusú félvezetők határán, a potenciál ugrását a vezetési sáv energiájának ugrása okozza a határon, a lineáris potenciált pedig az n típusú félvezetőben a határfelület környékén felhalmozódó pozitív töltések. Fontos, hogy az utóbbi eset megvalósításához nincs feltétlenül szükség külső feszültségforrásra. Így a határfelülethez közeli vezető elektronok hullámfüggvényének merőleges helyfüggésére a "quantum bouncer" Schrödinger-egyenlet vonatkozik. Ha a Fermi-energia és $k_BT$ megfelelő értékűek, akkor a vezetési elektronok a határra merőleges irányban bezáródnak, az alap, vagy esetleg az első néhány gerjesztett állapotban lehetnek. Ekkor ez elektronokat egy kétdimenziós effektív Schrödinger-egyenlet ír le. Ha a merőleges irányban fellépnek magasabb gerjesztett állapotok, akkor azokat belső szabadsági fokként kezelve több komponensű hullámfüggvénnyel lehet modellezni. Ezeket a kétdimenzióba korlátozott vezetési elektronokat nevezik kétdimenziós elektrongáznak (2DEG). További külső potenciálokkal bonyolult geometriájú csatornákat, kapukat lehet kialakítani. Fontos, hogy a kapuk feszültségének változtatásával a kapuk illetve csatornák geometriája és erőssége elektronikusan vezérelhető. Többek között hagyományos tranzisztorok előállítására, qubitek közötti kölcsönhatások szabályozására is alkalmasak.

%dolgozat menetének leírésa
A dolgozat első részét a háromdimenziós dobozba zárt részecske tárgyalásával kezdjük, tetszőleges irányú homogén erőtérben, és három egydimenziós egyenletre redukáljuk a Schrödinger-egyenletet. A dolgozat további részében főleg az egydimenziós problémát vizsgáljuk. Az Airy-függvények alapvető matematikai tulajdonságainak ismertetése után analitikus megoldást mutatunk az egydimenziós zárt doboz esetére. Az energiaszintekre vonatkozó transzcendens egyenletet leszámítva, az energia sajátfüggvényekre és normálási faktoraikra explicit analitikus képleteket vezetünk le. Röviden tárgyaljuk a falak nélküli esetet, és a hozzá tartozó sajátállaptok normálását és teljességi relációját.
A dolgozat második részében a szemiklasszikus közelítést vizsgáljuk. Összevetjük a szemiklasszikus és egyéb közelítésekkel kapott energiaszinteket az implicit egyenletből kapott energiákkal, és megadjuk a Airy-függvények aszimptotikus viselkedését a szemiklasszikus közelítés alapján.
A dolgozat harmadik részében az egy dimenziós eset Green-függvényét vizsgáljuk. Explicit analitikus képletet vezetünk le a zárt doboz esetére. Ezen Green-függvény határeseteiként levezetjük az egy fallal határolt "quantum bouncer", és a fal nélküli rendszer Green-függvényét. Ezek a képletek explicitek. Utóbbi esetében a Green-függvény diszkrét pólusai vágássá alakulnak a komplex energiasíkon. Ez után a dobozba zárt rendszer állapotsűrűségét és a fal nélküli rendszer lokális állapotsűrűségét meghatározzuk a Green-függvényeik alapján. Végül a Green-függvények perturbációs sorát vizsgáljuk numerikusan, a zárt doboz Green-függvényén szemléltetjük, hogy a perturbációs tag triviális változtatása (az egység operátor szám szorosának levonása) drámaian javíthatja a perturbációs sor konvergencia tartományát és sebességét, valamint numerikus módszerek esetén a végeredmény pontosságát is. Végül a függelékben bemutatjuk a Schrödinger-egyenlet időfejlődését ábrázoló kód működését.
\section{A dobozba zárt részecske homogén térben}
	\input
	\subsection{Három dimenzióban}
		A rendszer egy téglatest alakú dobozba zárt részecske. A doboz mérete $L_x$, $L_y$ és $L_z$. A dobozban homogén erőtér hat a részecskére, azaz $\boldsymbol{F} = \text{const}$. A potenciál így $V(x, y, z) = -\boldsymbol{F}_xx-\boldsymbol{F}_yy-\boldsymbol{F}_zz$. A rendszer időfüggő Schrödinger-egyenlete
\begin{equation}
	i\hbar\frac{\partial \psi(x, y, z, t)}{\partial t} = -\frac{\hbar^2}{2m} \Laplace \psi(x, y, z, t) + V(x, y, z)\psi(x, y, z, t).
	\label{3dbox:3dscheq}
\end{equation}
Az egyenlet kezdőfeltétele egy kezdeti állapot $t_0$-ban, $\psi(x, y, z, t_0) = \psi_0(x, y, z)$, az egyenlet határfeltételei pedig a hullámfüggvény határokon való eltűnése, $0=\left.\psi\right|_{x=0}=\left.\psi\right|_{x=L_x}=\left.\psi\right|_{y=0}=\left.\psi\right|_{y=L_y}=\left.\psi\right|_{z=0}=\left.\psi\right|_{z=L_z}$. Mivel ez a potenciál lineáris $x$, $y$ és $z$-ben, a Schrödinger-egyenlet szeparálható a
\begin{equation}
	\psi_{klm}(x, y, z, t) = e^{-\frac{iE_{klm}t}{\hbar}}\psi^{(x)}_k(x)\psi^{(y)}_l(y)\psi^{(z)}_m(z)
	\label{3dox:3dansatz}
\end{equation}
próbafüggvénnyel. A $\psi^{(i)}_n$ ($i=x, y, z$) függvényekre így az egy dimenziós stacionárius Schrödinger-egyenlet vonatkozik. A $\psi^{(x)}$-re vonatkozó egyenlet 
\begin{equation}
	-\frac{\hbar^2}{2m}\frac{d^2\psi^{(x)}_k(x)}{dx^2} + \boldsymbol{F}_xx\psi^{(x)}_k(x) = E^{(x)}_k\psi^{(x)}_k(x),
\end{equation}
a határfeltételek $0=\left.\psi^{(x)}_k\right|_{x=0}=\left.\psi^{(x)}_k\right|_{x=L_x}$. $\psi^{(y)}_l$ és $\psi^{(z)}_m$-re vonatkozó egyenletek hasonlóak. Az $E_{klm}$ energia a három egy dimenziós stacionárius Schrödinger-egyenlet sajátenergiáinak összege,
\begin{equation}
	E_{klm} = E^{(x)}_k+E^{(y)}_l+E^{(z)}_m.
\end{equation}
\Aeqref{3dbox:3dscheq} egyenlet általános megoldása \aeqref{3dox:3dansatz} próbafüggvények kezdőfeltételhez illesztett lineáris kombinációja,
\begin{equation}
	\psi(x,y,z,t) = \sum_{klm}C_{klm}\psi_{klm}(x,y,z,t).
\end{equation}
$C_{klm}$ együtthatók meghatározásához a szokásos hely reprezentáció beli skalárszorzást kell használni,
\begin{equation}
	C_{klm} = \frac{1}{N_{klm}}\int_0^{L_x}dx\int_0^{L_y}dy\int_0^{L_z}dz\,\psi_{klm}(x, y, z, t=0)^*\psi_0(x, y, z)
	\label{3dbox:ceq}
\end{equation}
\begin{equation}
	N_{klm} = \int_0^{L_x}dx\int_0^{L_y}dy\int_0^{L_z}dz\,|\psi_{klm}(x,y,z,t=0)|^2.
	\label{3dbox:3norm}
\end{equation}
\Aeqref{3dbox:ceq} egyenlet nem egyszerűsíthető tovább általános $\psi_0$ esetén, viszont \aeqref{3dbox:3norm} igen. Mivel $\psi_{klm}$ szorzat alakú, nem kell a tripla integrált elvégezni, hanem csak három egyszeres integrál szorzatát kell kiszámítani. Ez numerikus számításokban jelentős.
\begin{equation}
	N_{klm} = N^{(x)}_kN^{(y)}_lN^{(z)}_m,
\end{equation}
ahol az egyes $N$ tagok az egy dimenziós sajátfüggvények normájaként vannak definiálva.
\begin{equation}
	N^{(x)}_k = \int_0^{L_x}dx\,\left|\psi^{(x)}_k(x)\right|^2,
\end{equation}
$N^{(y)}_l$-re és $N^{(z)}_m$-re hasonló képletek vonatkoznak. 



%    Ahol $x^\prime = \sqrt[3]{\frac{2m\boldsymbol{F}_x}{\hbar^2}}x - \sqrt[3]{\frac{2m}{\hbar^2\boldsymbol{F}_x^2}}E_k$, $E_k$ pedig az 1 dimenziós probléma, $\Ti{\sqrt[3]{\frac{2m\boldsymbol{F}_x}{\hbar^2}}L - \sqrt[3]{\frac{2m}{\hbar^2\boldsymbol{F}_x^2}}E} - \Ti{-\sqrt[3]{\frac{2m}{\hbar^2\boldsymbol{F}_x^2}}E} = 0$, $k$. $\phi_k \left( x^\prime \right)$ az 1D-s rész TODO:REFERENCIA hullámfüggvénye. $y^\prime$, $z^\prime$, valamint $E_l$ és $E_m$ hasonlóan vannak definiálva a hozzájuk tartozó 1 dimenziós probléma alapján. A 3D-s hullámfüggvényhez tartozó energia az 1D-s megoldásokhoz tartozó energiák összege.
%    \begin{equation}
%        E = E_k + E_l + E_m
%    \end{equation}
%    Amennyiben valamelyik irányú komponense $\boldsymbol{F}$-nek 0, abban az esetben a hozzá tartozó 1D-s pprobléma a híres végtelen mély potenciálgödör, ahol
%    \begin{equation}
%        \phi_n = \sqrt{\frac{2}{L}}\sin\left(\frac{nx\pi}{L}\right)
%    \end{equation}
%    valamint
%    \begin{equation}
%        E_n = \frac{\hbar^2n^2}{2mL^2}
%    \end{equation}
%    
%    TODO: ÁBRA AZ EGYSZER FÜGGŐLEGES ESET ENERGIÁIRÓL, esetleg szintén L függvényében.
%    
%    TODO: ábra 2D -quantum chaos in billiards-
%    
%    TODO: 2D (3D?) videó link időfelődésről

	\subsection{Egy dimenzióban}
			A probléma egy 1D doboba zárt résecske homogén erőtérben. $F(x)=-F$, azaz $V(x) = Fx$.
	Az egyenlethez tartozó határfeltételek, ha a doboz hossza $L$:
	\begin{equation}
		\psi \big\rvert_0 = \psi \big \rvert_L = 0
	\end{equation}
	A megoldandó időfüggetlen Schrödinger-egyenlet:
	\begin{equation}
		-\frac{\hbar^2}{2m}\frac{d^2\psi}{dx^2} + Fx\psi = E\psi
	\end{equation}
	\begin{equation}
		\frac{d^2\psi}{dx^2} - \frac{2mFx}{\hbar^2}\psi = -\frac{2mE}{\hbar^2}\psi
	\end{equation}
	\begin{equation}
		\frac{d^2\psi}{dx^2} - \left(\frac{2mF}{\hbar^2}x - \frac{2mE}{\hbar^2}\right)\psi = 0
	\end{equation}
	Az Airy egyenlet ilyen alakra hozható a változó affin lineáris transzformációjával:
	\begin{equation}
		\frac{d^2y}{dx^{\prime 2}} - x^\prime y = 0
	\end{equation}
	$x^\prime = ax - b$, azaz $\frac{d}{dx} = a\frac{d}{dx^\prime}$:
	\begin{equation}
		\frac{d^2y}{dx^2} - \left(a^3x - a^2b\right)y = 0
	\end{equation}
	Az együtthatók összevetése alapján $a = \sqrt[3]{\frac{2mF}{\hbar^2}}$ és $b = \sqrt[3]{\frac{2m}{\hbar^2F^2}}E$. Így a Schrödinger-gyenlet megoldása:
	\begin{equation}
		\psi(x) = y(x^\prime) = y\left(\sqrt[3]{\frac{2mF}{\hbar^2}}x - \sqrt[3]{\frac{2m}{\hbar^2F^2}}E\right)
	\end{equation}
	, ahol $y(x) = \alpha \Ai{x} + \beta \Bi{x}$.
	A $\psi \big\rvert_0 = 0$ feltételből következik, hogy $\psi \propto \Bi{-b}\Ai{ax-b} - \Ai{-b}\Bi{ax-b}$. A második határfeltétel pedig meghatározza a lehetséges energiákat. A feltétel:
	\begin{equation}
		\Bi{-b}\Ai{aL-b} - \Ai{-b}\Bi{aL-b} = 0
	\end{equation}
	\begin{equation}
		\label{box_energiaszintek_egyenlet}
		\Ti{aL-b} - \Ti{-b} = 0
	\end{equation}
	\begin{equation}
		\Ti{\sqrt[3]{\frac{2mF}{\hbar^2}}L - \sqrt[3]{\frac{2m}{\hbar^2F^2}}E} - \Ti{-\sqrt[3]{\frac{2m}{\hbar^2F^2}}E} = 0
	\end{equation}
	\begin{figure}[H]
		\includegraphics[scale=1]{./figs/energiaszintek.pdf}
		\caption{Energiaszintek $L$ függvényében}
		\label{box_energiaszintek_abra}
	\end{figure}
	Amikor $FL \ll \frac{\pi^2\hbar^2}{2mL^2}$, a potenciál jól közelíthető konstans potenciállal, mivel az alapállapot energiájához képest is elhanyagolható a lineáris potenciál eltérése a konstans potenciáltól. Eben a esetben $E \propto n^2$. $E \ll FL$ esetben az energiaszintek jó közelítéssel konstanssá válnak. Ennek az oka, hogy $\lim_{L \to \infty}\psi(x) = \alpha \Ai{ax-b}$, mert a $Bi(x)$ exponenciálisan növekszik nagy $x$-ek esetén. Ebben az eseten az energiaszinteket a $\Ai{- \sqrt[3]{\frac{2m}{\hbar^2F^2}}E} = 0$ egyenlet határozza meg. Ezeket az aszimptotikus viselkedéseket \aref{box_energiaszintek_abra}. ábra jól mutatja.
    
    TODO: link 1D videóról


		\subsubsection{$F=0$ eset}
			Az $F=0$ eset megoldása egyszerű, az egyik legalapvetőbb példa egyszerű kvantummechanikai rendszerekre. A sajátfüggvények
\begin{equation}
	\psi_n(x) = \sqrt{\frac{2}{L}}\sin\left(\frac{n\pi x}{L}\right),
\end{equation}
($n=1,2,\dots$), a normálási faktorok
\begin{equation}
	N_n = 1.
\end{equation}
Minden sajátfüggvény egyre normált szinusz függvény, melyek $n-1$ helyen veszik fel a $0$ értéket $x=0$ és $x=L$ között. Sajátenergiáik
\begin{equation}
	E_n = \frac{n^2\pi^2\hbar^2}{2mL^2}.
\end{equation}
Ezek az energiaszintek hasznosak lesznek a numerikus számításokban az $F\neq 0$ eseten is. 
		\subsubsection{Airy függvények}
			Az Airy egyenlet
\begin{equation}
	\frac{d^2y}{dx^2} - xy = 0,
	\label{airy:airyeq}
\end{equation}
ennek az egyenletnek a megfelelő kezdőfeltételekhez illesztett megoldásai az úgynevezett Airy-függvények, $\Ai(x)$ és $\Bi(x)$.

Az Airy-függvények szorosan kapcsolódnak a Bessel-függvényekhez. Ez elentős mind az aszimptotikus alakjuk meghatározásához, mind a függvények numerikus kiértékeléséhez. A megoldást
\begin{equation}
	y(x) = x^{\frac{1}{2}}v\left(\frac{2}{3}x^{\frac{3}{2}}\right)
\end{equation}
alakban keresve a $x \geq 0$ tartományban a $v(x)$-re vonatkozó egyenlet a módosított Bessel-egyenlet $t=\frac{2}{3}x^{\frac{3}{2}}$ bevezetésével.
\begin{equation}
	t^2\frac{d^2v(t)}{dt^2} + t\frac{dv(t)}{dt} - \left(t^2 + \frac{1}{9}\right)v(t) = 0
\end{equation}
Leolvasható, hogy $\nu^2 = \frac{1}{9}$, azaz a $v(x)$-re vonatkozó egyenlet megoldásai az $I_{\frac{1}{3}}(x)$ és $I_{-\frac{1}{3}}(x)$ módosított Bessel-függvények lineáris kombinációi.
A két hagyományosan választott lineáris kombinációk a következőek:
\begin{equation}
	\Ai(x) = \frac{\sqrt{x}}{3}\left(I_{-\frac{1}{3}}\left(\frac{2}{3}x^{\frac{3}{2}}\right)-I_{\frac{1}{3}}\left(\frac{2}{3}x^{\frac{3}{2}}\right)\right)
	\label{airy:ai+}
\end{equation}
\begin{equation}
	\Bi(x) = \sqrt{\frac{x}{3}}\left(I_{-\frac{1}{3}}\left(\frac{2}{3}x^{\frac{3}{2}}\right)+I_{\frac{1}{3}}\left(\frac{2}{3}x^{\frac{3}{2}}\right)\right).
	\label{airy:ai+}
\end{equation}
$x \leq 0$ tartományban
\begin{equation}
	y(x) = (-x)^{\frac{1}{2}}v\left(\frac{2}{3}(-x)^{\frac{3}{2}}\right)
\end{equation}
alakban keresve a megoldást a $v(x)$-re kapott egyenlet a Bessel-egyenlet, megint $\nu^2 = \frac{1}{9}$.
\begin{equation}
	t^2\frac{d^2v(t)}{dt^2} + t\frac{dv(t)}{dt} + \left(t^2 - \frac{1}{9}\right)v(t) = 0
\end{equation}
Az $x=0$ pontban megkövetelt analitikusságnak megfelelően $x \geq 0$ esetén
\begin{equation}
	\Ai(-x) = \frac{\sqrt{x}}{3}\left(J_{-\frac{1}{3}}\left(\frac{2}{3}x^{\frac{3}{2}}\right)-J_{\frac{1}{3}}\left(\frac{2}{3}x^{\frac{3}{2}}\right)\right)
	\label{airy:ai-}
\end{equation}
\begin{equation}
	\Bi(-x) = \sqrt{\frac{x}{3}}\left(J_{-\frac{1}{3}}\left(\frac{2}{3}x^{\frac{3}{2}}\right)+J_{\frac{1}{3}}\left(\frac{2}{3}x^{\frac{3}{2}}\right)\right),
	\label{airy:bi-}
\end{equation}
ahol $J_\nu(x)$ a Bessel-függvények.
\begin{figure}
	\centering
	\includegraphics[scale=1]{./figs/airy.pdf}
	\caption[Airy-függvények]{$\Ai(x)$ és $\Bi(x)$ grafikonja.}
\end{figure}
Érdemes definiálni a
\begin{equation}
	\Ti(x) = \frac{\Ai(x)}{\Bi(x)}
\end{equation}
függvényt.

$x \to \infty$ aszimptotikus alak:
\begin{equation}
	\Ai\left(-x\right) = \frac{1}{\sqrt{\pi}x^{1/4}}\cos\left(\frac{2}{3}x^{3/2} - \frac{\pi}{4}\right) + \mathcal{O}\left(x^{-5/4}\right)
\end{equation}
\begin{equation}
	\Bi\left(-x\right) = -\frac{1}{\sqrt{\pi}x^{1/4}}\sin\left(\frac{2}{3}x^{3/2} - \frac{\pi}{4}\right) + \mathcal{O}\left(x^{-5/4}\right)
\end{equation}
\begin{equation}
	\Ti\left(-x\right) = -\cot\left( \frac{2}{3}x^{3/2} - \frac{\pi}{4} \right) + \mathcal{O}\left(x^{-5/4}\right)
\end{equation}
\begin{equation}
	\Ai(x) = \frac{1}{2\sqrt{\pi}x^{1/4}}e^{-\frac{2}{3}x^{\frac{3}{2}}}+\mathcal{O}\left(x^{-5/4}\right)
\end{equation}
\begin{equation}
	\Bi(x) = \frac{1}{ \sqrt{\pi}x^{1/4}}e^{ \frac{2}{3}x^{\frac{3}{2}}}+\mathcal{O}\left(x^{-5/4}\right)
\end{equation}

Az állapotok normájának kiszámításához szükség van az Airy-függvények szorzatának integráljára. \cite{Albright_1977} (A.16) szerint
\begin{equation}
	\int y^2\;dx = xy^2 - {y^\prime}^2,
	\label{airy:normintegral}
\end{equation}
ahol $y$ az Airy egyenlet tetszőleges megoldása.

A Green-függvény meghatározása közben felmerül a Wronski-determinánsa az Airy-függvényeknek, ez \cite{NIST:DLMF} (9.2.7) szerint
\begin{equation}
	\mathcal{W} \{ \Ai(x), \Bi(x) \} = \Ai(x)\Bip(x) - \Bi(x)\Aip(x) = \frac{1}{\pi}.
	\label{airy:wronski}
\end{equation}









		\subsubsection{Véges $F$ eset}
			\Aeqref{airy:airyeq} egyenlet \eqref{3dbox:1deq} alakúra hozható a
\begin{equation}
	x = ax^\prime - bE,
\end{equation}
\begin{equation}
	y(x) = y(ax^\prime - bE)
\end{equation}
helyettesítésekkel. A helyettesítés után $\frac{d}{dx} = \frac{1}{a}\frac{d}{dx^\prime}$, és \aeqref{airy:airyeq} alakja
\begin{equation}
	\frac{d^2y(ax-bE)}{{dx^\prime}^2} - \left(a^3x - a^2bE\right)y(ax-bE) = 0.
\end{equation}
Ezt az egyenletet összevetve \eqref{3dbox:1deq} egyenlettel $a$ és $b$ értéke leolvasható,
\begin{equation}
	a = \sqrt[3]{\frac{2mF}{\hbar^2}},
\end{equation}
\begin{equation}
	b = \sqrt[3]{\frac{2m}{\hbar^2F^2}}.
\end{equation}
Az egy dimenziós időfüggetlen Schrödinger-egyenlet megoldása
\begin{equation}
	\psi(x) = c_1\Ai(ax-bE)+c_2\Bi(ax-bE),
\end{equation}
melyet a határfeltételekhez kell illeszteni,
\begin{equation}
	\psi(0) = \psi(L) = 0.
\end{equation}
A $\psi(0) = 0$ feltételből következik, hogy $\psi \propto \Bi(-bE)\Ai(ax-bE) - \Ai(-bE)\Bi(ax-bE)$. A második határfeltétel pedig meghatározza a lehetséges energiákat,
\begin{equation}
		0 = \psi(L) = \Bi(-bE)\Ai(aL-bE) - \Ai(-bE)\Bi(aL-bE).
\end{equation}
Felhasználva a $\Ti(x)$ függvényt, az egyenlet kompakt és jól közelíthető alakra hozható,
\begin{equation}
	\label{box_energiaszintek_egyenlet}
	\Ti(aL-bE) - \Ti(-bE) = 0.
\end{equation}
\begin{figure}[H]
	\centering
	\includegraphics[scale=1]{./figs/energiaszintek.pdf}
	\caption[Egzakt energiaszintek]{Egzakt energia szintek, $bE$ és $aL$ közötti relációval ábrázolva. Az ába jobb alsó sarkán látható, hogy $E \ll FL$ esetén az energiaszintek $L$-től függetlenek lesznek, mivel a félvégtelen tér beli homogén tér energiaszintjeit közelítik.}
	\label{box_energiaszintek_abra}
\end{figure}
Amikor $FL \ll \frac{\pi^2\hbar^2}{2mL^2}$, a potenciál jól közelíthető konstans potenciállal, mivel az alapállapot energiájához képest is elhanyagolható a lineáris potenciál eltérése a konstans potenciáltól. Eben a esetben $E \propto n^2$. $E \ll FL$ esetben az energiaszintek jó közelítéssel konstanssá válnak. Ennek az oka, hogy $\lim_{L \to \infty}\psi(x) = \alpha \Ai\left(ax-b\right)$, mert a $\Bi\left(x\right)$ exponenciálisan növekszik nagy $x$-ek esetén. Ebben az eseten az energiaszinteket a $\Ai(-bE) = 0$ egyenlet határozza meg. Ezeket az aszimptotikus viselkedéseket \aref{box_energiaszintek_abra}. ábra jól mutatja, később a Szemiklasszikus közelítés vizsgálata során részletesebben tárgyaljuk.

\begin{equation}
	\psi_k(x) = \Bi(-bE_k)\Ai(ax-bE_k) - \Ai(-bE_k)\Bi(ax-bE_k)
\end{equation}
sajátállapotokhoz tartozó normálás analitikusan meghatározható. Mivel $\psi_k$ sajátállapotok valós értékűek, $\left|\psi_k(x)\right|^2 = \psi_k(x)^2$, így \aeqref{airy:normintegral} egyenlet közvetlenül alkalmazható,
\begin{dmath}
	N_k = \int_0^Ldx\,\left|\psi_k(x)\right|^2 = \left.\left(x-\frac{bE}{a}\right)\psi_k(x)^2 - \frac{1}{a^3}\psi_k^\prime(x)^2\right|_{x=0}^{x=L} = \frac{1}{a}\left(\frac{1}{\pi^2}-\left(\Bi(-bE)\Aip(aL-bE)-\Ai(-bE)\Bip(aL-bE)\right)^2\right).
\end{dmath}
A $\psi_k$-t tartalmazó tagok kiesnek a határokon, mert a határfeltételeknek megfelelően $\psi_k=0$ $x=0$ és $x=L$-ben. A maradék tag $x=0$-beli értéke $\frac{1}{\pi^2}$ az Airy-függvények Wronski-determinánsa \eqref{airy:wronski} miatt. \Aref{vegesf:eigenstates}. ábra az első néhány sajátállapotot szemlélteti,  $1$-re normálva az $N_k$ együtthatók segítségével.
\begin{figure}[H]
	\centering
	\includegraphics[scale=1]{./figs/allapotok.pdf}
	\caption[sajátállapotok]{Az első $4$ energia sajátállapot $aL=8$ hosszúságú doboz esetén, $1$-re normálva, azaz $\frac{1}{\sqrt{N_n}}\psi_n(x)$ függvényeket ábrázolja. ($n=0,1,2,3$)}
	\label{vegesf:eigenstates}
\end{figure}
	
%A probléma egy 1D dobozba zárt résecske homogén erőtérben. $F(x)=-F$, azaz $V(x) = Fx$.
%	Az egyenlethez tartozó határfeltételek, ha a doboz hossza $L$:
%	\begin{equation}
%		\phi \big\rvert_0 = \phi \big \rvert_L = 0
%	\end{equation}
%	A megoldandó időfüggetlen Schrödinger-egyenlet:
%	\begin{equation}
%		-\frac{\hbar^2}{2m}\frac{d^2\phi}{dx^2} + Fx\phi = E\phi
%	\end{equation}
%	\begin{equation}
%		\frac{d^2\phi}{dx^2} - \frac{2mFx}{\hbar^2}\phi = -\frac{2mE}{\hbar^2}\phi
%	\end{equation}
%	\begin{equation}
%		\frac{d^2\phi}{dx^2} - \left(\frac{2mF}{\hbar^2}x - \frac{2mE}{\hbar^2}\right)\phi = 0
%	\end{equation}
%	Az Airy egyenlet ilyen alakra hozható a változó affin lineáris transzformációjával:
%	\begin{equation}
%		\frac{d^2y}{dx^{\prime 2}} - x^\prime y = 0
%	\end{equation}
%	$x^\prime = ax - bE$, azaz $\frac{d}{dx} = a\frac{d}{dx^\prime}$:
%	\begin{equation}
%		\frac{d^2y}{dx^2} - \left(a^3x - a^2bE\right)y = 0
%	\end{equation}
%	Az együtthatók összevetése alapján $a = \sqrt[3]{\frac{2mF}{\hbar^2}}$ és $b = \sqrt[3]{\frac{2m}{\hbar^2F^2}}$. Így a Schrödinger-egyenlet megoldása:
%	\begin{equation}
%		\phi(x) = y(x^\prime) = y\left(\sqrt[3]{\frac{2mF}{\hbar^2}}x - \sqrt[3]{\frac{2m}{\hbar^2F^2}}E\right)
%	\end{equation}
%	, ahol $y(x) = \alpha \Ai\left(x\right) + \beta \Bi\left(x\right)$.
%	A $\phi \big\rvert_0 = 0$ feltételből következik, hogy $\phi \propto \Bi\left(-bE\right)\Ai\left(ax-bE\right) - \Ai\left(-bE\right)\Bi\left(ax-bE\right)$. A második határfeltétel pedig meghatározza a lehetséges energiákat. A feltétel:
%	\begin{equation}
%		\Bi\left(-bE\right)\Ai\left(aL-bE\right) - \Ai\left(-bE\right)\Bi\left(aL-bE\right) = 0
%	\end{equation}
%	\begin{equation}
%		\label{box_energiaszintek_egyenlet}
%		\Ti{aL-bE} - \Ti{-bE} = 0
%	\end{equation}
%	\begin{equation}
%		\Ti{\sqrt[3]{\frac{2mF}{\hbar^2}}L - \sqrt[3]{\frac{2m}{\hbar^2F^2}}E} - \Ti{-\sqrt[3]{\frac{2m}{\hbar^2F^2}}E} = 0
%	\end{equation}
%	\begin{figure}[H]
%		\includegraphics[scale=1]{./figs/energiaszintek.pdf}
%		\caption[Egzakt energiaszintek]{Egzakt energia szintek, $bE$ és $aL$ közötti relációval ábrázolva. Az ába jobb alsó sarkán látható, hogy $E \ll FL$ esetén az energiaszintek $L$-től függetlenek lesznek, mivel a félvégtelen tér beli homogén tér energiaszintjeit közelítik.}
%		\label{box_energiaszintek_abra}
%	\end{figure}
%	Amikor $FL \ll \frac{\pi^2\hbar^2}{2mL^2}$, a potenciál jól közelíthető konstans potenciállal, mivel az alapállapot energiájához képest is elhanyagolható a lineáris potenciál eltérése a konstans potenciáltól. Eben a esetben $E \propto n^2$. $E \ll FL$ esetben az energiaszintek jó közelítéssel konstanssá válnak. Ennek az oka, hogy $\lim_{L \to \infty}\psi(x) = \alpha \Ai\left(ax-b\right)$, mert a $\Bi\left(x\right)$ exponenciálisan növekszik nagy $x$-ek esetén. Ebben az eseten az energiaszinteket a $\Ai\left(- \sqrt[3]{\frac{2m}{\hbar^2F^2}}E\right) = 0$ egyenlet határozza meg. Ezeket az aszimptotikus viselkedéseket \aref{box_energiaszintek_abra}. ábra jól mutatja.
%    
%    TODO: link 1D videóról
%
		\subsubsection{Falak nélküli eset}
			\label{nowall}
Falak hiányában a Schrödinger-egyenlet továbbra is \eqref{3dbox:1deq}, azonban a határfeltételek különböznek. A fizikai kép az, hogy $V(x)=Fx$ potenciál esetén az $x\to\infty$-ből nem jönnek részecskék, és nem is tartózkodnak ott. Ezek problémás állapotok lennének, végtelen energiával rendelkeznének. Tehát a szórásállapotokra vonatkozó feltétel, hogy
\begin{equation}
	\lim_{x\to\infty}\psi(x) = 0.
	\label{nowall:boundary}
\end{equation}
Mivel itt folytonos spektrumról van szó, az eddigi normálás helyett az állapotokat Dirac-deltára kell normálni. Ebben a feladatban az energia és energia sajátállapot között egy az egyhez megfeleltetés van, ellenben a jól ismert szabad részecske esetével. Ennek oka, hogy itt $x\to\infty$-ből nem jönnek részecskék. Ennek következtében az a sajátállapotokat $\Ket{E}$ egyértelmen jelöli.
\Aeqref{nowall:boundary} feltétel azt jelenti, hogy az Airy-függvények közül a $\Bi(ax-bE)$ nem szerepel a lineáris kominációban, a megoldás tisztán az $\Ai(ax-bE)$ függvény lesz,
\begin{equation}
	\Braket{x|E}=N\Ai(ax-bE).
\end{equation}
A szórásállapotokra vonatkozó normálási feltétel
\begin{equation}
	\Braket{E|E^\prime}=\delta(E-E^\prime).
\end{equation}
Ez alapján $N$ meghatározható \eqref{airy:delta} azonosság felhasználásával,
\begin{dmath}
	\delta(E-E^\prime)=N^2\int_{-\infty}^\infty\Ai(ax-bE)\Ai(ax-bE^\prime)\,dx=N^2\frac{1}{ab}\delta(E-E^\prime).
	\label{nowall:orthog}
\end{dmath}
Ez alapján $N=\sqrt{ab}=\sqrt[3]{\frac{2m}{\hbar^2\sqrt{F}}}$, és
\begin{equation}
	\Braket{x|E}=\psi_E(x)=\sqrt{ab}\Ai(ax-bE).
	\label{nowell:sajátfüggvény}
\end{equation}
A teljességi reláció is leellenőrizhető \aeqref{airy:delta} egyenlet alapján,
\begin{dmath}
	\int_{-\infty}^\infty dE\,\Ket{E}\Bra{E}=ab\int_{-\infty}^\infty dE\int_{-\infty}^\infty dx\int_{-\infty}^\infty dy\,\Ai(ax-bE)\Ai(ay-bE)\Ket{x}\Bra{y}=\int_{-\infty}^\infty dx\int_{-\infty}^\infty dy\,\delta(x-y)\Ket{x}\Bra{y}=\op{I}.
	\label{nowall:comleteness}
\end{dmath}
\Aeqref{nowall:orthog} egyenlet a $\op{H}$ operátor hermitikusságából következik, hiszen a hermitikus operátorok sajátállapotai ortogonálisak egymásra. \Aeqref{nowall:comleteness} teljességi reláció is arra utal, hogy az összes fizikai sajátállapotot megtaláltuk a csupán $\Ai(x)$ függvényt tartalmazó állapotok keresésével. Ha hiányozna valamely fizikai állapot, akkor nem lehetne a megtalált sajátfüggvények lineáris kombinációjaként tetszőleges hullámfüggvényt előállítani, és így a teljességi reláció nem teljesülne.

Érdemes a fizikai intuícióval összevetni az Airy-függvény Fourier-transzformáltját. Az Airy-függény Fourier transzformáltja
\begin{equation}
	\int_{-\infty}^\infty\Ai(x)e^{-ikx}\,dx=e^{ik^3/3}.
\end{equation}
Ez azt jelenti, hogy az impulzus térben a hullámfüggvény
\begin{equation}
	\psi_E(p)=\frac{1}{\sqrt{2\pi F\hbar}}\exp\left(i\left(\frac{1}{3}\left(\frac{p}{a\hbar}\right)^3-\frac{pE}{F\hbar}\right)\right),
\end{equation}
\begin{equation}
	\rvert\psi_E(p)\lvert^2=\frac{1}{2\pi F\hbar}.
\end{equation}
Az impulzus hullámfüggvény amplitúdója nem függ az impulzustól! Ez nem meglepő, mert a klasszikus esetben az impulzus időfejlődése 
\begin{equation}
	p(t)=-Ft+p_0,
\end{equation}
tehát minden részecske egy kis $dp$ tartományban $dp/F$ időt tölt, adott impulzushoz tartozó részecskesűrűség értéke független az impulzustól. Ennek a klasszikus fizika beli megállapításnak a megfelelője, hogy $\lvert\psi_E(p)\rvert^2$ $p$-től független.







\section{Szemiklasszikus közelítés}
	\begin{equation}
		nh = \oint p \, dq = 
	\end{equation}
	$E/F < L$ esete:
	\begin{equation}
		2\int_0^{E/F}\sqrt{2m\left( E-Fx \right)}\,dx = -\frac{2}{3mF}\left(2m\left( E-Fx \right)\right)^{\frac{3}{2}}\bigg \rvert_0^{E/F} = \frac{4\sqrt{2m}E^{3/2}}{3F}
	\end{equation}
	\begin{equation}
		E_n = \left(\frac{3nhF}{4\sqrt{2m}}\right)^{2/3}
	\end{equation}
	$E/F > L$ esete:
	\begin{equation}
		-\frac{2}{3mF}\left(2m\left( E-Fx \right)\right)^{\frac{3}{2}}\bigg \rvert_0^{L} = \frac{4\sqrt{2m}}{3F}\left(E^{3/2} - \left(E - FL\right)^{3/2}\right) = nh
	\end{equation}
	$E \gg FL$ esetén a különbség az $E^{3/2}$ függvény deriváltjának segítségével helyettesíthető:
	\begin{equation}
		nh \approx 2\sqrt{2m}E^{1/2}L
	\end{equation}
	\begin{equation}
		E_n \approx \frac{n^2h^2}{8mL^2}
	\end{equation}
	
	\begin{figure}[H]
		\includegraphics[scale=1]{./figs/energiaszintkozelites.pdf}
		\caption[Szemiklasszikus energiaszintek]{Az ábra a szemiklasszikus energiaszinteket hasonlítja össze az egzakt energiaszintekkel. Ez az ábra is a $bE$ és $aL$ közötti relációt ábrázolja. A szemiklasszikus közelítés nagy kvantumszámok illetve $E \gg FL$ esetén pontos. Utóbbi oka, hogy ebben az esetben a potenciál elhanyagolható, és a potenciál nélküli végtelen potenciálgödör energiaszintjeit pedig a szemiklasszikus közelítés egzaktul megadja.}
	\end{figure}
	
	\begin{figure}[H]
		\includegraphics[scale=1]{./figs/infsquareenergia.pdf}
		\caption[Végtelen potenciálgödör energiaszintjei]{Az ábrán a végtelen potenciálgödör és az egzakt energiaszintek összehasonlítása látható. Ez csak az $E \gg FL$ esetben jó közelítés, a szemiklasszikus energiaszintek jóval pontosabbak.}
	\end{figure}


	\subsection{Szemiklasszikus energiaszintek}
		\input
		\subsubsection{Összehasonlítás az egzakt eredménnyel}
			$x \rightarrow \infty$ aszimptotikus alak:
	\begin{equation}
		\Ai{-x} = \frac{1}{\sqrt{\pi}x^{1/4}}\cos\left(\frac{2}{3}x^{3/2} - \frac{\pi}{4}\right) + \mathcal{O}\left(x^{-5/4}\right)
	\end{equation}
	\begin{equation}
		\Bi{-x} = -\frac{1}{\sqrt{\pi}x^{1/4}}\sin\left(\frac{2}{3}x^{3/2} - \frac{\pi}{4}\right) + \mathcal{O}\left(x^{-5/4}\right)
	\end{equation}
	\begin{equation}
		\Ti{-x} = -\cot\left( \frac{2}{3}x^{3/2} - \frac{\pi}{4} \right) + \mathcal{O}\left(x^{-5/4}\right)
	\end{equation}
	
	Ezzel a közelítéssel \aref{box_energiaszintek_egyenlet}. egyenlet alakja:
	\begin{equation}
		\cot\left(\frac{2}{3}\left(b-aL\right)^{3/2} - \frac{\pi}{4}\right) = \cot\left(\frac{2}{3}b^{3/2} - \frac{\pi}{4}\right)
	\end{equation}
	, azaz
	\begin{equation}
		\frac{2}{3}b^{3/2} - \frac{2}{3}\left(b-aL\right)^{3/2} = n\pi
	\end{equation}
	. Az $a$ és $b$ behelyettesítésével az egyenlet
	\begin{equation}
		\frac{2\sqrt{2m}}{3F\hbar}\left(E^{3/2} - \left(E - FL\right)^{3/2}\right) = n\pi
	\end{equation}
	Ez megegyezik a szemiklasszikus kvantálással kapott eredménnyel, ami azt jelenti, hogy a szemiklasszikus közelítés jól működik nagy energiáknál, hibája $\mathcal{O}\left(E^{-5/4}\right)$ nagyságrendű.


	\subsection{Airy függvények aszimptotikája}
		\input
\section{Homogén tér Green-függvénye}
	A reolvens operátor definíciója
\begin{equation}
    \op{G}\left( E \right) = \frac{1}{\op{H} - E}
\end{equation}
és ezen operátorhoz tartozó két változós függvény a Green-függény.
\begin{equation}
    G\left( x, y; E \right) = \Bra{x}G\left(E\right)\Ket{y}
\end{equation}
A Green-függvény név indokolt, és ennek a segítségével fogom meghatározni a Green-függvényeket konkrét esetben. A teljességi reláció beszúrásával látható, hogy a kvantummechanikai Green-függény megegyezik a differenciálegyenletek elméletéből ismert Green-függvénnyel.
\begin{equation}
    \left(\op{H} - E\right) \op{G}\left( E \right) = \op{I}
\end{equation}

\begin{equation}
    \int \mathrm{d}x^\prime \Bra{x}\left(\op{H} - E\right) \Ket{x^\prime}\Bra{x^\prime} \op{G}\left( E \right)\Ket{y} = \Bra{x}\op{I}\Ket{y} = \delta \left(x - y\right)
\end{equation} 
A $\Bra{x}\left(\op{H} - E\right) \Ket{x^\prime}$ maggal vett konvolúció a $\op{H} - E$ operátor hatása. Ezért
\begin{equation}
    \left(\op{H}_x - E\right) G\left(x, y; E\right) = \delta\left(x - y\right)
    \label{green:deltaeq}
\end{equation}
ami a differenciálegyenletek elméletéből ismert Green-függvény definíciója. Ebben a konkrét esetben
\begin{equation}
    \left( -\frac{\hbar^2}{2m}\frac{\partial^2}{\partial x^2} + Fx - E \right) G\left(x, y; E\right) = \delta\left(x - y\right)
\end{equation}
ami azt jelenti, hogy az $x < y$ tartományban
\begin{equation}
    G\left(x, y; E\right) = C_1 \Ai{\sqrt[3]{\frac{2mF}{\hbar^2}}x - \sqrt[3]{\frac{2m}{\hbar^2F^2}}E} + C_2 \Bi{\sqrt[3]{\frac{2mF}{\hbar^2}}x - \sqrt[3]{\frac{2m}{\hbar^2F^2}}E}
\end{equation}
illetve az $x > y$ tartományban
\begin{equation}
    G\left(x, y; E\right) = C_3 \Ai{\sqrt[3]{\frac{2mF}{\hbar^2}}x - \sqrt[3]{\frac{2m}{\hbar^2F^2}}E} + C_4 \Bi{\sqrt[3]{\frac{2mF}{\hbar^2}}x - \sqrt[3]{\frac{2m}{\hbar^2F^2}}E}
\end{equation}
, ahol a $C$ együtthatók függhetnek $y$ és $E$ értékétől. A $C$ együtthatók meghatározásához a doboz eredeti határfeltételeit $x = 0$ és $x = L$ pontban, valamint az $x = y$ pontban \aref{green:deltaeq}. egyenlet $y$ körüli integrálásából kapot feltételeket kell felhasználni. 

















\appendix
\section{Szabad részecske gyorsuló koordinátarendszerben}
	\input{tex/accelerating.tex}
\section{Numerikus számítások}
	\subsection{Momentumok időfejlődése}
		\begin{figure}[H]
	\includegraphics[scale=1]{./figs/expectations.pdf}
	\caption{Várható értékek és szórások időfejlődése}
\end{figure}

	\subsection{Hullámfüggvény időfejlődése}
		\subsubsection{1D}
		\subsubsection{2D}

%	Azokat a parmaétereket keresem, ahol az alapállapot $E = FL$:
%	\begin{equation}
%		\Ti{\sqrt[3]{\frac{2mF}{\hbar^2}}L - \sqrt[3]{\frac{2m}{\hbar^2F^2}}FL} - \Ti{-\sqrt[3]{\frac{2m}{\hbar^2F^2}}FL} = 0
%	\end{equation}
%	, azaz
%	\begin{equation}
%		\Ai{-\sqrt[3]{\frac{2mF}{\hbar^2}}L} = 0
%	\end{equation}
%	. Az első gyöke az Airy függvénynek megadja azt az esetet, amikor az alapállapot energiája $FL$, és nem pedig valamelyik gerjesztett állapoté.
%	\begin{equation}
%		-a_1 = \sqrt[3]{\frac{2mF}{\hbar^2}}L \approx 2.338
%	\end{equation}
	
    \newpage
	\phantomsection
	\bibliographystyle{abeld}
	\addcontentsline{toc}{section}{Hivatkozások}
    \bibliography{tex/ref}
\end{document}















