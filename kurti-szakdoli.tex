\documentclass[pdftex,12pt,a4paper]{article}
\pdfpagewidth 8.5in
\pdfpageheight 11.6in
\linespread{1.3}
\usepackage{anysize}
\marginsize{2.5cm}{2.5cm}{2.5cm}{2.5cm}

\usepackage[utf8]{inputenc}
\usepackage[T1]{fontenc}
\usepackage[magyar]{babel}
\usepackage{indentfirst}
\usepackage{amsmath}
\usepackage{float}
\usepackage{graphicx}
\usepackage{braket}
\usepackage[unicode,pdftex]{hyperref}
\usepackage{hyperref}
\usepackage{breqn}

\usepackage{listings}
\usepackage{xcolor}

\definecolor{codegreen}{rgb}{0,0.6,0}
\definecolor{codegray}{rgb}{0.3,0.3,0.3}
\definecolor{codepurple}{rgb}{0.58,0,0.82}
\definecolor{backcolour}{rgb}{0.90,0.90,0.87}

\lstdefinestyle{mystyle}{
    backgroundcolor=\color{backcolour},   
    commentstyle=\color{codegreen},
    keywordstyle=\color{magenta},
    numberstyle=\small\color{codegray},
    stringstyle=\color{codepurple},
    basicstyle=\ttfamily\small,
    breakatwhitespace=false,         
    breaklines=true,                 
    captionpos=b,                    
    keepspaces=true,                 
    numbers=left,                    
    numbersep=5pt,                  
    showspaces=false,                
    showstringspaces=false,
    showtabs=false,                  
    tabsize=2
}
\lstset{style=mystyle}

\DeclareMathOperator{\Ai}{Ai}
\DeclareMathOperator{\Bi}{Bi}
\DeclareMathOperator{\Aip}{Ai^\prime}
\DeclareMathOperator{\Bip}{Bi^\prime}
\DeclareMathOperator{\Ti}{Ti}
\DeclareMathOperator{\sgn}{sgn}
%\DeclareMathOperator{\max}{max}
\let\Im\relax
\DeclareMathOperator{\Im}{Im}
\DeclareMathOperator{\Tr}{Tr}
\newcommand{\op}[1]{\hat{#1}}
\newcommand{\norm}[1]{\left\lVert #1 \right\rVert}
\newcommand*\Laplace{\mathop{}\!\mathbin\bigtriangleup}

\newcommand{\aeqref}[1]{\az{\eqref{#1}}}
\newcommand{\Aeqref}[1]{\Az{\eqref{#1}}}

\hypersetup{
    colorlinks,
    citecolor=black,
    filecolor=black,
    linkcolor=black,
    urlcolor=black
}
\hypersetup{	
	pdftitle={Schrödinger macskája dobozban: falak közé zárt részecske kvantumállapotai homogén külső térben},
	pdfsubject={Kvantum mechanika lineáris potenciálban falak között.},
	pdfauthor={Kürti Zoltán}}

\frenchspacing
\begin{document}
\numberwithin{equation}{section}
\numberwithin{figure}{section}
\numberwithin{table}{section}
\addtolength{\marginparwidth}{50pt}

\pdfbookmark[1]{Címlap}{cim}
\pagenumbering{roman}
\label{cimlap}
\thispagestyle{empty} 

\null\vskip-0.8truein
\centerline{\Large\sc Szakdolgozat}\vskip0.6truein

\centerline{\bf\LARGE Falak közé zárt kvantum részecske homogén térben:}\vskip0.15truein
\centerline{\bf\LARGE "Schrödinger macskája dobozban"}

\vskip0.4truein\centerline{\Large\sc Kürti Zoltán}\vskip0.10truein
\centerline{\Large\sl Fizika BSc., fizikus szakirány}\vskip0.06truein
%\centerline{\Large\sl III. évfolyam }\vskip0.3truein


%\centerline{\psfig{file= ./fig/elte.pdf}}
%\centerline{\psfig{file=../fig/elte.ps}}
%\centerline{\psfig{file=./fig/elte_cimer_szines.jpg}}
\centerline{\includegraphics[scale=0.5]{./figs/elte_cimer_color.pdf}}
\vskip0.4truein
\centerline{\Large Témavezetők:}\vskip0.2truein
\centerline{\Large{\sc{ Dr. Cserti József}} }\vskip0.001truein
\centerline{egyetemi tanár}\vskip0.15truein
\centerline{\Large\sc Dr. Györgyi Géza}\vskip0.001truein
\centerline{egyetemi docens}\vskip0.2truein
\centerline{\Large \sc \bf Eötvös Loránd Tudományegyetem}\vskip0.010truein
\centerline{\Large  Komplex Rendszerek Fizikája Tanszék}\vskip0.15truein
\centerline{\Large\bf 2021}
\newpage

\begin{abstract}
	Kvantummechanikai iskolapélda a homogén térbe helyezett egydimenziós
	részecske. Ezt három dimenzióra kiterjesztve és két fal közé zárva
	keressük az energia sajátállapotokat. Annyi előrelátható, hogy a nyílt
	vagy félig nyílt esetekben használható, reguláris Airy függvény itt nem
	elegendő a megoldáshoz, ennyiben túlmegyünk a tankönyvi feladaton. Az
	aszimptotikus függvényalakok segítségével előállítjuk a magasan
	gerjesztett állapotok energiáit és hullámfüggvényeit, s ezeket
	összehasonlítjuk a közvetlenül a Bohr--Sommerfeld-módszerrel kapott
	eredménnyel. Numerikusan szemléltetjük fizikailag érdekes kezdőállapotok
	időfejlődését. Vizsgáljuk a rezolvenst és az állapotsűrűséget.%, továbbá a sokrészecske rendszerekre való általánosítás lehetőségét.
	
%\centerline{\bf Köszönetnyilvánítás }\vskip0.15truein
	
\end{abstract}

\newpage
\phantomsection
\pdfbookmark[1]{Tartalomjegyzék}{tartalom}
\tableofcontents
%\newpage
\listoffigures
\listoftables
\newpage
\pagenumbering{arabic}
\phantomsection

\section{Bevezetés}
	%nem méréselmélet
A dolgozat címében a Schrödinger macskája méréselméleti utalás ellenére nem foglalkozunk méréselméleti kérdésekkel. A cím csupán a dobozba zárt macska és a dobozba zárt és homogén térbe helyezett kvantum részecske hasonlóságára utal.

%a fizikai rendszer
A dolgozatban tárgyalt rendszer egy belső szabadsági fokokkal nem rendelkező részecske homogén erőtérben, különböző határfeltételekkel. A központi probléma a zárt doboznak megfelelő határfeltétel esete, egy vagy háromdimenzióban. Egy dimenzióban vizsgáljuk az alulról zárt, felülről nyitott dobozt, az úgyevezett "quantum bouncer"-t \cite{vankov2009quantum}, \cite{doi:10.1119/1.10024}, \cite{doi:10.1119/1.16673}. A falak nélküli csupán a lineáris potenciálnak alávetett részecske esetét \cite[137-138.o.]{Vallee:2010:AFA} is vizsgáljuk egydimenzióban.

%az irodalom, hiányzik belőle a felülről zárt eset, Bi
Az irodalomban több helyen megtalálható a "quantum bouncer" ahogy ezt előzőleg említettük. Megtalálható továbbá a \cite{Landau1981Quantum}, \cite{Griffiths2004Introduction} és \cite{Sakurai:1167961} tankönyvekben is, külön elnevezés nélkül. Utóbbi a $V=k\lvert x \rvert$ potenciált vizsgálja, ami triviális kiterjesztése a "quantum bouncer" problémának a Dirichlet-határfeltételen kívül a Neumann-határfeltétellel kapott állapotok megengedésével. A megoldásokat meghatározó egyenlet egy másodrendű lineáris differenciálegyenlet, két független megoldása az úgynevezett $\Ai$ és $\Bi$ Airy-függvények. Ezek közül az $\Ai$ korlátos, míg a $\Bi$ exponenciálisan növekszik pozitív argumentumok esetén. Az előbb említett forrásokban mind csak az $\Ai$ Airy-függvény merül fel, a $\Bi$ esetleges fizikai jelentőségéről nincs szó, a végtelen beli exponenciális növekedés miatt a $\Bi$ függvény fel sem merül. Az $\Ai$ függvény természetesen felmerül minden szemiklasszikus közelítéssel foglalkozó tankönyvben, hiszen az analitikus fordulópontokban a szemiklasszikus megoldásokat az $\Ai$ függvény aszimptotikája illeszti össze. A \cite{doi:10.1007/s12043-001-0081-1} cikkben felmerül a $\Bi$ függvény is, mivel a véges potenciálgödröt vizsgálják és ebben az esetben csak az egyik tartományból lehet kizárni a $\Bi$ függvényt a végtelenben való növekedése miatt. A dolgozatban részletesebben kidolgozzuk a cikkben említett potenciálgödör végtelen mély esetét. Érdemes megjegyezni hogy az említett rendszerek Green-függvényeiben mind felbukkan a $\Bi$ Airy-függvény, még a falak nélküli esetben is.

%didaktika
Klasszikus mechanikában a szabad részecske tárgyalása után legtöbbször az egyenletesen gyorsuló részecske tárgyalása következik, így a kvantummechanika megalapozásának szempontjából jelentős didaktikai szerepe van a lineáris potenciál alapos vizsgálatának. A $\Bi$ függvény fizikai szerepének vizsgálata így indokolt lenne a kvantummechanikába bevezető tankönyvek esetében is, azonban az elterjedt tankönyvekből ez hiányzik.

%a fizikai jelentősége a problémánk
Talán a legjelentősebb fizikai alkalmazása a lineáris potenciálnak a szilárdtest-fizikában van. \cite{Beenakker_1991}-ben számos alkalmazásra lehet példát találni, a jelenségek elméleti leírását és a kapcsolódó kísérleteket s tárgyalják. Két anyag határán vagy a külső elektromos tér, vagy az anyagi minőségek különbségei miatt a vezetési elektronokra az anyaghatárra merőleges irányban ható potenciál jó közelítéssel lineáris, alul egy végtelen potenciálugrással modellezhető potenciálgáttal. P típusú félvezetőt bevonva szigetelő réteggel, és a szigetelő réteg másik oldalára nagy pozitív feszültséget kapcsolva az effektív potenciál az előbb leírt "quantum bouncer" potenciállal írható le. Ha a kapufeszültség jól van megválasztva, a p típusú félvezető a szigetelő síkhoz közeli tartományában az elektronok betölthetnek állapotokat a vezetési sávból. Hasonló helyzet alakulhat ki megfelelően választott p és n típusú félvezetők határán, a potenciál ugrását a vezetési sáv energiájának ugrása okozza a határon, a lineáris potenciált pedig az n típusú félvezetőben a határfelület környékén felhalmozódó pozitív töltések. Fontos, hogy az utóbbi eset megvalósításához nincs feltétlenül szükség külső feszültségforrásra. Így a határfelülethez közeli vezető elektronok hullámfüggvényének merőleges helyfüggésére a "quantum bouncer" Schrödinger-egyenlet vonatkozik. Ha a Fermi-energia és $k_BT$ megfelelő értékűek, akkor a vezetési elektronok a határra merőleges irányban bezáródnak, az alap, vagy esetleg az első néhány gerjesztett állapotban lehetnek. Ekkor ez elektronokat egy kétdimenziós effektív Schrödinger-egyenlet ír le. Ha a merőleges irányban fellépnek magasabb gerjesztett állapotok, akkor azokat belső szabadsági fokként kezelve több komponensű hullámfüggvénnyel lehet modellezni. Ezeket a kétdimenzióba korlátozott vezetési elektronokat nevezik kétdimenziós elektrongáznak (2DEG). További külső potenciálokkal bonyolult geometriájú csatornákat, kapukat lehet kialakítani. Fontos, hogy a kapuk feszültségének változtatásával a kapuk illetve csatornák geometriája és erőssége elektronikusan vezérelhető. Többek között hagyományos tranzisztorok előállítására, qubitek közötti kölcsönhatások szabályozására is alkalmasak.

%dolgozat menetének leírésa
A dolgozat első részét a háromdimenziós dobozba zárt részecske tárgyalásával kezdjük, tetszőleges irányú homogén erőtérben, és három egydimenziós egyenletre redukáljuk a Schrödinger-egyenletet. A dolgozat további részében főleg az egydimenziós problémát vizsgáljuk. Az Airy-függvények alapvető matematikai tulajdonságainak ismertetése után analitikus megoldást mutatunk az egydimenziós zárt doboz esetére. Az energiaszintekre vonatkozó transzcendens egyenletet leszámítva, az energia sajátfüggvényekre és normálási faktoraikra explicit analitikus képleteket vezetünk le. Röviden tárgyaljuk a falak nélküli esetet, és a hozzá tartozó sajátállaptok normálását és teljességi relációját.
A dolgozat második részében a szemiklasszikus közelítést vizsgáljuk. Összevetjük a szemiklasszikus és egyéb közelítésekkel kapott energiaszinteket az implicit egyenletből kapott energiákkal, és megadjuk a Airy-függvények aszimptotikus viselkedését a szemiklasszikus közelítés alapján.
A dolgozat harmadik részében az egy dimenziós eset Green-függvényét vizsgáljuk. Explicit analitikus képletet vezetünk le a zárt doboz esetére. Ezen Green-függvény határeseteiként levezetjük az egy fallal határolt "quantum bouncer", és a fal nélküli rendszer Green-függvényét. Ezek a képletek explicitek. Utóbbi esetében a Green-függvény diszkrét pólusai vágássá alakulnak a komplex energiasíkon. Ez után a dobozba zárt rendszer állapotsűrűségét és a fal nélküli rendszer lokális állapotsűrűségét meghatározzuk a Green-függvényeik alapján. Végül a Green-függvények perturbációs sorát vizsgáljuk numerikusan, a zárt doboz Green-függvényén szemléltetjük, hogy a perturbációs tag triviális változtatása (az egység operátor szám szorosának levonása) drámaian javíthatja a perturbációs sor konvergencia tartományát és sebességét, valamint numerikus módszerek esetén a végeredmény pontosságát is. Végül a függelékben bemutatjuk a Schrödinger-egyenlet időfejlődését ábrázoló kód működését.
\section{A dobozba zárt részecske homogén térben}
	\input
	\subsection{Három dimenzióban}
		A rendszer egy téglatest alakú dobozba zárt részecske. A doboz mérete $L_x$, $L_y$ és $L_z$. A dobozban homogén erőtér hat a részecskére, azaz $\boldsymbol{F} = \text{const}$. A potenciál így $V(x, y, z) = -\boldsymbol{F}_xx-\boldsymbol{F}_yy-\boldsymbol{F}_zz$. A rendszer időfüggő Schrödinger-egyenlete
\begin{equation}
	i\hbar\frac{\partial \psi(x, y, z, t)}{\partial t} = -\frac{\hbar^2}{2m} \Laplace \psi(x, y, z, t) + V(x, y, z)\psi(x, y, z, t).
	\label{3dbox:3dscheq}
\end{equation}
Az egyenlet kezdőfeltétele egy kezdeti állapot $t_0$-ban, $\psi(x, y, z, t_0) = \psi_0(x, y, z)$, az egyenlet határfeltételei pedig a hullámfüggvény határokon való eltűnése, $0=\left.\psi\right|_{x=0}=\left.\psi\right|_{x=L_x}=\left.\psi\right|_{y=0}=\left.\psi\right|_{y=L_y}=\left.\psi\right|_{z=0}=\left.\psi\right|_{z=L_z}$. Mivel ez a potenciál lineáris $x$, $y$ és $z$-ben, a Schrödinger-egyenlet szeparálható a
\begin{equation}
	\psi_{klm}(x, y, z, t) = e^{-\frac{iE_{klm}t}{\hbar}}\psi^{(x)}_k(x)\psi^{(y)}_l(y)\psi^{(z)}_m(z)
	\label{3dox:3dansatz}
\end{equation}
próbafüggvénnyel. A $\psi^{(i)}_n$ ($i=x, y, z$) függvényekre így az egy dimenziós stacionárius Schrödinger-egyenlet vonatkozik. A $\psi^{(x)}$-re vonatkozó egyenlet 
\begin{equation}
	-\frac{\hbar^2}{2m}\frac{d^2\psi^{(x)}_k(x)}{dx^2} + \boldsymbol{F}_xx\psi^{(x)}_k(x) = E^{(x)}_k\psi^{(x)}_k(x),
\end{equation}
a határfeltételek $0=\left.\psi^{(x)}_k\right|_{x=0}=\left.\psi^{(x)}_k\right|_{x=L_x}$. $\psi^{(y)}_l$ és $\psi^{(z)}_m$-re vonatkozó egyenletek hasonlóak. Az $E_{klm}$ energia a három egy dimenziós stacionárius Schrödinger-egyenlet sajátenergiáinak összege,
\begin{equation}
	E_{klm} = E^{(x)}_k+E^{(y)}_l+E^{(z)}_m.
\end{equation}
\Aeqref{3dbox:3dscheq} egyenlet általános megoldása \aeqref{3dox:3dansatz} próbafüggvények kezdőfeltételhez illesztett lineáris kombinációja,
\begin{equation}
	\psi(x,y,z,t) = \sum_{klm}C_{klm}\psi_{klm}(x,y,z,t).
\end{equation}
$C_{klm}$ együtthatók meghatározásához a szokásos hely reprezentáció beli skalárszorzást kell használni,
\begin{equation}
	C_{klm} = \frac{1}{N_{klm}}\int_0^{L_x}dx\int_0^{L_y}dy\int_0^{L_z}dz\,\psi_{klm}(x, y, z, t=0)^*\psi_0(x, y, z)
	\label{3dbox:ceq}
\end{equation}
\begin{equation}
	N_{klm} = \int_0^{L_x}dx\int_0^{L_y}dy\int_0^{L_z}dz\,|\psi_{klm}(x,y,z,t=0)|^2.
	\label{3dbox:3norm}
\end{equation}
\Aeqref{3dbox:ceq} egyenlet nem egyszerűsíthető tovább általános $\psi_0$ esetén, viszont \aeqref{3dbox:3norm} igen. Mivel $\psi_{klm}$ szorzat alakú, nem kell a tripla integrált elvégezni, hanem csak három egyszeres integrál szorzatát kell kiszámítani. Ez numerikus számításokban jelentős.
\begin{equation}
	N_{klm} = N^{(x)}_kN^{(y)}_lN^{(z)}_m,
\end{equation}
ahol az egyes $N$ tagok az egy dimenziós sajátfüggvények normájaként vannak definiálva.
\begin{equation}
	N^{(x)}_k = \int_0^{L_x}dx\,\left|\psi^{(x)}_k(x)\right|^2,
\end{equation}
$N^{(y)}_l$-re és $N^{(z)}_m$-re hasonló képletek vonatkoznak. 



%    Ahol $x^\prime = \sqrt[3]{\frac{2m\boldsymbol{F}_x}{\hbar^2}}x - \sqrt[3]{\frac{2m}{\hbar^2\boldsymbol{F}_x^2}}E_k$, $E_k$ pedig az 1 dimenziós probléma, $\Ti{\sqrt[3]{\frac{2m\boldsymbol{F}_x}{\hbar^2}}L - \sqrt[3]{\frac{2m}{\hbar^2\boldsymbol{F}_x^2}}E} - \Ti{-\sqrt[3]{\frac{2m}{\hbar^2\boldsymbol{F}_x^2}}E} = 0$, $k$. $\phi_k \left( x^\prime \right)$ az 1D-s rész TODO:REFERENCIA hullámfüggvénye. $y^\prime$, $z^\prime$, valamint $E_l$ és $E_m$ hasonlóan vannak definiálva a hozzájuk tartozó 1 dimenziós probléma alapján. A 3D-s hullámfüggvényhez tartozó energia az 1D-s megoldásokhoz tartozó energiák összege.
%    \begin{equation}
%        E = E_k + E_l + E_m
%    \end{equation}
%    Amennyiben valamelyik irányú komponense $\boldsymbol{F}$-nek 0, abban az esetben a hozzá tartozó 1D-s pprobléma a híres végtelen mély potenciálgödör, ahol
%    \begin{equation}
%        \phi_n = \sqrt{\frac{2}{L}}\sin\left(\frac{nx\pi}{L}\right)
%    \end{equation}
%    valamint
%    \begin{equation}
%        E_n = \frac{\hbar^2n^2}{2mL^2}
%    \end{equation}
%    
%    TODO: ÁBRA AZ EGYSZER FÜGGŐLEGES ESET ENERGIÁIRÓL, esetleg szintén L függvényében.
%    
%    TODO: ábra 2D -quantum chaos in billiards-
%    
%    TODO: 2D (3D?) videó link időfelődésről

	\subsection{Egy dimenzióban}
			A probléma egy 1D doboba zárt résecske homogén erőtérben. $F(x)=-F$, azaz $V(x) = Fx$.
	Az egyenlethez tartozó határfeltételek, ha a doboz hossza $L$:
	\begin{equation}
		\psi \big\rvert_0 = \psi \big \rvert_L = 0
	\end{equation}
	A megoldandó időfüggetlen Schrödinger-egyenlet:
	\begin{equation}
		-\frac{\hbar^2}{2m}\frac{d^2\psi}{dx^2} + Fx\psi = E\psi
	\end{equation}
	\begin{equation}
		\frac{d^2\psi}{dx^2} - \frac{2mFx}{\hbar^2}\psi = -\frac{2mE}{\hbar^2}\psi
	\end{equation}
	\begin{equation}
		\frac{d^2\psi}{dx^2} - \left(\frac{2mF}{\hbar^2}x - \frac{2mE}{\hbar^2}\right)\psi = 0
	\end{equation}
	Az Airy egyenlet ilyen alakra hozható a változó affin lineáris transzformációjával:
	\begin{equation}
		\frac{d^2y}{dx^{\prime 2}} - x^\prime y = 0
	\end{equation}
	$x^\prime = ax - b$, azaz $\frac{d}{dx} = a\frac{d}{dx^\prime}$:
	\begin{equation}
		\frac{d^2y}{dx^2} - \left(a^3x - a^2b\right)y = 0
	\end{equation}
	Az együtthatók összevetése alapján $a = \sqrt[3]{\frac{2mF}{\hbar^2}}$ és $b = \sqrt[3]{\frac{2m}{\hbar^2F^2}}E$. Így a Schrödinger-gyenlet megoldása:
	\begin{equation}
		\psi(x) = y(x^\prime) = y\left(\sqrt[3]{\frac{2mF}{\hbar^2}}x - \sqrt[3]{\frac{2m}{\hbar^2F^2}}E\right)
	\end{equation}
	, ahol $y(x) = \alpha \Ai{x} + \beta \Bi{x}$.
	A $\psi \big\rvert_0 = 0$ feltételből következik, hogy $\psi \propto \Bi{-b}\Ai{ax-b} - \Ai{-b}\Bi{ax-b}$. A második határfeltétel pedig meghatározza a lehetséges energiákat. A feltétel:
	\begin{equation}
		\Bi{-b}\Ai{aL-b} - \Ai{-b}\Bi{aL-b} = 0
	\end{equation}
	\begin{equation}
		\label{box_energiaszintek_egyenlet}
		\Ti{aL-b} - \Ti{-b} = 0
	\end{equation}
	\begin{equation}
		\Ti{\sqrt[3]{\frac{2mF}{\hbar^2}}L - \sqrt[3]{\frac{2m}{\hbar^2F^2}}E} - \Ti{-\sqrt[3]{\frac{2m}{\hbar^2F^2}}E} = 0
	\end{equation}
	\begin{figure}[H]
		\includegraphics[scale=1]{./figs/energiaszintek.pdf}
		\caption{Energiaszintek $L$ függvényében}
		\label{box_energiaszintek_abra}
	\end{figure}
	Amikor $FL \ll \frac{\pi^2\hbar^2}{2mL^2}$, a potenciál jól közelíthető konstans potenciállal, mivel az alapállapot energiájához képest is elhanyagolható a lineáris potenciál eltérése a konstans potenciáltól. Eben a esetben $E \propto n^2$. $E \ll FL$ esetben az energiaszintek jó közelítéssel konstanssá válnak. Ennek az oka, hogy $\lim_{L \to \infty}\psi(x) = \alpha \Ai{ax-b}$, mert a $Bi(x)$ exponenciálisan növekszik nagy $x$-ek esetén. Ebben az eseten az energiaszinteket a $\Ai{- \sqrt[3]{\frac{2m}{\hbar^2F^2}}E} = 0$ egyenlet határozza meg. Ezeket az aszimptotikus viselkedéseket \aref{box_energiaszintek_abra}. ábra jól mutatja.
    
    TODO: link 1D videóról


		\subsubsection{$F=0$ eset}
			Az $F=0$ eset megoldása egyszerű, az egyik legalapvetőbb példa egyszerű kvantummechanikai rendszerekre. A sajátfüggvények
\begin{equation}
	\psi_n(x) = \sqrt{\frac{2}{L}}\sin\left(\frac{n\pi x}{L}\right),
\end{equation}
($n=1,2,\dots$), a normálási faktorok
\begin{equation}
	N_n = 1.
\end{equation}
Minden sajátfüggvény egyre normált szinusz függvény, melyek $n-1$ helyen veszik fel a $0$ értéket $x=0$ és $x=L$ között. Sajátenergiáik
\begin{equation}
	E_n = \frac{n^2\pi^2\hbar^2}{2mL^2}.
\end{equation}
Ezek az energiaszintek hasznosak lesznek a numerikus számításokban az $F\neq 0$ eseten is. 
		\subsubsection{Airy függvények}
			Az Airy egyenlet
\begin{equation}
	\frac{d^2y}{dx^2} - xy = 0,
	\label{airy:airyeq}
\end{equation}
ennek az egyenletnek a megfelelő kezdőfeltételekhez illesztett megoldásai az úgynevezett Airy-függvények, $\Ai(x)$ és $\Bi(x)$.

Az Airy-függvények szorosan kapcsolódnak a Bessel-függvényekhez. Ez elentős mind az aszimptotikus alakjuk meghatározásához, mind a függvények numerikus kiértékeléséhez. A megoldást
\begin{equation}
	y(x) = x^{\frac{1}{2}}v\left(\frac{2}{3}x^{\frac{3}{2}}\right)
\end{equation}
alakban keresve a $x \geq 0$ tartományban a $v(x)$-re vonatkozó egyenlet a módosított Bessel-egyenlet $t=\frac{2}{3}x^{\frac{3}{2}}$ bevezetésével.
\begin{equation}
	t^2\frac{d^2v(t)}{dt^2} + t\frac{dv(t)}{dt} - \left(t^2 + \frac{1}{9}\right)v(t) = 0
\end{equation}
Leolvasható, hogy $\nu^2 = \frac{1}{9}$, azaz a $v(x)$-re vonatkozó egyenlet megoldásai az $I_{\frac{1}{3}}(x)$ és $I_{-\frac{1}{3}}(x)$ módosított Bessel-függvények lineáris kombinációi.
A két hagyományosan választott lineáris kombinációk a következőek:
\begin{equation}
	\Ai(x) = \frac{\sqrt{x}}{3}\left(I_{-\frac{1}{3}}\left(\frac{2}{3}x^{\frac{3}{2}}\right)-I_{\frac{1}{3}}\left(\frac{2}{3}x^{\frac{3}{2}}\right)\right)
	\label{airy:ai+}
\end{equation}
\begin{equation}
	\Bi(x) = \sqrt{\frac{x}{3}}\left(I_{-\frac{1}{3}}\left(\frac{2}{3}x^{\frac{3}{2}}\right)+I_{\frac{1}{3}}\left(\frac{2}{3}x^{\frac{3}{2}}\right)\right).
	\label{airy:ai+}
\end{equation}
$x \leq 0$ tartományban
\begin{equation}
	y(x) = (-x)^{\frac{1}{2}}v\left(\frac{2}{3}(-x)^{\frac{3}{2}}\right)
\end{equation}
alakban keresve a megoldást a $v(x)$-re kapott egyenlet a Bessel-egyenlet, megint $\nu^2 = \frac{1}{9}$.
\begin{equation}
	t^2\frac{d^2v(t)}{dt^2} + t\frac{dv(t)}{dt} + \left(t^2 - \frac{1}{9}\right)v(t) = 0
\end{equation}
Az $x=0$ pontban megkövetelt analitikusságnak megfelelően $x \geq 0$ esetén
\begin{equation}
	\Ai(-x) = \frac{\sqrt{x}}{3}\left(J_{-\frac{1}{3}}\left(\frac{2}{3}x^{\frac{3}{2}}\right)-J_{\frac{1}{3}}\left(\frac{2}{3}x^{\frac{3}{2}}\right)\right)
	\label{airy:ai-}
\end{equation}
\begin{equation}
	\Bi(-x) = \sqrt{\frac{x}{3}}\left(J_{-\frac{1}{3}}\left(\frac{2}{3}x^{\frac{3}{2}}\right)+J_{\frac{1}{3}}\left(\frac{2}{3}x^{\frac{3}{2}}\right)\right),
	\label{airy:bi-}
\end{equation}
ahol $J_\nu(x)$ a Bessel-függvények.
\begin{figure}
	\centering
	\includegraphics[scale=1]{./figs/airy.pdf}
	\caption[Airy-függvények]{$\Ai(x)$ és $\Bi(x)$ grafikonja.}
\end{figure}
Érdemes definiálni a
\begin{equation}
	\Ti(x) = \frac{\Ai(x)}{\Bi(x)}
\end{equation}
függvényt.

$x \to \infty$ aszimptotikus alak:
\begin{equation}
	\Ai\left(-x\right) = \frac{1}{\sqrt{\pi}x^{1/4}}\cos\left(\frac{2}{3}x^{3/2} - \frac{\pi}{4}\right) + \mathcal{O}\left(x^{-5/4}\right)
\end{equation}
\begin{equation}
	\Bi\left(-x\right) = -\frac{1}{\sqrt{\pi}x^{1/4}}\sin\left(\frac{2}{3}x^{3/2} - \frac{\pi}{4}\right) + \mathcal{O}\left(x^{-5/4}\right)
\end{equation}
\begin{equation}
	\Ti\left(-x\right) = -\cot\left( \frac{2}{3}x^{3/2} - \frac{\pi}{4} \right) + \mathcal{O}\left(x^{-5/4}\right)
\end{equation}
\begin{equation}
	\Ai(x) = \frac{1}{2\sqrt{\pi}x^{1/4}}e^{-\frac{2}{3}x^{\frac{3}{2}}}+\mathcal{O}\left(x^{-5/4}\right)
\end{equation}
\begin{equation}
	\Bi(x) = \frac{1}{ \sqrt{\pi}x^{1/4}}e^{ \frac{2}{3}x^{\frac{3}{2}}}+\mathcal{O}\left(x^{-5/4}\right)
\end{equation}

Az állapotok normájának kiszámításához szükség van az Airy-függvények szorzatának integráljára. \cite{Albright_1977} (A.16) szerint
\begin{equation}
	\int y^2\;dx = xy^2 - {y^\prime}^2,
	\label{airy:normintegral}
\end{equation}
ahol $y$ az Airy egyenlet tetszőleges megoldása.

A Green-függvény meghatározása közben felmerül a Wronski-determinánsa az Airy-függvényeknek, ez \cite{NIST:DLMF} (9.2.7) szerint
\begin{equation}
	\mathcal{W} \{ \Ai(x), \Bi(x) \} = \Ai(x)\Bip(x) - \Bi(x)\Aip(x) = \frac{1}{\pi}.
	\label{airy:wronski}
\end{equation}









		\subsubsection{Véges $F$ eset}
			\Aeqref{airy:airyeq} egyenlet \eqref{3dbox:1deq} alakúra hozható a
\begin{equation}
	x = ax^\prime - bE,
\end{equation}
\begin{equation}
	y(x) = y(ax^\prime - bE)
\end{equation}
helyettesítésekkel. A helyettesítés után $\frac{d}{dx} = \frac{1}{a}\frac{d}{dx^\prime}$, és \aeqref{airy:airyeq} alakja
\begin{equation}
	\frac{d^2y(ax-bE)}{{dx^\prime}^2} - \left(a^3x - a^2bE\right)y(ax-bE) = 0.
\end{equation}
Ezt az egyenletet összevetve \eqref{3dbox:1deq} egyenlettel $a$ és $b$ értéke leolvasható,
\begin{equation}
	a = \sqrt[3]{\frac{2mF}{\hbar^2}},
\end{equation}
\begin{equation}
	b = \sqrt[3]{\frac{2m}{\hbar^2F^2}}.
\end{equation}
Az egy dimenziós időfüggetlen Schrödinger-egyenlet megoldása
\begin{equation}
	\psi(x) = c_1\Ai(ax-bE)+c_2\Bi(ax-bE),
\end{equation}
melyet a határfeltételekhez kell illeszteni,
\begin{equation}
	\psi(0) = \psi(L) = 0.
\end{equation}
A $\psi(0) = 0$ feltételből következik, hogy $\psi \propto \Bi(-bE)\Ai(ax-bE) - \Ai(-bE)\Bi(ax-bE)$. A második határfeltétel pedig meghatározza a lehetséges energiákat,
\begin{equation}
		0 = \psi(L) = \Bi(-bE)\Ai(aL-bE) - \Ai(-bE)\Bi(aL-bE).
\end{equation}
Felhasználva a $\Ti(x)$ függvényt, az egyenlet kompakt és jól közelíthető alakra hozható,
\begin{equation}
	\label{box_energiaszintek_egyenlet}
	\Ti(aL-bE) - \Ti(-bE) = 0.
\end{equation}
\begin{figure}[H]
	\centering
	\includegraphics[scale=1]{./figs/energiaszintek.pdf}
	\caption[Egzakt energiaszintek]{Egzakt energia szintek, $bE$ és $aL$ közötti relációval ábrázolva. Az ába jobb alsó sarkán látható, hogy $E \ll FL$ esetén az energiaszintek $L$-től függetlenek lesznek, mivel a félvégtelen tér beli homogén tér energiaszintjeit közelítik.}
	\label{box_energiaszintek_abra}
\end{figure}
Amikor $FL \ll \frac{\pi^2\hbar^2}{2mL^2}$, a potenciál jól közelíthető konstans potenciállal, mivel az alapállapot energiájához képest is elhanyagolható a lineáris potenciál eltérése a konstans potenciáltól. Eben a esetben $E \propto n^2$. $E \ll FL$ esetben az energiaszintek jó közelítéssel konstanssá válnak. Ennek az oka, hogy $\lim_{L \to \infty}\psi(x) = \alpha \Ai\left(ax-b\right)$, mert a $\Bi\left(x\right)$ exponenciálisan növekszik nagy $x$-ek esetén. Ebben az eseten az energiaszinteket a $\Ai(-bE) = 0$ egyenlet határozza meg. Ezeket az aszimptotikus viselkedéseket \aref{box_energiaszintek_abra}. ábra jól mutatja, később a Szemiklasszikus közelítés vizsgálata során részletesebben tárgyaljuk.

\begin{equation}
	\psi_k(x) = \Bi(-bE_k)\Ai(ax-bE_k) - \Ai(-bE_k)\Bi(ax-bE_k)
\end{equation}
sajátállapotokhoz tartozó normálás analitikusan meghatározható. Mivel $\psi_k$ sajátállapotok valós értékűek, $\left|\psi_k(x)\right|^2 = \psi_k(x)^2$, így \aeqref{airy:normintegral} egyenlet közvetlenül alkalmazható,
\begin{dmath}
	N_k = \int_0^Ldx\,\left|\psi_k(x)\right|^2 = \left.\left(x-\frac{bE}{a}\right)\psi_k(x)^2 - \frac{1}{a^3}\psi_k^\prime(x)^2\right|_{x=0}^{x=L} = \frac{1}{a}\left(\frac{1}{\pi^2}-\left(\Bi(-bE)\Aip(aL-bE)-\Ai(-bE)\Bip(aL-bE)\right)^2\right).
\end{dmath}
A $\psi_k$-t tartalmazó tagok kiesnek a határokon, mert a határfeltételeknek megfelelően $\psi_k=0$ $x=0$ és $x=L$-ben. A maradék tag $x=0$-beli értéke $\frac{1}{\pi^2}$ az Airy-függvények Wronski-determinánsa \eqref{airy:wronski} miatt. \Aref{vegesf:eigenstates}. ábra az első néhány sajátállapotot szemlélteti,  $1$-re normálva az $N_k$ együtthatók segítségével.
\begin{figure}[H]
	\centering
	\includegraphics[scale=1]{./figs/allapotok.pdf}
	\caption[sajátállapotok]{Az első $4$ energia sajátállapot $aL=8$ hosszúságú doboz esetén, $1$-re normálva, azaz $\frac{1}{\sqrt{N_n}}\psi_n(x)$ függvényeket ábrázolja. ($n=0,1,2,3$)}
	\label{vegesf:eigenstates}
\end{figure}
	
%A probléma egy 1D dobozba zárt résecske homogén erőtérben. $F(x)=-F$, azaz $V(x) = Fx$.
%	Az egyenlethez tartozó határfeltételek, ha a doboz hossza $L$:
%	\begin{equation}
%		\phi \big\rvert_0 = \phi \big \rvert_L = 0
%	\end{equation}
%	A megoldandó időfüggetlen Schrödinger-egyenlet:
%	\begin{equation}
%		-\frac{\hbar^2}{2m}\frac{d^2\phi}{dx^2} + Fx\phi = E\phi
%	\end{equation}
%	\begin{equation}
%		\frac{d^2\phi}{dx^2} - \frac{2mFx}{\hbar^2}\phi = -\frac{2mE}{\hbar^2}\phi
%	\end{equation}
%	\begin{equation}
%		\frac{d^2\phi}{dx^2} - \left(\frac{2mF}{\hbar^2}x - \frac{2mE}{\hbar^2}\right)\phi = 0
%	\end{equation}
%	Az Airy egyenlet ilyen alakra hozható a változó affin lineáris transzformációjával:
%	\begin{equation}
%		\frac{d^2y}{dx^{\prime 2}} - x^\prime y = 0
%	\end{equation}
%	$x^\prime = ax - bE$, azaz $\frac{d}{dx} = a\frac{d}{dx^\prime}$:
%	\begin{equation}
%		\frac{d^2y}{dx^2} - \left(a^3x - a^2bE\right)y = 0
%	\end{equation}
%	Az együtthatók összevetése alapján $a = \sqrt[3]{\frac{2mF}{\hbar^2}}$ és $b = \sqrt[3]{\frac{2m}{\hbar^2F^2}}$. Így a Schrödinger-egyenlet megoldása:
%	\begin{equation}
%		\phi(x) = y(x^\prime) = y\left(\sqrt[3]{\frac{2mF}{\hbar^2}}x - \sqrt[3]{\frac{2m}{\hbar^2F^2}}E\right)
%	\end{equation}
%	, ahol $y(x) = \alpha \Ai\left(x\right) + \beta \Bi\left(x\right)$.
%	A $\phi \big\rvert_0 = 0$ feltételből következik, hogy $\phi \propto \Bi\left(-bE\right)\Ai\left(ax-bE\right) - \Ai\left(-bE\right)\Bi\left(ax-bE\right)$. A második határfeltétel pedig meghatározza a lehetséges energiákat. A feltétel:
%	\begin{equation}
%		\Bi\left(-bE\right)\Ai\left(aL-bE\right) - \Ai\left(-bE\right)\Bi\left(aL-bE\right) = 0
%	\end{equation}
%	\begin{equation}
%		\label{box_energiaszintek_egyenlet}
%		\Ti{aL-bE} - \Ti{-bE} = 0
%	\end{equation}
%	\begin{equation}
%		\Ti{\sqrt[3]{\frac{2mF}{\hbar^2}}L - \sqrt[3]{\frac{2m}{\hbar^2F^2}}E} - \Ti{-\sqrt[3]{\frac{2m}{\hbar^2F^2}}E} = 0
%	\end{equation}
%	\begin{figure}[H]
%		\includegraphics[scale=1]{./figs/energiaszintek.pdf}
%		\caption[Egzakt energiaszintek]{Egzakt energia szintek, $bE$ és $aL$ közötti relációval ábrázolva. Az ába jobb alsó sarkán látható, hogy $E \ll FL$ esetén az energiaszintek $L$-től függetlenek lesznek, mivel a félvégtelen tér beli homogén tér energiaszintjeit közelítik.}
%		\label{box_energiaszintek_abra}
%	\end{figure}
%	Amikor $FL \ll \frac{\pi^2\hbar^2}{2mL^2}$, a potenciál jól közelíthető konstans potenciállal, mivel az alapállapot energiájához képest is elhanyagolható a lineáris potenciál eltérése a konstans potenciáltól. Eben a esetben $E \propto n^2$. $E \ll FL$ esetben az energiaszintek jó közelítéssel konstanssá válnak. Ennek az oka, hogy $\lim_{L \to \infty}\psi(x) = \alpha \Ai\left(ax-b\right)$, mert a $\Bi\left(x\right)$ exponenciálisan növekszik nagy $x$-ek esetén. Ebben az eseten az energiaszinteket a $\Ai\left(- \sqrt[3]{\frac{2m}{\hbar^2F^2}}E\right) = 0$ egyenlet határozza meg. Ezeket az aszimptotikus viselkedéseket \aref{box_energiaszintek_abra}. ábra jól mutatja.
%    
%    TODO: link 1D videóról
%
		\subsubsection{Falak nélküli eset}
			\label{nowall}
Falak hiányában a Schrödinger-egyenlet továbbra is \eqref{3dbox:1deq}, azonban a határfeltételek különböznek. A fizikai kép az, hogy $V(x)=Fx$ potenciál esetén az $x\to\infty$-ből nem jönnek részecskék, és nem is tartózkodnak ott. Ezek problémás állapotok lennének, végtelen energiával rendelkeznének. Tehát a szórásállapotokra vonatkozó feltétel, hogy
\begin{equation}
	\lim_{x\to\infty}\psi(x) = 0.
	\label{nowall:boundary}
\end{equation}
Mivel itt folytonos spektrumról van szó, az eddigi normálás helyett az állapotokat Dirac-deltára kell normálni. Ebben a feladatban az energia és energia sajátállapot között egy az egyhez megfeleltetés van, ellenben a jól ismert szabad részecske esetével. Ennek oka, hogy itt $x\to\infty$-ből nem jönnek részecskék. Ennek következtében az a sajátállapotokat $\Ket{E}$ egyértelmen jelöli.
\Aeqref{nowall:boundary} feltétel azt jelenti, hogy az Airy-függvények közül a $\Bi(ax-bE)$ nem szerepel a lineáris kominációban, a megoldás tisztán az $\Ai(ax-bE)$ függvény lesz,
\begin{equation}
	\Braket{x|E}=N\Ai(ax-bE).
\end{equation}
A szórásállapotokra vonatkozó normálási feltétel
\begin{equation}
	\Braket{E|E^\prime}=\delta(E-E^\prime).
\end{equation}
Ez alapján $N$ meghatározható \eqref{airy:delta} azonosság felhasználásával,
\begin{dmath}
	\delta(E-E^\prime)=N^2\int_{-\infty}^\infty\Ai(ax-bE)\Ai(ax-bE^\prime)\,dx=N^2\frac{1}{ab}\delta(E-E^\prime).
	\label{nowall:orthog}
\end{dmath}
Ez alapján $N=\sqrt{ab}=\sqrt[3]{\frac{2m}{\hbar^2\sqrt{F}}}$, és
\begin{equation}
	\Braket{x|E}=\psi_E(x)=\sqrt{ab}\Ai(ax-bE).
	\label{nowell:sajátfüggvény}
\end{equation}
A teljességi reláció is leellenőrizhető \aeqref{airy:delta} egyenlet alapján,
\begin{dmath}
	\int_{-\infty}^\infty dE\,\Ket{E}\Bra{E}=ab\int_{-\infty}^\infty dE\int_{-\infty}^\infty dx\int_{-\infty}^\infty dy\,\Ai(ax-bE)\Ai(ay-bE)\Ket{x}\Bra{y}=\int_{-\infty}^\infty dx\int_{-\infty}^\infty dy\,\delta(x-y)\Ket{x}\Bra{y}=\op{I}.
	\label{nowall:comleteness}
\end{dmath}
\Aeqref{nowall:orthog} egyenlet a $\op{H}$ operátor hermitikusságából következik, hiszen a hermitikus operátorok sajátállapotai ortogonálisak egymásra. \Aeqref{nowall:comleteness} teljességi reláció is arra utal, hogy az összes fizikai sajátállapotot megtaláltuk a csupán $\Ai(x)$ függvényt tartalmazó állapotok keresésével. Ha hiányozna valamely fizikai állapot, akkor nem lehetne a megtalált sajátfüggvények lineáris kombinációjaként tetszőleges hullámfüggvényt előállítani, és így a teljességi reláció nem teljesülne.

Érdemes a fizikai intuícióval összevetni az Airy-függvény Fourier-transzformáltját. Az Airy-függény Fourier transzformáltja
\begin{equation}
	\int_{-\infty}^\infty\Ai(x)e^{-ikx}\,dx=e^{ik^3/3}.
\end{equation}
Ez azt jelenti, hogy az impulzus térben a hullámfüggvény
\begin{equation}
	\psi_E(p)=\frac{1}{\sqrt{2\pi F\hbar}}\exp\left(i\left(\frac{1}{3}\left(\frac{p}{a\hbar}\right)^3-\frac{pE}{F\hbar}\right)\right),
\end{equation}
\begin{equation}
	\rvert\psi_E(p)\lvert^2=\frac{1}{2\pi F\hbar}.
\end{equation}
Az impulzus hullámfüggvény amplitúdója nem függ az impulzustól! Ez nem meglepő, mert a klasszikus esetben az impulzus időfejlődése 
\begin{equation}
	p(t)=-Ft+p_0,
\end{equation}
tehát minden részecske egy kis $dp$ tartományban $dp/F$ időt tölt, adott impulzushoz tartozó részecskesűrűség értéke független az impulzustól. Ennek a klasszikus fizika beli megállapításnak a megfelelője, hogy $\lvert\psi_E(p)\rvert^2$ $p$-től független.







\section{Szemiklasszikus közelítés}
	\begin{equation}
		nh = \oint p \, dq = 
	\end{equation}
	$E/F < L$ esete:
	\begin{equation}
		2\int_0^{E/F}\sqrt{2m\left( E-Fx \right)}\,dx = -\frac{2}{3mF}\left(2m\left( E-Fx \right)\right)^{\frac{3}{2}}\bigg \rvert_0^{E/F} = \frac{4\sqrt{2m}E^{3/2}}{3F}
	\end{equation}
	\begin{equation}
		E_n = \left(\frac{3nhF}{4\sqrt{2m}}\right)^{2/3}
	\end{equation}
	$E/F > L$ esete:
	\begin{equation}
		-\frac{2}{3mF}\left(2m\left( E-Fx \right)\right)^{\frac{3}{2}}\bigg \rvert_0^{L} = \frac{4\sqrt{2m}}{3F}\left(E^{3/2} - \left(E - FL\right)^{3/2}\right) = nh
	\end{equation}
	$E \gg FL$ esetén a különbség az $E^{3/2}$ függvény deriváltjának segítségével helyettesíthető:
	\begin{equation}
		nh \approx 2\sqrt{2m}E^{1/2}L
	\end{equation}
	\begin{equation}
		E_n \approx \frac{n^2h^2}{8mL^2}
	\end{equation}
	
	\begin{figure}[H]
		\includegraphics[scale=1]{./figs/energiaszintkozelites.pdf}
		\caption[Szemiklasszikus energiaszintek]{Az ábra a szemiklasszikus energiaszinteket hasonlítja össze az egzakt energiaszintekkel. Ez az ábra is a $bE$ és $aL$ közötti relációt ábrázolja. A szemiklasszikus közelítés nagy kvantumszámok illetve $E \gg FL$ esetén pontos. Utóbbi oka, hogy ebben az esetben a potenciál elhanyagolható, és a potenciál nélküli végtelen potenciálgödör energiaszintjeit pedig a szemiklasszikus közelítés egzaktul megadja.}
	\end{figure}
	
	\begin{figure}[H]
		\includegraphics[scale=1]{./figs/infsquareenergia.pdf}
		\caption[Végtelen potenciálgödör energiaszintjei]{Az ábrán a végtelen potenciálgödör és az egzakt energiaszintek összehasonlítása látható. Ez csak az $E \gg FL$ esetben jó közelítés, a szemiklasszikus energiaszintek jóval pontosabbak.}
	\end{figure}


	\subsection{Szemiklasszikus energiaszintek}
		A dobozba zárt részecske esetében két esetet kell vizsgálni a szemiklasszikus energiaszintek meghatározásához. Az első eset, amikor az energia $E < FL$, tehát a fordulópont a második fal elérése előtt van. Ebben az esetben a Maslov index $\frac{3}{4}$ \cite{brack:semiclassical} (2.4.1 fejezet). Az $x=0$ fordulópontban a szemiklasszikus hullámfüggvény $\frac{\pi}{4}$ fázist vesz fel, az $x=E/F$ fordulópontban pedig $\frac{\pi}{2}$ fázist vesz fel,
\begin{equation}
	\left(n+\frac{3}{4}\right)h=\oint p\,dq=2\int_0^{E/F}\sqrt{2m\left( E-Fx \right)}\,dx=\frac{4\sqrt{2m}}{3F}E^{3/2}.
	\label{semiclassicallevels:e1}
\end{equation}
A második eset amikor $E > FL$, ekkor a fordulópontok $0$-ban és $L$-ben vannak, és a Maslov index $1$. Mind az $x=0$, mind az $x=L$ fordulópontban $\frac{\pi}{2}$ fázis vesz fel a szemiklasszikus hullámfüggvény,
\begin{equation}
	\left(n+1\right)h=\oint p\,dq=2\int_0^{L}\sqrt{2m\left(E-Fx\right)}\,dx=\frac{4\sqrt{2m}}{3F}\left(E^{3/2}-\left(E-FL\right)^{3/2}\right).
	\label{semiclassicallevels:e2}
\end{equation}
\begin{figure}[H]
	\centering
	\includegraphics[scale=1]{./figs/energiaszintkozelites.pdf}
	\caption[Szemiklasszikus energiaszintek]{Az ábra a szemiklasszikus energiaszinteket hasonlítja össze az egzakt energiaszintekkel. Ez az ábra is a $bE$ és $aL$ közötti relációt ábrázolja. A szemiklasszikus közelítés nagy kvantumszámok illetve $E \gg FL$ esetén pontos. Utóbbi oka, hogy ebben az esetben a potenciál elhanyagolható, és a potenciál nélküli végtelen potenciálgödör energiaszintjeit pedig a szemiklasszikus közelítés egzaktul megadja.}
	\label{semiclassicallevels:kozelites}
\end{figure}
Előfordulhat, hogy valamely $n$-re egyszerre van \eqref{semiclassicallevels:e1} és \eqref{semiclassicallevels:e2} egyszerre van megoldása, ahol $E$ a megfelelő tartományba esik. Ez azt jelenti, hogy a szemiklasszikus közelítés hibáján belül nem lehet meghatározni, hogy a valódi energiszint $FL$ felett, vagy alatt van. \Aref{semiclassicallevels:kozelites}. ábra az $E$-$L$ diagrammon szemlélteti a szemiklasszikus köelítés pontosságát. Két különböző esetben is pontos a szemiklasszikus közelítés. Nagy kvantumszámok esetében általánosságban is igaz, hogy pontos a szemiklasszikus közelítés. Ezen felül $E\gg FL$ esetében is pontos, ennek oka, hogy ilyenkor a lineáris potenciál elhanyagolható, viszont az így kapott problémát, a végtelen potenciálgödröt, a szemiklasszikus közelítés egzaktul írja le. \Aref{semiclassicallevels:allapotszam}. ábra szemlélteti a szemiklasszikus és egzakt állapotszámok viszonyát. A szemiklasszikus energiaszintekre vonatkozó egyenleteket minden esetben kézenfekvő az állapotok számának meghatározására használni, hiszen az egyenlet alapból $n$-re van rendezve a Maslov-indextől eltekintve. 
\begin{figure}[H]
	\centering
	\includegraphics[scale=1]{./figs/allapotszam.pdf}
	\caption[Szemiklasszikus állapotszám]{A szemiklasszikus és egzakt energiaszintek összevetése. A kék vonal az egzakt energiák által meghatározott állapotszám. A narancssárga vonal pedig \aeqref{semiclassicallevels:e1} és \aeqref{semiclassicallevels:e2} egyenletekből kifejezett $n$ az energia függvényében, $E$ és $FL$ relációjának megfelelően.}
	\label{semiclassicallevels:allapotszam}
\end{figure}
Amennyiben $E \gg FL$ \aeqref{semiclassicallevels:e2} egyenleten a különbség az $E^{3/2}$ függvény deriváltjának segítségével helyettesíthető,
\begin{equation}
	(n+1)h \approx FL\frac{d}{dE}\left(\frac{4\sqrt{2m}}{3F}E^{3/2}\right)=2\sqrt{2m}E^{1/2}L.
\end{equation}
Átrendezve az egyenletet energiára a megszokott végtelen potenciálgödör energiaszintjeit kapjuk,
\begin{equation}
	E_n \approx \frac{(n+1)^2h^2}{8mL^2}.
\end{equation}
Ezeket az energiaszinteket \aref{semiclassicallevels:squarewell}. ábra összeveti az $E$-$L$ diagrammon az egzakt energiaszintekkel.
\begin{figure}[H]
	\centering
	\includegraphics[scale=1]{./figs/infsquareenergia.pdf}
	\caption[Végtelen potenciálgödör energiaszintjei]{Az ábrán a végtelen potenciálgödör és az egzakt energiaszintek összehasonlítása látható. Ez csak az $E \gg FL$ esetben jó közelítés, a szemiklasszikus energiaszintek jóval pontosabbak.}
	\label{semiclassicallevels:squarewell}
\end{figure}
	\subsection{Összehasonlítás az egzakt eredménnyel}
		$x \rightarrow \infty$ aszimptotikus alak:
	\begin{equation}
		\Ai{-x} = \frac{1}{\sqrt{\pi}x^{1/4}}\cos\left(\frac{2}{3}x^{3/2} - \frac{\pi}{4}\right) + \mathcal{O}\left(x^{-5/4}\right)
	\end{equation}
	\begin{equation}
		\Bi{-x} = -\frac{1}{\sqrt{\pi}x^{1/4}}\sin\left(\frac{2}{3}x^{3/2} - \frac{\pi}{4}\right) + \mathcal{O}\left(x^{-5/4}\right)
	\end{equation}
	\begin{equation}
		\Ti{-x} = -\cot\left( \frac{2}{3}x^{3/2} - \frac{\pi}{4} \right) + \mathcal{O}\left(x^{-5/4}\right)
	\end{equation}
	
	Ezzel a közelítéssel \aref{box_energiaszintek_egyenlet}. egyenlet alakja:
	\begin{equation}
		\cot\left(\frac{2}{3}\left(b-aL\right)^{3/2} - \frac{\pi}{4}\right) = \cot\left(\frac{2}{3}b^{3/2} - \frac{\pi}{4}\right)
	\end{equation}
	, azaz
	\begin{equation}
		\frac{2}{3}b^{3/2} - \frac{2}{3}\left(b-aL\right)^{3/2} = n\pi
	\end{equation}
	. Az $a$ és $b$ behelyettesítésével az egyenlet
	\begin{equation}
		\frac{2\sqrt{2m}}{3F\hbar}\left(E^{3/2} - \left(E - FL\right)^{3/2}\right) = n\pi
	\end{equation}
	Ez megegyezik a szemiklasszikus kvantálással kapott eredménnyel, ami azt jelenti, hogy a szemiklasszikus közelítés jól működik nagy energiáknál, hibája $\mathcal{O}\left(E^{-5/4}\right)$ nagyságrendű.


	\subsection{Airy függvények aszimptotikája}
		Klasszikus mechanikai megfontolások alapján meghatározhatóak az Airy-függvények aszimptotikus alakjai, a pontos fázistól eltekintve. Ez nem meglepő, mert a hullámfüggvény amplitúdója a megtalálási valószínűséggel van kapcsolatban. A hullámfüggvény lokális közelítése egy síkhullámmal, vagyis a fázis deriváltja az impulzussal van kapcsolatban. Így a klasszikus mechanika alapján lehet a hullámfüggvény amplitúdójára és fázisára következtetni.

\Aref{nowall}. fejezetben leírt rendszert vizsgáljuk, $E=0$ választásával, azaz a klasszikus esetben a fordulópont $x=0$-ban van. Kvantum mechanika szerint a megtalálási valószínűség $|\psi|$-tel arányos, klasszikus mechanikában pedig a $dx$ tartományon való áthaladás idejével, $\frac{dx}{v}$-vel arányos. Mivel a kérdéses állapot szórásállapot, nem normálható. Ezért a valószínűségeknél csak arányosságról beszélhetünk, egy részecske rendszerre vonatkozó valószínűségsűrűségként nem értelmezhető. Egy lehetséges interpretáció a szórásállapotok esetében $|\psi|^2$-re, hogy nem kölcsönható részecske áramról van szó, és a résecskék sűrűsége $|\psi|^2$-tel arányos. A klasszikus esetben hasonló a helyzet, a $\frac{dx}{v}$ a részecskesűrűséggel arányos. A két módon kapott részecskesűrűség egyenlőségének feltételezésével a hullámfüggvény amplitúdójának viselkedését kapjuk,
\begin{equation}
	\frac{dx}{v}=\sqrt{-\frac{m}{2Fx}}dx\propto \lvert\psi(x)\rvert^2dx,
\end{equation}
a klasszikus mechanikából ismert energia megmaradás szerint. Átrendezve
\begin{equation}
	\psi(x)\propto\frac{1}{\sqrt[4]{-x}}.
\end{equation}
A hullámfüggvény fázisának meghatározása a de Broglie hullámhossz, $p=\hbar k$, és a klasszikus impulzus alapján történik. Abban az esetben, ha az amplitúdó ami közelítőleg megkapható az előző egyenletből, kicsit változik a de Broglie hullámhossz alatt,
\begin{equation}
	\psi(x)\propto\exp\left(i\int_{x_0}^xk\left(x^\prime\right)\,dx^\prime\right).
\end{equation}
A klasszikus energia megmaradás meghatározza az impulzust, ami alapján a de Broglie hullámszám
\begin{equation}
	k=\frac{\sqrt{2mF}}{\hbar}\sqrt{-x}.
\end{equation}
A $k$ integrálja könnyen kiszámítható,
\begin{equation}
	\int \frac{\sqrt{2mF}}{\hbar}\sqrt{-x}\,dx=\frac{2}{3}\left(-ax\right)^{3/2}.
\end{equation}
Az amplitúdóra és hullámhosszra vonatkozó feltételeket összekombinálva
\begin{equation}
	\psi(x)\propto\Ai(ax)\propto\frac{1}{\sqrt[4]{-ax}}\sin\left(\frac{2}{3}\left(-ax\right)^{3/2}\right).
\end{equation}






\section{Homogén tér Green-függvénye}
	A reolvens operátor definíciója
\begin{equation}
    \op{G}\left( E \right) = \frac{1}{\op{H} - E}
\end{equation}
és ezen operátorhoz tartozó két változós függvény a Green-függény.
\begin{equation}
    G\left( x, y; E \right) = \Bra{x}G\left(E\right)\Ket{y}
\end{equation}
A Green-függvény név indokolt, és ennek a segítségével fogom meghatározni a Green-függvényeket konkrét esetben. A teljességi reláció beszúrásával látható, hogy a kvantummechanikai Green-függény megegyezik a differenciálegyenletek elméletéből ismert Green-függvénnyel.
\begin{equation}
    \left(\op{H} - E\right) \op{G}\left( E \right) = \op{I}
\end{equation}

\begin{equation}
    \int \mathrm{d}x^\prime \Bra{x}\left(\op{H} - E\right) \Ket{x^\prime}\Bra{x^\prime} \op{G}\left( E \right)\Ket{y} = \Bra{x}\op{I}\Ket{y} = \delta \left(x - y\right)
\end{equation} 
A $\Bra{x}\left(\op{H} - E\right) \Ket{x^\prime}$ maggal vett konvolúció a $\op{H} - E$ operátor hatása. Ezért
\begin{equation}
    \left(\op{H}_x - E\right) G\left(x, y; E\right) = \delta\left(x - y\right)
    \label{green:deltaeq}
\end{equation}
ami a differenciálegyenletek elméletéből ismert Green-függvény definíciója. Ebben a konkrét esetben
\begin{equation}
    \left( -\frac{\hbar^2}{2m}\frac{\partial^2}{\partial x^2} + Fx - E \right) G\left(x, y; E\right) = \delta\left(x - y\right)
\end{equation}
ami azt jelenti, hogy az $x < y$ tartományban
\begin{equation}
    G\left(x, y; E\right) = C_1 \Ai{\sqrt[3]{\frac{2mF}{\hbar^2}}x - \sqrt[3]{\frac{2m}{\hbar^2F^2}}E} + C_2 \Bi{\sqrt[3]{\frac{2mF}{\hbar^2}}x - \sqrt[3]{\frac{2m}{\hbar^2F^2}}E}
\end{equation}
illetve az $x > y$ tartományban
\begin{equation}
    G\left(x, y; E\right) = C_3 \Ai{\sqrt[3]{\frac{2mF}{\hbar^2}}x - \sqrt[3]{\frac{2m}{\hbar^2F^2}}E} + C_4 \Bi{\sqrt[3]{\frac{2mF}{\hbar^2}}x - \sqrt[3]{\frac{2m}{\hbar^2F^2}}E}
\end{equation}
, ahol a $C$ együtthatók függhetnek $y$ és $E$ értékétől. A $C$ együtthatók meghatározásához a doboz eredeti határfeltételeit $x = 0$ és $x = L$ pontban, valamint az $x = y$ pontban \aref{green:deltaeq}. egyenlet $y$ körüli integrálásából kapot feltételeket kell felhasználni. 

















	\subsection{Egzakt Green-függvény}
		A Green-függvény név indokolt: a teljességi reláció beszúrásával látható, hogy a kvantummechanikai Green-függény megegyezik a differenciálegyenletek elméletéből ismert Green-függvénnyel.
\begin{equation}
    \left(E-\op{H}\right)\op{G}\left( E \right) = \op{I},
\end{equation}
azaz
\begin{equation}
    \int dx^\prime \Bra{x}\left(E-\op{H}\right) \Ket{x^\prime}\Bra{x^\prime} \op{G}\left( E \right)\Ket{y} = \Bra{x}\op{I}\Ket{y} = \delta \left(x - y\right).
\end{equation} 
A $\Bra{x}\left(E-\op{H}\right) \Ket{x^\prime}$ maggal vett konvolúció az $E-\op{H}$ operátor hatása, ezért
\begin{equation}
    \left(E-\op{H}_x\right) G\left(x, y; E\right) = \delta\left(x - y\right),
\end{equation}
amely a differenciálegyenletek elméletéből ismert Green-függvény definíciója. Ebben a konkrét esetben
\begin{equation}
    \left(E +\frac{\hbar^2}{2m}\frac{\partial^2}{\partial x^2} - Fx \right) G\left(x, y; E\right) = \delta\left(x - y\right),
	\label{green:deltaeq}
\end{equation}
amely azt jelenti, hogy az $x < y$ tartományban, illetve $y < x$ tartományban a Green-függvény a homogén egyenlet megoldása. A homogén megoldások illesztését az eredeti differenciálegyenlet határfeltételei, valamint az $x = y$ pontban \aeqref{green:deltaeq} egyenlet $y$ körüli integrálásából kapott feltételek határozzák meg. A doboz falára vonatkozó határfeltételek
\begin{equation}
	\left. G\left(x,y;E\right)\right\rvert_{x = 0} = 0,
	\label{green:01}
\end{equation}
\begin{equation}
	\left. G\left(x,y;E\right)\right\rvert_{x = L} = 0.
	\label{green:02}
\end{equation}
\Aref{green:deltaeq}. egyenlet $x$ szerinti integrálja $y$ körüli $\epsilon$ sugarú környezetében az $\epsilon \to 0^+$ határesetben
\begin{equation}
	\lim_{\epsilon \to 0^+}\left.\frac{\partial}{\partial x}G\left(x,y;E \right)\right\rvert_{x = y - \epsilon}^{x = y + \epsilon} = \frac{2m}{\hbar^2}.
	\label{egzakt:jump}
\end{equation}
Itt a jobb oldal integrálja $\left. \theta\left(x - y\right) \right\rvert_{x = y - \epsilon}^{x = y + \epsilon} = 1$ az előírt határesetben. Mivel $G(x,y;E)$-ről feltesszük, hogy folytonos, a bal oldal integrálja is folytonos, leszámítva a deriváltakat tartalmazó tagokat. A határeset elvégzése közben a deriváltakat nem tartalmazó tagok így kiesnek. \Aref{green:deltaeq}. egyenlet $\int_{y-\epsilon}^{y+\epsilon}dx^\prime \int_{y-\epsilon}^{x^\prime} \,dx$ integrálja az $\epsilon \to 0^+$ határesetben
\begin{equation}
	\lim_{\epsilon \to 0^+}\left.G\left(x,y;E \right)\right\rvert_{x = y - \epsilon}^{x = y + \epsilon} = 0
	\label{green:continuity}
\end{equation}
folytonossági feltételt adja. A jobb oldal integrálja $\left. \left(x - y\right) \theta\left(x - y\right) \right\rvert_{x=y-\epsilon}^{x=y+\epsilon}$, ami a határesetben $0$. Az $\left(Fx - E\right)G\left(x,y;E\right)$ integrálja is $0$ a határesetben, az előző integrálhoz hasonló módon.

Valós energiákra $G(x,y;E)=G(y,x;E)^*$. Ezt a szimmetria tulajdonságot fel lehet használni a Green-függvényre adott ansatz pontosítására az $x<y$ és $y<x$ $x$-$y$ csere szimmetriájának megkövetelésével. Ez automatikusan kielégíti \aeqref{green:continuity} egyenletet. A tartomány peremén a homogén megoldás eltűnését megkövetelve \aeqref{green:01} és \aeqref{green:02} teljesül. Érdemes bevezetni a
\begin{equation}
	\begin{aligned}
		u &= ax-bE,
		v &= ay-bE
	\end{aligned}
	\label{egzakt:uv}
\end{equation}
jelöléseket. A fent leírt három kritériumot és szimmetria tulajdonságot teljesítő ansatz a
\begin{equation}
	G\left(x,y;E\right) = C_0(E)\times
	\begin{cases}
		\begin{split}
			\Bigl(\Ti(aL-bE)\Bi(v)-\Ai(v)\Bigr)\times\\
			\Bigl(\Ti(-bE)\Bi(u)-\Ai(u)\Bigr)
		\end{split}& x \leq y\\
		\begin{split}
			\Bigl(\Ti(aL-bE)\Bi(u)-\Ai(u)\Bigr)\times\\
			\Bigl(\Ti(-bE)\Bi(v)-\Ai(v)\Bigr)
		\end{split}& x \geq y
	\end{cases}.
	\label{egzakt:ansatz}
\end{equation}
A $C_0(E)$ együtthatót úgy kell megválasztani, hogy \aeqref{egzakt:jump} egyenlet teljesüljön. \Aeqref{egzakt:jump} egyenletbe behelyettesítve \aeqref{egzakt:ansatz} egyenlet, és osztva $C_0(E)$-vel,
\begin{dmath}
	\frac{1}{C_0(E)}\frac{2m}{\hbar^2}=\frac{1}{C_0(E)}\lim_{\epsilon \to 0^+}\left.\frac{\partial G(x,y;E)}{\partial x}\right\rvert_{x=y-\epsilon}^{x=y+\epsilon}=a\lim_{\epsilon \to 0^+}\Bigl(-\Ti(aL-bE)\Bip(u)\Ai(v)-\Ti(-bE)\Aip(u)\Bi(v)+\Ti(aL-bE)\Bi(v)\Aip(u)+\Ti(-bE)\Ai(v)\Bip(u)\Bigr)=a\Bigl(\Ti(-bE)-\Ti(aL-bE)\Bigr)\Bigl(\Ai(v)\Bip(v)-\Aip(v)\Bi(v)\Bigr)=a\frac{\Ti(-bE)-\Ti(aL-bE)}{\pi}.
	\label{egzakt:c0calc}
\end{dmath}
A második egyenlőségnél kihasználtuk, hogy a $\Bi(v)\Bip(u)$-t és $\Ai(v)\Aip(u)$-t tartalmazó tagok kiesnek. A harmadik egyenlőségnél a határérérték kiértékelhető, az $\epsilon\to 0^+$ határesetben $u\to v$, így szorzat alakba írható az összeg. Végül a negyedik sorban a Wronski-determinánst használtuk fel, \eqref{airy:wronski} egyenletnek megfelelően. Az $a$ definíciója szerint $\frac{2m}{\hbar^2}=\frac{a^3}{F}$, így \eqref{egzakt:c0calc} átrendezésével
\begin{equation}
	C_0(E) = \frac{a^2}{F}\frac{\pi}{\Ti(-bE)-\Ti(aL-bE)}.
\end{equation}
Összesítve az eredményeket, a rendszer energiafüggő Green-függvénye
\begin{equation}
	G\left(x,y;E\right) = \frac{a^2}{F}\frac{\pi}{\Ti(-bE)-\Ti(aL-bE)}\times
	\begin{cases}
		\begin{split}
			\Bigl(\Ti(aL-bE)\Bi(v)-\Ai(v)\Bigr)\times\\
			\Bigl(\Ti(-bE)\Bi(u)-\Ai(u)\Bigr)
		\end{split}& x \leq y\\
		\begin{split}
			\Bigl(\Ti(aL-bE)\Bi(u)-\Ai(u)\Bigr)\times\\
			\Bigl(\Ti(-bE)\Bi(v)-\Ai(v)\Bigr)
		\end{split}& x \geq y
	\end{cases}
	\label{egzakt:greenfunction}
\end{equation}

\Aref{egzakt:1dgreens}. és \aref{egzakt:2dgreen}. ábra \aeqref{egzakt:greenfunction} Green-függvényt ábrázolja. A doboz mérete $aL=10$, és az energia, ahol a Green-függvény ki van értékelve $bE=5$.

\begin{figure}[H]
	\centering
	\includegraphics[scale=1]{./figs/1dgreens.pdf}
	\caption[Egy dimenziós Green-függvény]{}
	\label{egzakt:1dgreens}
\end{figure}

\begin{figure}[H]
	\centering
	\includegraphics[scale=0.65]{./figs/2dgreen.png}
	\caption[Két dimenziós Green-függvény]{}
	\label{egzakt:2dgreen}
\end{figure}

\Aeqref{green:greensum} egyenletnek megfelelően a Green-függvénynek pólusai vannak $E=E_n$-ben. Ezt \aeqref{egzakt:greenfunction} egyértelmen mutatja, mivel a nevezőjében \aeqref{box_energiaszintek_egyenlet} $0$-ra rendezett egyenlet bal oldala szerepel. Ennek az egyenletnek a gykei határozták meg az $E_k$ sajátenergiákat.

Egy érdekes matematikai következmény, hogy a Green-függvényre vonatkozó differenciál egyenlet megoldásával elvégeztük \aref{green:greensum}. egyenlet összegzését. Ez az összeg az Airy függvények szorzatának összege lenne, osztva $E-E_k$-val és a megfelelő $N_k$ normálási faktorral ahol $E_k$-t \aeqref{box_energiaszintek_egyenlet} transzcendens egyenlet határoz meg. A Green-függvényre vonatkozó differenciálegyenlet ismerete nélkül az összeg elvégzése reménytelennek látszana.
	\subsection{Green-függvény határesetei}
		A két falú doboz Green-függvényéből megfelelő határesetekben előállítható más fizikai rendszerek Green-függvénye is. Például az $L\to\infty$ határeset visszaadja a felül nyitott doboz Green-függvényét, avagy a földön pattogó kvantum részecske ("quantum bouncer") Green-függvényét. Egy következő transzformáció határeseteként megkaphatjuk a falak nélküli végtelen lineáris potenciálban mozgó részecske Green-függvényét. Ehhez mind a helykoordinátát, mind az energiát meg kell változtatni: $x\to x^\prime=x+d$, $y\to y\prime=y+d$ és $E\to E^\prime=ˇE+Fd$, végül a $d\to\infty$ határesetet kell venni.

Az $L\to\infty$ határeset könnyen elvégezhető. \Aeqref{airy:ai+approx} és \aeqref{airy:bi+approx} egyenletek szerint $\Ti(aL-bE)$ gyorsan $0$-hoz tart. Ezt az eredményt felhasználva az $x=0$-ban fallal bezárt részecske Green-függvénye $=Fx$ potenciálban
\begin{equation}
	G_{egy fal}\left(x,y;E\right) = -\frac{a^2}{F}\frac{\pi}{\Ti(-bE)}\times
	\begin{cases}
		\Ai(v)\Bigl(\Ti(-bE)\Bi(u)-\Ai(u)\Bigr)& x \leq y\\
		\Ai(u)\Bigl(\Ti(-bE)\Bi(v)-\Ai(v)\Bigr)& x \geq y
	\end{cases}.
	\label{limits:semiinfinite}
\end{equation}

A következő határesetet valamivel nehezebb kiszámítani. Ezt előre lehet sejteni, mert az eddigi Green-függvények olyan rendszereket írtak le, ahol minden állapot kötött állapot. A falak nélküli lineáris potenciálhoz nem tartoznak kötött állapotok, csak szórásállapotok vannak. Ez a változás megmutatkozik a Green-függvény pólusszerkezetében, utalva arra, hogy ez a határeset jelentősen megváltoztatja a Green-függvényt matematikai értelemben is. A feljebb említett átmenet,
\begin{equation}
	\begin{aligned}
		x^\prime&=x+d\\
		y^\prime&=y+d\\
		E^\prime&=E+Fd\\
		d       &\to\infty
	\end{aligned}.
	\label{limits:transitiontonowall}
\end{equation}
E az átmenet eltolja a helykoordinátát, miközben a részecske kinetikus energiáját, változatlanul tartja. Az $u$ $v$ változók értéke \eqref{egzakt:uv} egyenlet szerint változatlan marad, a $d\to\infty$ határérték nem változtatja az alakjukat. Mivel a falak nélküli rendszernek az egész valós energiatengely a spektruma, a Green-függvényt az $E^\prime=E+Fd\pm i\epsilon$ energiában vizsgáljuk, a $\Ti(-bE^\prime)$ viselkedését kell meghatározni nagy $E^\prime$ esetén. Felhasználva \aeqref{airy:tiapprox} egyenletet
\begin{dmath}
	\Ti(-x-i\epsilon)
	\approx-\frac{\cos\left(\frac{2}{3}(x+i\epsilon)^{3/2}-\frac{\pi}{4}\right)}{\sin\left(x+i\epsilon)^{3/2}-\frac{\pi}{4}\right)}
	\approx-\frac{\cos\left(\frac{2}{3}x^{3/2}+i\sqrt{x}\epsilon-\frac{\pi}{4}\right)}{\sin\left(\frac{2}{3}x^{3/2}+i\sqrt{x}\epsilon-\frac{\pi}{4}\right)}
	=-\frac{\cos\left(\frac{2}{3}x^{3/2}-\frac{\pi}{4}\right)\cosh\left(\sqrt{x}\epsilon\right)-i\sin\left(\frac{2}{3}x^{3/2}-\frac{\pi}{4}\right)\sinh\left(\sqrt{x}\epsilon\right)}{\sin\left(\frac{2}{3}x^{3/2}-\frac{\pi}{4}\right)\cosh\left(\sqrt{x}\epsilon\right)+i\cos\left(\frac{2}{3}x^{3/2}-\frac{\pi}{4}\right)\sinh\left(\sqrt{x}\epsilon\right)}
	=-\frac{\cos\left(\frac{2}{3}x^{3/2}-\frac{\pi}{4}\right)-i\sin\left(\frac{2}{3}x^{3/2}-\frac{\pi}{4}\right)\tanh\left(\sqrt{x}\epsilon\right)}{\sin\left(\frac{2}{3}x^{3/2}-\frac{\pi}{4}\right)+i\cos\left(\frac{2}{3}x^{3/2}-\frac{\pi}{4}\right)\tanh\left(\sqrt{x}\epsilon\right)}
	\approx-\frac{\cos\left(\frac{2}{3}x^{3/2}-\frac{\pi}{4}\right)-i\sin\left(\frac{2}{3}x^{3/2}-\frac{\pi}{4}\right)\sgn\left(\epsilon\right)}{\sin\left(\frac{2}{3}x^{3/2}-\frac{\pi}{4}\right)+i\cos\left(\frac{2}{3}x^{3/2}-\frac{\pi}{4}\right)\sgn\left(\epsilon\right)}.
\end{dmath}
A sorok közötti lépésekhez felhasználtuk a $(x+a)^\alpha\approx x^\alpha + \alpha x^{\alpha-1}a$ közelítést, a trigonometrikus addíciós képleteket, a képzetes argumentumú trigonometrikus függvények és hiperbolikus függvények kapcsolatát, valamint az előel függvény közelítését a $\tanh$ függvénnyel. Ezek a közelítések egzaktak az $x\to\infty$ határesetben, ezért
\begin{equation}
	\lim_{x\to\infty}\Ti(-x-i\epsilon)=
	\begin{cases}
		i &\epsilon > 0\\
		-i&\epsilon < 0
	\end{cases}.
	\label{limits:ti}
\end{equation}
Ez az eredmény kellett ahhoz, hogy \aeqref{limits:transitiontonowall} átmenet alapján meghatározzuk a fal nélküli lineáris $V=Fx$ potenciálhoz tartozó Green-függényt. Ha $\Im(E)>0$
\begin{dmath}
	G_{nincs fal}(x,y;E)=\lim_{d\to\infty}G_{egy fal}(x+d,y+d;E+Fd)=\frac{\pi a^2}{F}\times
	\begin{cases}
		\Ai(v)\Bigl(\Bi(u)-i\Ai(u)\Bigr)&x\leq y\\
		\Ai(u)\Bigl(\Bi(v)-i\Ai(v)\Bigr)&x\geq y
	\end{cases}.
	\label{limits:nowallgreen1}
\end{dmath}
Ha $\Im(E)<0$, akkor \aeqref{limits:ti} egyenlet $-i$ a limeszben, így
\begin{dmath}
	G_{nincs fal}(x,y;E)=\lim_{d\to\infty}G_{egy fal}(x+d,y+d;E+Fd)=\frac{\pi a^2}{F}\times
	\begin{cases}
		\Ai(v)\Bigl(\Bi(u)+i\Ai(u)\Bigr)&x\leq y\\
		\Ai(u)\Bigl(\Bi(v)+i\Ai(v)\Bigr)&x\geq y
	\end{cases},
	\label{limits:nowallgreen2}
\end{dmath}
ez a kifejezés csak az $i$ előjelében különbözik az előzőtől. Az egész valós tengely mentén ugrása van ennek a Green-függvénynek a képzetes részének. Ez egybevág azzal a korábbi eredménnyel hogy tetszőleges energiájú sajátállapotai lehetnek a fal nélküli rendszernek, mert a Green-függvénynek vágása van a folytonos spektrumhoz tartozó energiák mentén.












	\subsection{Állapotsűrűség}
		Ahogy azt a Green-függvények bevezetésénél említettük, alkalmasak a (lokális) állapotsűrűség meghatározására \cite[7. o.]{economou2006green},
\begin{equation}
	\rho\left(E\right) = -\frac{1}{\pi}\lim_{\epsilon \to 0^+} \Im\Tr\op{G}\left(E + i\epsilon\right),
	\label{green:densityeq}
\end{equation}
\begin{equation}
	\rho(x,E)=-\frac{1}{\pi}\lim_{\epsilon\to 0^+}\Im G(x,x,E+i\epsilon).
	\label{green:localdensityeq}
\end{equation}
$\rho(E)\,dE$ az állapotok száma egy $dE$ energiatartományban, az állapotsűrűség. $\rho(x,E)\,dE\,dx$ pedig a megtalálási valószínűséggel súlyozott állapotok száma $dx$ intervallumban $dE$ energiatartományban, az úgynevezett lokális állapotsűrűség.

Ezeket a formulákat numerikus módon közelítőleg ki lehet értékelni kicsi, de véges $\epsilon$ választásával, ezt szemlélteti \aref{green:állapotsűrség}. ábra.
\begin{figure}[H]
	\centering
	\includegraphics[scale=1]{./figs/dosfromgreen.pdf}
	\caption[Állapotsűrűség]{\Aref{green:densityeq}. képlet alapján számolt állapotsűrűség. A kék függvényt $\epsilon = 10^{-3}/b$, a narancssárga görbét pedig $\epsilon = 10^{-2}/b$ helyettesítéssel kaptuk. Látható, hogy $\epsilon$ csökkentésével a tüskék egyre keskenyebbek, és egyre magasabbak lesznek.}
	\label{green:állapotsűrség}
\end{figure}
Ennek a közelítésnek egy jó tulajdonsága, hogy a formula származtatásához a jól ismert
\begin{equation}
	\frac{1}{x\pm i\epsilon} = \frac{1}{x}\mp i\pi\delta(x)
\end{equation}
formulát lehet használni. Ennek a formulának a levezetése során a $\delta(x)$ állandó területű, de egyre szűkebb Lorentz-görbék határértékeként bukkan fel. Ez azt jelenti, hogy véges $\epsilon$ esetén is a sajátenergiákhoz tartozó csúcsok alatti terület változatlan, az állapotsűrűség $E$ szerinti integrálja nagy $E$-k és véges $\epsilon$ esetén is pontos marad.
\begin{figure}[H]
	\centering
	\includegraphics[scale=1]{./figs/numberofstatesfromgreen.pdf}
	\caption[Állapotok száma]{\Aref{green:állapotsűrség}. ábrán bemutatott függvények integrálja látható ezen az ábrán. Mind a két függvény ugrása közelítőleg $1$, ami at jelenti, hogy \aref{green:állapotsűrség}. ábrán látható tüskék alatti terület jó közelítéssel $1$. Az $\epsilon$ csökkentése a lépcsőfüggvényhez közelíti az integrált függvényt, ami egyezik az elvárásokkal.}
\end{figure}
\Aeqref{limits:semiinfinite} Green-függvényhez tartozó állapotsűrűség kvalitatíve nem különbözik az előző számítás menetétől és eredményétől, hiszen az előzőhöz hasonlóan csak diszkrét sajátenergiák vannak, ezeknek csupán az értékük különböző.

Más a helyzet \aeqref{limits:nowallgreen1}, \eqref{limits:nowallgreen2} Green-függvénnyel. Itt csak folytonos spektrumba tartozó sajátenergiák vannak, mind szórásállapotokhoz tartoznak. Ebben az esetben csak a lokális állapotsűrűséget lehet értelmezni, hiszen a sajátállapotok négyzetének integrálja végtelen, csak Dirac-deltára normálhatóak. \Aeqref{green:localdensityeq} egyenletnek megfelelően a határérték kiszámításához a pozitív képzetes részre vonatkozó \eqref{limits:nowallgreen1} kifejezést kell használni,
\begin{dmath}
	\rho(x,E)=-\frac{1}{\pi}\lim_{\epsilon\to 0^+}\Im\left\lbrace\frac{a^2\pi}{F}\Ai(ax-b(E+i\epsilon))\Bigl(\Bi(ax-b(E+i\epsilon))-i\Ai(ax-b(E+i\epsilon))\Bigr)\right\rbrace=\frac{a^2}{F}\Ai^2(ax-bE).
\end{dmath}
Nem meglepő módon ez az $E$ energiájú sajátállapot abszolútérték négyzete \eqref{nowell:sajátfüggvény}. A nomálási faktor is egyezik, hiszen $\frac{a^2}{F}=ab$.





















	\subsection{Perturbációszámítás}
		A perturbációszámítás a Green-függvény egyik legjelentősebb alkalmazása. Ebben a részben a Green-függvény perturációs sorának a konvergencia tulajdonságait vizsgáljuk. A konvergencia tartományát és sebességét befolyásolja a perturbáló operátor triviális módosítása, konkrétan a vizsgált példában az $\frac{FL}{2}\op{I}$ operátort a perturbáló tagból levonjuk és a perturálatlan operátorhoz hozzáadjuk. Ezzel a teljes Hamilton operátor nem változik, de a perturbációs sor konvergenciája igen.

A perturbációszámításhoz a Hamilton operátort két részre bontjuk,
\begin{equation}
	\op{H} = \op{H}_0 + \op{V}.
\end{equation}
A $\op{H}_0$ operátorhoz tartozó rezolvens operátor $\op{G}_0\left(E\right)$. Mind $\op{H}$ és mind $\op{H}_0$ kifejezhetőek a rezolvenseikkel, ha ezeket behelyettesítjük a fenti egyenletbe, implicit egyenletet kapunk $\op{G}\left(E\right)$-re nézve,
\begin{equation}
	-\op{G}^{-1}\left(E\right) - E = -\op{G}_0^{-1}\left(E\right) - E + \op{V}.
\end{equation}
Ezt kisebb átalakítások után fel lehet használni perturbációszámításra. Az egyenletet balról $\op{G}_0\left(E\right)$-vel, jobbról $\op{G}\left(E\right)$-vel szorozzuk, így
\begin{equation}
	\op{G}\left(E\right) = \op{G}_0\left(E\right) + \op{G}_0\left(E\right)\op{V}\op{G}\left(E\right)
	\label{green:pertmaster}
\end{equation}
eredményhez jutunk. Megfelelően definiálva $\op{G}_n\left(E\right)$ operátorokat,
\begin{equation}
	\op{G}_n\left(E\right) = \op{G}_0\left(E\right)\sum_{k=0}^n\left(\op{V}\op{G}_0\left(E\right)\right)^k,
\end{equation}
a $\op{G}_n$-ekre \aeqref{green:pertmaster} egyenlethez hasonló rekurziós összefüggés áll fent,
\begin{equation}
	\op{G}_{n+1}\left(E\right) = \op{G}_0\left(E\right) + \op{G}_0\left(E\right)\op{V}\op{G}_n\left(E\right).
	\label{perturbation:rekurzió}
\end{equation}
Ha $\norm{\op{V}\op{G}_0\left(E\right)} < 1$ akkor a $\op{G}_n$ sorozat konvergál. Operátor normának a Hilbert-tér normája által indukált normát vesszük, így az operátorok konvergenciája kompatibilis a Hilbert-tér beli konvergenciával.
\begin{equation}
	\norm{\op{A}}=\sup \Set{\left\lvert \op{A}\Ket{\phi} \right\rvert \texttt{, ahol} \Braket{\phi|\phi}=1}.
\end{equation}
A sor határértéke \aeqref{perturbation:rekurzió} miatt kielégíti \aeqref{green:pertmaster} egyenletet. Így konvergencia esetén
\begin{equation}
	\op{G}\left(E\right) = \op{G}_0\left(E\right)\sum_{n=0}^\infty\left(\op{V}\op{G}_0\left(E\right)\right)^n.
\end{equation}
Ez azt jelenti, hogy ha egy operátornak van projektor felbontása, akkor a normája a legnagyobb sajátérték abszolút értéke lesz, vagy általános esetben a sajátértékek szuprémuma. Ez hasznos jelen esetben is, mivel így meg tudjuk határozni az $\op{V}=a\op{x}+b$ operátor normáját. Ennek az operátornak a sajátfüggvényei a $\delta(x-x_0)$ függvények, így a sajátértékek maximuma a $[0,L]$ tartományban
\begin{equation}
	\norm{\op{V}}=\max(\rvert b \lvert, \lvert aL+b\rvert).
\end{equation}
\Aeqref{green:greensum} egyenlet alapján $\op{G}(E)$ normája is meghatározható, az összeg nevezői közül kiválasztva a legkisebb abszolút értékűt,
\begin{equation}
	\norm{\op{G}(E)}=\frac{1}{E-E_k},
\end{equation}
ahol $E$-hez a komplex síkon a legközelebbi sajátérték $E_k$. Ezek segítségével felső korlátot lehet adni a $\norm{\op{G}_1\op{V}}$-re.

A Hamilton-operátort eredetileg
\begin{equation}
	\begin{aligned}
		\op{H}=&\frac{\op{p}^2}{2m}+F\op{x}=\op{H}_0+\op{V}_1\\
		\op{H}_0=&\frac{\op{p}^2}{2m}\\
		\op{V}_1=&F\op{x}
	\end{aligned}
\end{equation}
részekre bontottuk. $\op{G}_0$ a $\op{H}_0$ operátor Green-függvénye. Ebben ha az esetben a $\op{G}_0$ pólusaitól legalább $FL$ távolságban, azaz $\rvert E-E_k\lvert>FL$, a komplex energia síkban a sor garantáltan konvergál, mert
\begin{equation}
	\norm{\op{G}_0(E)V_1}<\norm{\op{G}_0}\norm{(E)V_1}=\frac{FL}{\rvert E-E_k\lvert}<1.
\end{equation}
Vizsgálunk egy módosított felontást is,
\begin{equation}
	\begin{aligned}
		\op{H}=&\frac{\op{p}^2}{2m}+F\op{x}=\op{H}_0+\frac{FL}{2}+\op{V}_2\\
		\op{H}_0=&\frac{\op{p}^2}{2m}\\
		\op{V}_2=&F\op{x}-\frac{FL}{2}.
	\end{aligned}
\end{equation}
A perturbációszámítás során így a perturbálatlan operátor szerepét a $\op{H}_0+\frac{FL}{2}$ operátor tölti be. Ennek a Green-függvénye
\begin{equation}
	\frac{1}{E-\left(\op{H}_0+\frac{FL}{2}\right)}=\op{G}_0(E-\frac{FL}{2}),
\end{equation}
könnyen kifejezhető az eredeti eset Green-függvényével.

A $\frac{\op{p}^2}{2m}$ Green-függvénye a $[0,L]$ tartományban \aref{egzakt}. fejezethez hasonlóan meghatározható,
\begin{equation}
	G_0\left(x,y;E\right) = -\frac{\hbar}{\sqrt{2m}}\frac{1}{\sin\left(kL\right)}\times
	\begin{cases}
		\sin\left(k\left(y-L\right)\right)\sin\left(kx\right) & x\leq y\\
		\sin\left(k\left(x-L\right)\right)\sin\left(ky\right) & x\geq y\\
	\end{cases},
\end{equation}
\begin{equation}
	k = \frac{\sqrt{2mE}}{\hbar}.
\end{equation}

\begin{figure}[H]
	\centering
	\includegraphics[scale=1]{./figs/convergencerate.pdf}
	\caption[A Green-függvény perturbációs sorának konvergencia sebessége]{ye, awesome}
\end{figure}
\begin{figure}[H]
	\centering
	\includegraphics[scale=1]{./figs/convergenceOriginal.pdf}
	\includegraphics[scale=1]{./figs/convergenceImproved.pdf}
	\caption[A Green-függvény perturbációs sorának konvergenciatartománya]{Ez az ábra a két perturbációs sor konvergenciáját hasonlítja össze a komplex energia síkon. A felső ábra a $V=Fx$ perturbáló potenciálnak, míg az alsó a $V = Fx-FL/2$ perturbáció szerinti sornak felel meg. A fekete tartományok divergenciát jelölnek, míg a többi szín a sorfejtés tagjainak csökkenési sebességét jellemzik, a norma harmadolásához szükséges lépések számát megadva. A piros körökön kívüli tartomány a ?? formula által garantált konvergencia tartományát jelöli. A piros x-ek a $\hat{G}_0$ pólusait, a sárga x-ek pedig az egzakt $\hat{G}$ operátor pólusait jelölik.}
\end{figure}
\section{Összegzés}
	A dolgozat első részében visszavezettük a három dimenziós időfüggő problémát egy dimenziós időfüggetlen problémákra, majd analitikus képleteket adtunk az egy dimenziós probléma megoldásaira. Ezeket az egzakt képleteket összevetettük a szemiklasszikus közelítés eredményével, és a formulák fizikai interpretációját diszkutáltuk. Explicit analitikus képletet vezettünk le az időfüggetlen Green-függvényre, a pólusszerkezetét összevetettük az első részben kapott energiaszinteket meghatározó transzcendens egyenlettel. Szemléltettük a Green-függvény alkalmazhatóságát az állapotsűrűség numerikus illetve analitikus meghatározására is.

Egy konkrét példán bemutattuk, hogy a Hamiton-operátor önkényes felbontása perturbálatlan Hamilton-operátorra és perturbáló operátorra nagy mértékben befolyásolja a perturbációs sor konvergencia tulajdonságait. A példánkban a perturbáló operátor normájának minimalizálása egy triviális tag levonásával jelentősen javította a perturbációs sor konvergenciáját. Több részecske rendszereket leíró Green-függvények perturbációszámítása hatalmas jelentőséggel bír, számos fizikai témakör egyik fő eszköze, így a jövőben érdemes megvizsgálni, hogy milyen lehetőség van esetleg triviális tagok levonásával módosított perturbáció szerinti sorfejtés optimalizálására.
\appendix
\section{Szabad részecske gyorsuló koordinátarendszerben}
	\input{tex/accelerating.tex}
\section{Numerikus számítások}
	\subsection{Momentumok időfejlődése}
		\begin{figure}[H]
	\includegraphics[scale=1]{./figs/expectations.pdf}
	\caption{Várható értékek és szórások időfejlődése}
\end{figure}

	\subsection{Hullámfüggvény időfejlődése}
		\subsubsection{1D}
		\subsubsection{2D}
	
    \newpage
	\phantomsection
	\bibliographystyle{abeld}
	\addcontentsline{toc}{section}{Hivatkozások}
    \bibliography{tex/ref}
\end{document}















